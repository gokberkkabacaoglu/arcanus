\documentclass{article}
\usepackage{amsmath,amssymb}
\usepackage{hyperref}
\usepackage{bigints}
\title{\bf{Stabilization for Navier-Stokes with Boussinesq Buoyancy}}
\author{Nicholas Malaya \\ Institute for Computational Engineering and Sciences \\ University of Texas at Austin} \date{}

\begin{document}
\maketitle

%
% intro
%
Our process is the following: 

\begin{itemize}
 \item Strong form Navier Stokes + Bouss
 \item Cast into weak form
 \item Prepare as an operator $Lu=f$
 \item Calculate Fr\'echet derivative
 \item Separate into differential (P) and constant (Z) components,
       $L'[u] = P + Z$
 \item Choose stabilization operator such that $S = -P^*$
 \item Then stabilization has form, $a_h(u,\phi) = a(u,\phi) + \langle
       Lu,S\phi \rangle_\tau$
\end{itemize}

This is essentially the least-squares stabilization proposed by Hughes
and extended to natural convection by Becker and Braack. 

%
% start the real work
%
\newpage

\subsection{Navier-Stokes}

We begin with the incompressible Navier-Stokes equations with Bousinesq
buoyancy,
\begin{align}
 \frac{\partial u}{\partial t} + u \cdot \nabla u &= -\frac{1}{\rho}
 \nabla p + \nu \nabla^2 u + g \frac{T'}{T_0} \\
 \nabla \cdot u &= \, 0 \\
 \rho c_p \frac{\partial T}{\partial t} + u \cdot \nabla T &= \nabla
 \cdot (k \nabla T)
\end{align}
e.g. the momentum, continuity and energy equations, respectively. In
order to cast these into weak form we multiply by appropriate test
functions $\phi = \left[v,w,q\right] \in H^1_0(\Omega)$ and integrate over
the domain, $\Omega \in \mathbb{R}^n$. Our system of equations now
appears as, 
\begin{align}
 yar
\end{align}

Note that both the pressure term as well as the viscous term were
integrated by parts to reduce the required order of the solution on those
state variables. 

We define the inner product by the shorthand notation $(u,v) =
\bigintsss_\Omega u\cdot v dx $, giving our equations the form,  
\begin{align}
 (\dot u,v) + ((u \cdot \nabla) u, v) - (p,\nabla \cdot v) + \nu (\nabla
 u, \nabla v) = (g \frac{T'}{T_0},v)
\end{align}


Lagrangian is therefore,
$\mathcal{L}$

\end{document}
