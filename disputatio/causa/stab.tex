\documentclass{article}
\usepackage{amsmath,amssymb}
\usepackage{hyperref}
\usepackage{bigints}
\title{\bf{Stabilization for Navier-Stokes with Boussinesq Buoyancy}}
\author{Nicholas Malaya \\ Institute for Computational Engineering and Sciences \\ University of Texas at Austin} \date{}

\begin{document}
\maketitle

\section{Stabilization Derivation}
%
% intro
%
Our process is the following: 

\begin{itemize}
 \item Strong form Navier Stokes + Bouss
 \item Cast into weak form
 \item Prepare as an operator $Lc=f$
 \item Calculate Fr\'echet derivative
 \item Separate into differential (P) and constant (Z) components,
       $L'[c] = P + Z$
 \item Choose stabilization operator such that $S = -P^*$
 \item Then stabilization has form, $a_h(c,\phi) = a(c,\phi) + \langle
       Lc,S\phi \rangle_\tau$
\end{itemize}

This is essentially the least-squares stabilization proposed by Hughes
and extended to natural convection by Becker and Braack. 

%
% start the real work
%
\newpage

\subsection{Weak Formulation of Equations of Interest}

We begin with the incompressible Navier-Stokes equations with Bousinesq
buoyancy,
\begin{align}
 \nabla \cdot u &= \, 0 \label{eq_cont}\\
 \frac{\partial u}{\partial t} + u \cdot \nabla u &= -\frac{1}{\rho}
 \nabla p + \nu \nabla^2 u + g \frac{T'}{T_0} \label{eq_mom}\\
 \rho c_p \frac{\partial T}{\partial t} + u \cdot \nabla T &= \nabla
 \cdot (k \nabla T) \label{eq_energy}
\end{align}
e.g. the continuity, momentum and energy equations, respectively. Our 
state vector $c =  \left[p,u,T \right]$ In order to cast these into weak
form we multiply by appropriate test 
functions $\phi = \left[q,v,w \right] \in H^1_0(\Omega)$ and integrate over
the domain, $\Omega \in \mathbb{R}^n$. Our system of equations now
appears as, 
\begin{align}
  \bigintsss_\Omega q \nabla \cdot u \, dx &= 0 \\
 \bigintsss_\Omega \dot u \cdot v \, dx +
 \bigintsss_\Omega  (u \cdot \nabla) \, u \cdot v \, dx &=
 \bigintsss_\Omega \frac{p}{\rho} \nabla \cdot v \, dx - \nu \bigintsss_\Omega \nabla u \cdot \nabla v
 \,dx + \bigintsss_\Omega g \frac{T'}{T_0} \cdot v \, dx \\ 
 \rho c_p \bigintsss_\Omega \dot T \cdot w \, dx + \bigintsss_\Omega (u
 \cdot \nabla) T \cdot w \, dx  &= -\bigintsss_\Omega (k \nabla T) \cdot
 \nabla w \, dx
\end{align}

where an ``over-dot'' denotes time diffentiation, e.g. $\dot u =
\frac{\partial u}{\partial t}$. Note that both the pressure term as well
as the viscous term were integrated by parts to reduce the required
order of the solution on those state variables.  

We define the inner product by the shorthand notation $(u,v) =
\bigintsss_\Omega u\cdot v dx $, giving our equations the form,  
\begin{align}
 (\nabla \cdot u, q) &= 0 \\
 (\dot u,v) + (u \cdot \nabla u, v) - (p,\nabla \cdot v) + \nu (\nabla
 u, \nabla v) &= (g \frac{T'}{T_0},v) \\
 \rho c_p (\dot T,w) + (u \cdot \nabla T,w) + (k \nabla T,\nabla w) &= 0.
\end{align}

This defines our weak form operator $a(c,\phi)$. Our full equations will
also include a stabilization term such that,  
\begin{equation}
 a_h(c,\phi) = a(c,\phi) +  \langle Lc,S\phi \rangle_\tau
\end{equation}

The subsequent section will define the operators L and S. 

%
% subsection
%
\subsection{Tau Stabilization and the operators L and S}


To form the stabilization terms, 

\begin{equation}
 \langle Lc,S\phi \rangle_\tau
\end{equation}
we must define the operators L and S. The operator L is simply the PDEs
in Equation \ref{eq_cont}-\ref{eq_energy} written in operator form. S is
defined as the negative adjoint of the differential terms in L, e.g.
\begin{align}
 L'[c] = P + Z \\
 S = -P^*. 
\end{align}
Where P are the differential terms, and Z the constant terms. 

Our objective is now to construct the adjoint operator of L. This is
accomplished using the Fr\'echet derivative, which defines the
functional derivative on L. In general this is accomplished by taking
the first variation of a function $\Pi(u)$ around a base state, $u$,
\begin{equation}
 \delta\, \Pi(u) = \lim_{\epsilon \to 0} \frac{\Pi(u+\epsilon \hat u) -
  \Pi(u)}{\epsilon} =
  \frac{\partial \Pi(u +\epsilon \hat u)}{\partial \epsilon}
  \bigg|_{\epsilon = 0}
\end{equation}
for all $\hat u$ and $\epsilon > 0$ with $u + \epsilon \hat u \in
H^1_0(\Omega)$.

We not consider the 
The first variation of state for the
momentum equation is, 

%Lagrangian is therefore, $\mathcal{L}$

\end{document}
