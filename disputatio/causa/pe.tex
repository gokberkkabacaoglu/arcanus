\documentclass{article}
\usepackage{amsmath,amssymb}
\usepackage{bm}

\newcommand{\uveci}{{\bm{\hat{\textnormal{\bfseries\i}}}}}
\newcommand{\uvecj}{{\bm{\hat{\textnormal{\bfseries\j}}}}}
\newcommand{\uveck}{{\bm{\hat{\textnormal{\bfseries\k}}}}}

\title{\bf{Order of Magnitude Analysis of Energy Flux into the SoV}}
\author{Nicholas Malaya and Robert D. Moser\\ Institute for Computational Engineering and Sciences \\ University of Texas at Austin} \date{}

\begin{document}
\maketitle

This document details an attempt to provide a very rough estimate of the
total energy flowing into the Solar Vortex apparatus. This information
is desired to provide a context to the energy fluxes measured through
the top of the device. 

At present, we consider only the energy flowing into the device due to
the ambient conditions, in particular, the incoming wind and heat
through the front hemisphere of the circular device. We consider a large
(3m radius) device incoming freestream velocity of 5 m/s. The surface
temperature is 343 Kelvin, with a specified inflow boundary layer
bridging the ground temperature to the ambient air conditions of 313
Kelvin. These numbers were chosen based on information provided by the
field team in Arizona during the summer of 2014. 

There are two forms of energy to consider: kinetic and enthalpy. We
begin by considering the kinetic energy flux through the front of the
apparatus. 

\section*{Kinetic Energy Flux}

From the first law of thermodynamics we can express the kinetic energy
flux as a surface integral over the upstream face of the device, 
\begin{equation*}
\int_{CS} \frac{\vec V^2}{2} \rho \vec V \cdot \hat n dA.
\end{equation*}
%
% could cite fluid dynamics book here
% pg. 239
%

Where, $\vec V = u \uveci + v \uvecj + w \uvecj.$ 
We assume our freestream velocity has no components in y and z, 
$\vec V = u \uveci + 0 \uvecj + 0 \uvecj.$ The surface of interest, S,
is a perfect cylinder, and therefore has a surface element of, 
\begin{equation*}
dS = Rd\theta dz. 
\end{equation*}
An outward pointing vector from this surface has the form, 
\begin{equation*}
x \uveci + y \uvecj. 
\end{equation*}
The normal vector is then,
\begin{equation*}
\hat n = \frac{x \uveci + y \uvecj }{||x \uveci + y \uvecj||} =
 \frac{r\text{ cos}\theta \uveci + r\text{ sin}\theta \uvecj}{r} =
 \text{cos}\theta \uveci + \text{sin}\theta \uvecj. 
\end{equation*}
Our integral now has the form, 
\begin{align*}
\int_{CS} \frac{\vec V^2}{2} \rho \vec V \cdot \hat n dA & = R \rho \int
 \int \frac{(u \uveci)^2}{2} (u \uveci) \cdot
 (\text{cos}\theta \uveci + \text{sin}\theta \uvecj) d\theta dz \\
 & = \frac{1}{2} R \rho u^3 \int^{z_\text{max}}_0
 \int^{\frac{3\pi}{2}}_{\frac{\pi}{2}} \text{cos}\theta d\theta dz \\ 
 & = \frac{1}{2} R \rho u^3 z_\text{max}
 \int^{\frac{3\pi}{2}}_{\frac{\pi}{2}} \text{cos}\theta d\theta \\
 & = -R \rho u^3 z_\text{max}. 
\end{align*}

The negative sign here indicates that the kinetic energy is flowing into
the surface, in opposition to the outward facing unit normal, $\hat
n$. Characteristic values for this analysis are, $u = 5$ m/s, $\rho =
1.225$ Kg/$m^3$, R$= 3$ m, and $z_{\text{max}} = 2.5$ m. This provides
an estimate of 1148.44 Watts as the incoming kinetic energy flux across
the SoV vanes. Or, approximately 1.15 kW. 

Note that this is only an accounting of the incoming flow energy. It is certainly
impossible to extract all this energy from the flow, where Betz-like considerations 
would impose a maximum in the possible energy extraction. We only present an idealized
estimate of the total energy available in the flow, to provide a very rough estimate.


\section*{Gravitational Potential Energy Flux}


Now we estimate the gravitational potential energy flux by integrating 
the boussinesq term by the height of the vanes, 
\begin{align*}
  & = u \int_{-h}^0 \Delta \rho g z dz. \\
  & = u \int_{-h}^0 \rho' g z^2 dz. \\
  & = \frac{1}{3} u \rho' g z^3 \bigg|_{-h}^{0}
\end{align*}
Where the substitution, $\Delta \rho = \rho z$ was made. Furthermore, we
note that $\rho' = -\beta \rho_0 \Delta T$, resulting in, 
%
% cite monin-yaglom page 59
%
\begin{equation}
 \text{Power } = \frac{1}{3} u \beta \rho_0 \Delta T g h^3.
\end{equation}

Using $\rho_0 = 1.225$ Kg/$m^3$, $T_{\text{ref}}=313$, $\beta_T = 0.003194$
(This is just 1/$T_{\text{ref}}$), $g=9.81$ m/$s^2$ results in an
estimate of 50 Watts for the gravitational potential energy. 

%
% old discussion
%
% The thermal energy flux is essentially the specific enthalpy. Again 
% examining our 1st law energy balance rate equation, 
% \begin{equation*}
% \frac{dE}{dt} = \dot Q - \dot W + \dot m_{in}\left( h_{in} + \frac{V_{in}^2}{2} + gz_{out} \right)
% - \dot m_{out}\left( h_{out} + \frac{V_{out}^2}{2} + gz_{out} \right).
% \end{equation*}
% Now we make several convenient simplifying assumptions. 
% Namely, that the system is steady state ($\frac{dE}{dt} = 0$), 
% there is no heat transfer with the surroundings ($\dot Q = 0$), 
% and there are no appreciable contributions from gravitational 
% potential energy ($z_{out}-z_{in} = 0$). Furthermore, we 
% previously provided an estimate for the kinetic energy contribution. 
% Finally, we assume that the mass flux is approximately the same 
% between the inlet and outlet. 

% Our resulting equation is then, 
% \begin{equation*}
%   \dot W_{\text{thermal}} = \dot m \left(h_{in}-h_{out}\right).
% \end{equation*}
% At the very moderate temperatures considered, 
% for the working fluid (air), it is reasonable to model 
% the system as an ideal gas. Under these conditions, 
% the specific enthalpy will only depend on the temperature. 

% Our expression then becomes, 
% \begin{equation*}
%   \dot W_{\text{thermal}} = \dot m \left(h(T_{in})-h(T_{out})\right).
% \end{equation*}
% We now further approximate this system by making the 
% assumption that the specific heats are constant, or that, 
% \begin{equation*}
%   h(T_{in})-h(T_{out}) = c_p(T_{in}-T_{out})
% \end{equation*}
% Here, $T_{out}$ is the ambient temperature. For $T_{in}$ we 
% approximate the varying vertical gradient by a constant value equal to the
% mean value of the function over that interval, e.g.
% \begin{equation*}
% \bar f = \frac{1}{b-a} \int^b_a f(x) dx. 
% \end{equation*}
% With a gradient of $2/3$ Kelvin/meter this results in a delta T of
% approximately 2 degrees Kelvin. $c_p$ of air at this temperature is
% approximately 1006.5 kJ/(kg K). The resulting estimate of the thermal
% energy is substantial: 69 kW, or 60x larger than the kinetic energy of
% the wind. 

\end{document}
