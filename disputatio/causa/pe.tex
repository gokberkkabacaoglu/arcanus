\documentclass{article}
\usepackage{amsmath,amssymb}
\usepackage{bm}

\newcommand{\uveci}{{\bm{\hat{\textnormal{\bfseries\i}}}}}
\newcommand{\uvecj}{{\bm{\hat{\textnormal{\bfseries\j}}}}}
\newcommand{\uveck}{{\bm{\hat{\textnormal{\bfseries\k}}}}}

\title{\bf{Order of Magnitude Analysis of Energy Flux into the SoV}}
\author{Nicholas Malaya \\ Institute for Computational Engineering and Sciences \\ University of Texas at Austin} \date{}

\begin{document}
\maketitle

This document details an attempt to provide a very rough estimate of the
total energy flowing into the Solar Vortex apparatus. This information
is desired to provide a context to the energy fluxes measured through
the top of the device. 

At present, we consider only the energy flowing into the device due to
the ambient conditions, in particular, the incoming wind and heat
through the front hemisphere of the circular device. We consider a large
(3m radius) device incoming freestream velocity of 5 m/s. The surface
temperature is 343 Kelvin, with a specified inflow boundary layer
bridging the ground temperature to the ambient air conditions of 313
Kelvin. These numbers were chosen based on information provided by the
field team in Arizona during the summer of 2014. 

There are two forms of energy to consider: kinetic and enthalpy. We
begin by considering the kinetic energy flux through the front of the
apparatus. 

\section*{Kinetic Energy Flux}

From the first law of thermodynamics we can express the kinetic energy
flux as a surface integral over the upstream face of the device, 
\begin{equation*}
\int_{CS} \frac{\vec V^2}{2} \rho \vec V \cdot \hat n dA.
\end{equation*}
%
% could cite fluid dynamics book here
% pg. 239
%

Where, $\vec V = u \uveci + v \uvecj + w \uvecj.$ 
We assume our freestream velocity has no components in y and z, 
$\vec V = u \uveci + 0 \uvecj + 0 \uvecj.$ The surface of interest, S,
is a perfect cylinder, and therefore has a surface element of, 
\begin{equation*}
dS = Rd\theta dz. 
\end{equation*}
An outward pointing vector from this surface has the form, 
\begin{equation*}
x \uveci + y \uvecj. 
\end{equation*}
The normal vector is then,
\begin{equation*}
\hat n = \frac{x \uveci + y \uvecj }{||x \uveci + y \uvecj||} =
 \frac{r\text{ cos}\theta \uveci + r\text{ sin}\theta \uvecj}{r} =
 \text{cos}\theta \uveci + \text{sin}\theta \uvecj. 
\end{equation*}
Our integral now has the form, 
\begin{align*}
\int_{CS} \frac{\vec V^2}{2} \rho \vec V \cdot \hat n dA & = R \rho \int
 \int \frac{(u \uveci)^2}{2} (u \uveci) \cdot
 (\text{cos}\theta \uveci + \text{sin}\theta \uvecj) d\theta dz \\
 & = \frac{1}{2} R \rho u^3 \int^{z_\text{max}}_0 \int^{\frac{3\pi}{2}}_{\frac{\pi}{2}} \text{cos}\theta d\theta dz \\
 & = \frac{1}{2} R \rho u^3 z_\text{max}
 \int^{\frac{3\pi}{2}}_{\frac{\pi}{2}} \text{cos}\theta d\theta \\
 & = -R \rho u^3 z_\text{max}. 
\end{align*}

The negative sign here indicates that the kinetic energy is flowing into
the surface, in opposition to the outward facing unit normal, $\hat
n$. Characteristic values for this analysis are, $u = 5$ m/s, $\rho =
1.225$ Kg/$m^3$, R$= 3$ m, and $z_{\text{max}} = 2.5$ m. This provides
an estimate of 1148.44 Watts as the incoming kinetic energy flux across
the SoV vanes. Or, approximately 1.15 kW. 

\section*{Thermal Energy Flux}



\end{document}
