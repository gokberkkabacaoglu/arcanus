\documentclass{article}
\usepackage{amsmath,amssymb}
\usepackage{hyperref}
\title{\bf{Drag Polar Analysis}}
\author{Nicholas Malaya \\ Institute for Computational Engineering and Sciences \\ University of Texas at Austin} \date{}

\begin{document}
\maketitle

\newpage

The power extracted by the turbine is, 
\begin{equation}
 P = \Omega Q
\end{equation}
where the torque, Q, is, 
\begin{equation}
 Q = B \int_{r_{\text{min}}}^{r_{\text{max}}} F'_{\tau}\, r\, dr.
\end{equation}
Here, B is the number of blades $r_{\text{max}}$ and $r_{\text{min}}$
are the turbine radii, and $F'_{\tau}$ is the force per unit
length on the turbine, which is, 
\begin{equation}
 F'_{\tau} = \frac{1}{2}\rho U^2 \, c \, C_{\tau}.
\end{equation}
$U$ is magnitude of velocity. $c$ is the blade chord
length, which is assumed to be constant (not a function of the radius,
for instance). Finally, $C_{\tau}$ is the tangential force coefficient,
which depends on the local lift and drap coefficients, as well as the
flow angle, $\phi$, 
\begin{equation}
 C_{\tau} = C_L \,\text{sin}(\phi) + C_D \,\text{cos}(\phi)
\end{equation}
Combining the equations above results in an expression for the power
that explicitly depends on the lift and drag coefficients, 
\begin{equation*}
 P = \frac{\rho\, c\, \Omega B}{2}
  \int_{r_{\text{min}}}^{r_{\text{max}}} U(r)^2 \left(C_L
						     \,\text{sin}(\phi)
						     + C_D
						     \,\text{cos}(\phi)
						    \right) r\,dr. 
\end{equation*}
This equation is separable, 
\begin{align*}
 P_L = \frac{\rho\, c\, \Omega B}{2}
  \int_{r_{\text{min}}}^{r_{\text{max}}} U(r)^2 \, C_L(\phi,r) \,\text{sin}(\phi)\, r\,dr, \\
 P_D = \frac{\rho\, c\, \Omega B}{2}
  \int_{r_{\text{min}}}^{r_{\text{max}}} U(r)^2 \, C_D(\phi,r) \,\text{cos}(\phi)\, r\,dr. 
\end{align*}
Note that we have assumed $C_D = C_D(\phi,r)$ and $C_L = C_L(\phi,r)$,
namely, that the coefficients vary with the flow direction and may vary
radially, due to twisting the blade angle. Our objective is
now to discover what these unknown functions of lift and drag are. To do
this, we specify an optimization problem such that, 
\begin{equation*} 
 \text{Max } P(C_L,C_D) \quad \text{ subject to: }
  \begin{cases}
    C_L < ?, \\
    C_D > 0. \\
  \end{cases}
\end{equation*}



% \begin{align*}
%  \text{Max }& P(C_L,C_D) \\
%  \text{Subject to: }& C_L < ?, \\
%  & C_D < ?
% \end{align*}

\section{Questions}

\begin{itemize}
 \item Does this need regularization to ensure well-posedness?
 \item Boundary conditions are periodic
 \item What about supporting twist? (e.g. $\beta = \beta(r)$)
 \item Can we constrain $C_L, C_D$?
\end{itemize}

\end{document}
