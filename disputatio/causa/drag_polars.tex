\documentclass{article}
\usepackage{amsmath,amssymb}
\usepackage{hyperref}
\title{\bf{Drag Polar Analysis}}
\author{Nicholas Malaya \\ Institute for Computational Engineering and Sciences \\ University of Texas at Austin} \date{}

\begin{document}
\maketitle

\newpage

The power extracted by the turbine is, 
\begin{equation}
 P = \omega Q
\end{equation}
where the torque, Q, is, 
\begin{equation}
 Q = B \int_{r_{\text{min}}}^{r_{\text{max}}} F_{\tau}\, r\, dr.
\end{equation}
Here, B is the number of blades $r_{\text{max}}$ and $r_{\text{min}}$
are the turbine radii, and $F_{\tau}$ is the force on the turbine, which
is, 
\begin{equation}
 F_{\tau} = \frac{1}{2}\rho u_{\tau}^2 \, c \, C_{\tau}.
\end{equation}
$u_\tau$ is the relative velocity, which in this case would be
tangential to the turbine rotation direction. $c$ is the blade chord
length, which is assumed to be constant (not a function of the radius,
for instance). Finally, $C_{\tau}$ is the tangential force coefficient,
which depends on the local lift and drap coefficients, as well as the
flow angle, $\phi$, 
\begin{equation}
 C_{\tau} = C_L \,\text{sin}(\phi) + C_D \,\text{cos}(\phi)
\end{equation}
Combining the equations above results in an expression for the power
that explicitly depends on the lift and drag coefficients, 
\begin{equation*}
 P = \frac{\rho\, c\, \omega B}{2}
  \int_{r_{\text{min}}}^{r_{\text{max}}} u_{\tau}^2 \left(C_L
						     \,\text{sin}(\phi)
						     + C_D
						     \,\text{cos}(\phi)
						    \right) r\,dr. 
\end{equation*}
This equation is separable, 
\begin{align*}
 P_L = \frac{\rho\, c\, \omega B}{2}
  \int_{r_{\text{min}}}^{r_{\text{max}}} u_{\tau}^2 \, C_L \,\text{sin}(\phi)\, r\,dr, \\
 P_D = \frac{\rho\, c\, \omega B}{2}
  \int_{r_{\text{min}}}^{r_{\text{max}}} u_{\tau}^2 \, C_D \,\text{cos}(\phi)\, r\,dr. 
\end{align*}
Note that $C_D = C_D(\phi,r)$ and $C_L = C_L(\phi,r)$. Our objective is
now to discover what these unknown functions of lift and drag are. To do
this, we specify an optimization problem such that, 
\begin{equation*} 
 \text{Max } P(C_L,C_D) \quad \text{ subject to: }
  \begin{cases}
    C_L < ?\\
    C_D < ? \\
  \end{cases}
\end{equation*}

% \begin{align*}
%  \text{Max }& P(C_L,C_D) \\
%  \text{Subject to: }& C_L < ?, \\
%  & C_D < ?
% \end{align*}

\section{Questions}

\begin{itemize}
 \item Does this need regularization to ensure well-posedness?
 \item Boundary conditions are periodic
 \item What about supporting twist?
 \item Can we constrain $C_L, C_D$?
\end{itemize}

\end{document}
