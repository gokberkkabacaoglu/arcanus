\documentclass{article}
\usepackage{amsmath,amssymb}

\title{\bf{Drag Formulation}}
\author{Nicholas Malaya \\ Department of Mechanical Engineering \\ University of Texas at Austin} \date{}

\begin{document}
\maketitle

\newpage
%
% ||
%
Our unit forcing vector $\hat f$ is the normalized inner product of the 
velocity, $u$, and the tangent vector $\hat t$, 
\begin{equation}
- \frac{u \cdot \hat t}{|| u \cdot \hat t ||} = \hat f. 
\end{equation}

The force applied is then, 
\begin{equation}
 F = \hat f \, C_f \, \frac{1}{2} \frac{\rho || u ||^2}{\delta}
\end{equation}
Where $C_f$ is the skin friction coefficient (which must be determined),
$\rho $ is the density and $\delta$ the channel half width. 
For $C_f$, we use Dean's Correlation\cite{?}, 

%
%
%https://books.google.com/books?id=JBTlucgGdegC&pg=SA13-PA51&lpg=SA13-PA51&dq=dean%27s+correlation+fluids&source=bl&ots=auW8XopUC9&sig=DDxuQONFvqly5KQOocSPr39rS70&hl=en&sa=X&ved=0ahUKEwii3vX9wZHOAhXmxYMKHUqlChYQ6AEIJTAB#v=onepage&q=dean%27s%20correlation%20fluids&f=false  
%
\begin{equation}
 C_f = 0.073 (Re)^{-0.25}. 
\end{equation}
Where the Reynolds number is defined as, 
\begin{equation}
 Re = \frac{u\, \delta}{\nu}.
\end{equation}

Some roughness elements existed across the vanes. In addition to
non-smooth surface materials, the design of the field test apparatus
ultimately relied upon posts to hold the turning vanes. These posts were
pieces of wood which were modeled as roughness elements. In order to
determine $C_f$ for these cases, the Colebrook formula\cite{Colebrook367},
% Colebrook, C. F.; White, C. M. (3 August 1937). "Experiments with
% Fluid Friction in Roughened Pipes". Proceedings of the Royal Society
% of London. Series A, Mathematical and Physical Sciences. 161 (906):
% 367–381. doi:10.1098/rspa.1937.0150 
was used to provide an estimate for the friction factor given a
roughness height, $\epsilon/D$,  
\begin{equation}
 \frac{1}{\sqrt{f}} = -2.0 \text{ log}\left(\frac{\epsilon/D}{3.7} +
				       \frac{2.51}{Re\sqrt{f}}\right).
\end{equation}
 With the assumption that for large roughness the Reynolds number term
 contribution is not significant, the function is no longer implicit in
 $f$ and can be determined directly, 
\begin{equation}
 f = \left(2.0 \text{ log}\left(\frac{\epsilon/D}{3.7}\right)\right)^2. 
\end{equation}
%Then, a Moody chart was consulted to provide the the 
Note that D here is the hydraulic diameter which must account for the
two plates (we treat the vanes as a channel),
\begin{equation}
D_H = \frac{4 A}{P} = \frac{4 (2w)\delta}{2w} = 4 \delta
\end{equation}
Where $\delta$ is the channel half width.

\end{document}
