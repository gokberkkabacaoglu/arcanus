\documentclass{article}
\usepackage{amsmath,amssymb}
\usepackage{hyperref}
\title{\bf{Penalty Function and Turbine Modeling}}
\author{Nicholas Malaya, Roy Stogner, Robert D. Moser \\ Institute for Computational Engineering and Sciences \\ University of Texas at Austin} \date{}

\begin{document}
\maketitle

\newpage

\section{Penalty Function Formulation}

Using Laplace's equation as a test problem, 
\begin{equation}
 \nabla^2 u = 0 
\end{equation}
We impose a dirichlet-like boundary condition, 
\begin{equation}
 u|_{\Gamma_D} = g
\end{equation}
With a test function, $v$, the weak formulation is now, 
\begin{equation}
\int_{\Omega}  - \nabla u \cdot \nabla v - \frac{1}{\epsilon}
 \int_{\partial \Omega} (u-g) \cdot v = 0 
\end{equation}
Given, 
\begin{equation}
\int_{\Omega}  - \nabla u \cdot \nabla v - 
 \int_{\partial \Omega} v \partial_n u = 0 
\end{equation}
Then, 
\begin{align}
-\frac{1}{\epsilon}(u-g) = \partial_n u \\
 u = g - \epsilon \partial_n u 
\end{align}
In this manner, we can see that the velocity inthe penalty function
region is constrained to match the value $g$, much like a
slip-condition. So long as the value of $\epsilon$ is sufficiently
large, the velocities will closely match. 

%Here, $\beta$ is a large scale factor artificially introduced in order
%to ensure that the no-slip condition holds. 

\section{Turbine}

The penalty function will be added to the navier-stokes as, 
% \begin{align}
%  \partial_t(\rho u) = \text{NS} &+ \frac{1}{\epsilon}(u-u_t)\cdot n_p \\
%                                 &+ \frac{1}{\epsilon}(u-\omega u_t)\cdot n_p \\
% \end{align}

\begin{align}
 \partial_t(\rho u) = \text{NS} + \frac{1}{\epsilon}(u-\omega u_t)\cdot n_p
\end{align}

We will start by formulating a constant rotation speed
turbine. The rotor will have counter-clockwise spin with angular
velocity omega.   

\begin{verbatim}
  base_velocity=
                '{(r<r_max)*(z<zmax) * r * -sin(theta)*omega}
                 {(r<r_max)*(z<zmax) * r * cos(theta)*omega}
                 {0}'
\end{verbatim}
There are a few problems with this formulation. One, in a turbine, the
fluid velocity is not necessarily identical to the blade velocity. 
The turbine speed can take a range of values, and is often quoted as the
`tip-speed ratio'', $\lambda$, which is,
 \begin{equation}
  \lambda = \frac{\omega R}{v}.
 \end{equation}
Here, R is the radius of the turbine blades, omega the angular velocity,
 and v is the velocity of the fluid. Values for $\lambda$ generate power
 in the range of 0-16, and are typically between 4 and 14. 

Zmax will be the height of the vanes, which is approximately 0.84 meters in the laboratory, 
and 1.0795 meters for the two-meter SoV configuration. Rmax will be the inner diameter of the vanes. 

%
% ---------------------------------------------------
% old discussion on determining the rotation speed:
% ---------------------------------------------------
% We now need to choose $\omega$. At the Betz limit, the velocities will be
% $V_{\text{out}}/V_{\text{in}} = 1/3$. Furthermore, the control volume
% analysis implies that,
% \begin{equation}
%  V_{\text{turbine}} = \frac{1}{2}\left(V_{\text{in}} + V_{\text{out}} \right)
% \end{equation}
% then,
% \begin{align}
%  V_{\text{turbine}} &= \left(\frac{1}{2}\right) \left(\frac{4}{3}\right) V_{\text{in}} \\
%  &= \frac{2}{3} V_{\text{in}}.
% \end{align}
% Note this is not the velocity of the turbine, but the velocity of the
% fluid around the turbine. We can estimate the turbine speed from the
% ``tip-speed ratio'', $\lambda$, which is,
% \begin{equation}
%  \lambda = \frac{\omega R}{v}.
% \end{equation}
% Here, R is the radius of the turbine blades, omega the angular velocity,
% and v is the velocity of the fluid. Values for $\lambda$ generate power
% in the range of 0-16, and are typically between 4 and 14. Thus, our
% angular velocity of the turbine is, 
% \begin{equation}
%  \omega = \frac{ 2 V_{\text{in}}\lambda}{3 R_{\text{max}}}.
% \end{equation}

\section{References}
Look at penalty function papers: 
\begin{itemize}
 \item Babuska's paper on the B.C. \url{http://www.ams.org/journals/mcom/1973-27-122/S0025-5718-1973-0351118-5/S0025-5718-1973-0351118-5.pdf}
 \item penalty boundary condition on Navier-Stokes velocity turns out to be equivalent to a "slip" boundary condition: \url{http://onlinelibrary.wiley.com/doi/10.1002/fld.2264/full}
\end{itemize}

\end{document}
