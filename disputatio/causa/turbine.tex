\documentclass{article}
\usepackage{amsmath,amssymb}
\title{\bf{Turbine Modeling}}
\author{Nicholas Malaya, Roy Stogner, Robert D. Moser \\ Institute for Computational Engineering and Sciences \\ University of Texas at Austin} \date{}

\begin{document}
\maketitle

\newpage

A constant rotation speed turbine. Counter-clockwise spin with angular velocity omega. 

\begin{verbatim}
  base_velocity=
  '{(r<r_max)*(z<zmax)*r*-sin(theta)*omega}{(r<r_max)*(z<zmax)*r*cos(theta)*omega}{0}'
\end{verbatim}

Zmax will be the height of the vanes, which is approximately 0.84 meters in the laboratory, 
and 1.0795 meters for the two-meter SoV configuration. Rmax will be the inner diameter of the vanes. 

We now need to choose $\omega$. At the Betz limit, the velocities will be
$V_{\text{out}}/V_{\text{in}} = 1/3$. Furthermore, the control volume
analysis implies that,
\begin{equation}
 V_{\text{turbine}} = \frac{1}{2}\left(V_{\text{in}} + V_{\text{out}} \right)
\end{equation}
then,
\begin{align}
 V_{\text{turbine}} &= \left(\frac{1}{2}\right) \left(\frac{4}{3}\right) V_{\text{in}} \\
 &= \frac{2}{3} V_{\text{in}}.
\end{align}
Note this is not the velocity of the turbine, but the velocity of the
fluid around the turbine. We can estimate the turbine speed from the
``tip-speed ratio'', $\lambda$, which is,
\begin{equation}
 \lambda = \frac{\omega R}{v}.
\end{equation}
Here, R is the radius of the turbine blades, omega the angular velocity,
and v is the velocity of the fluid. Values for $\lambda$ generate power
in the range of 0-16, and are typically between 4 and 14. Thus, our
angular velocity of the turbine is, 
\begin{equation}
 \omega = \frac{ 2 V_{\text{in}}\lambda}{3 R_{\text{max}}}.
\end{equation}


\end{document}
