\section{Computational Methods and Software}
\label{sec:method}

\subsection{Equations (discretization and finite elements)}

We use the 3-D incompressible Navier-Stokes in order to model a
low-speed fluid flow with heat convection and diffusion and buoyancy. 

\begin{align*}
    \bv{R}\left(\left[
    \begin{array}{l}
        \bv{u} \\
        p \\
        T 
    \end{array}
    \right]\right) \equiv& 
    \left[
    \begin{array}{l}
        \frac{\partial (\rho \bv{u})}{\partial t} + \rho \bv{u} \cdot
    \nabla \bv{u} + \nabla p - \mu \nabla^2 \bv{u} + 
    \rho \beta_T (T - T_0) \bv{g} \\
    \nabla \cdot \bv{u} \\
    \frac{\partial (\rho c_p T)}{\partial t} + \rho c_p \bv{u} \cdot
    \nabla T - k \nabla^2 T
    \end{array} 
    \right] = 0
\end{align*}

The Navier-Stokes equations are derived from conservation of mass,
momentum and energy, and are a highly reliable model for low-speed flows
of the sort encountered here. The boussinesq approximation for the
buoyancy is less generally applicable, but still very accurate for a
wide variety of flows in nature, such as atmospheric fronts, oceanic
circulation, etc. The Boussinesq buoyancy approximation relies upon the
difference in density in the fluid being negligible except for gravitational
forces which are large enough to make the specific weight appreciably different
between the two fluids.


\subsection{Modeling the vanes}

Talk about penalty method here
% Be sure to pull out your old penalty method write up here.
% Will want to use an abridged version, and expand that in the full doc

\subsection{Software}

GRINS blurb (dont forget paper)

\subsection{Simulation geometry}

check out val doc here
dont forget to add wall spacing function to document

probably need to create a table of runs, perhaps that goes in results
