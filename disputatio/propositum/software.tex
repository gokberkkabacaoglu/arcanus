\section{Computational Methods and Software}
\label{sec:software}

\begin{itemize}
\item discretized equations, finite elements, stabilized
\item penalty formulation, also consider discussing sep model
\item software (GRINS+Libmesh)
\item verification of software
\item tool-chain, simulation machine and hardware
\item simulation geometry and boundary conditions for wind/thermal-only
\end{itemize}

\subsection{Equations (discretization and finite elements)}

weak form of equations
motivate finite elements

The resulting system of ODEs are discretized in time using a theta-method. 
Unsteady solver, backward Euler. 

mention stabilized and reference stabilization
cite braack here
cite hughes papers

Reference the paper on sane stabilization parameters. 

\subsection{Penalty Method Implementation}

Talk about penalty method here
cite babuska
cite immersed bounday methods?

\subsection{Software}

mention the FEM element order, any information about 
stuff like that

The libMesh\cite{libMeshPaper} finite element library
provides a wide set of tools with which to build a mesh-based
application. However, originally libMesh applications were required to
reimplement many kernels common to finite element applications,
including assembly loops, time integration schemes, etc. 

We therefore utilize the Finite Element GRINS library\cite{GRINSpaper}.
GRINS was designed to support multiphysics finite element
applications, the reusability and extensibility of mathematical
modeling kernels, supporting interfaces to existing solver and
discretization libraries to enable modern solution strategies, while, at
the same time, retaining flexibility to effectively tackle the science
or engineering problem of focus. 

GRINS provides a platform that enables powerful numerical algorithms
such as adjoint-based AMR, adaptive modeling, sensitivity analysis,
and, eventually, enabling uncertainty quantification.

GRINS stands for, ``General Reacting Incompressible Navier-Stokes'',
which roughly encapsulates the physical regimes it was originally
designed to simulate. 
GRINS is open source, and available on github\cite{github}. It is
released under LGPL2.1\cite{lgpl}. 

%The remainder of this subsection is devoted to
%discussing the underlying libraries used and the description of the
%GRINS framework.  
% PETSC\cite{petsc} trilinos\cite{trilinos}

% GRINS also utilizes the fparser\cite{fparser}
% library to support both parsing and compilation of mathematical
% functions into high 
% performance kernels. This capability allows for easy specification of
% boundary conditions, initial conditions, or constitutive equations from an input file. 

% Currently, libMesh has been scaled tens of thousands of cores and has
% been run on over 100,000 cores on the BG/Q machine Mira at Argonne National
% Lab\cite{libmesh-scaling}

%In principle, alternative software libraries/frameworks such as
%FEniCS\cite{fenics}, OpenFOAM\cite{openfoam}, etc. would likely be
%capable of simulating this regime. 

\subsection{Tool Chain and Simulation Custodianship}

Runs are queued on TACC\cite{tacc} supercomputer's 
Lonestar Four and Stampede. Run durations are typically 
twelve hours to perform several hundred timesteps. 
These runs are generally submitted to the production queue and are  
264-528 processing cores, 
or 22-44 nodes on lonestar (with 12 cores per node), and a similar number
for Stampede. The runs typically have several million degrees of freedom (DoF), 
and thus the local (per core) DoF maintained at $O(10^4)$. This was selected due to 
memory considerations and after a strong scaling analysis of the codebase on these 
resources, as well as consulting with the software primary developers. 

After a run terminates, several worker scripts are automatically invoked. 
These scripts automatically archive the run (outside of the volatile /scratch 
production directories) and simultaneously, label the concluded run with unique metadata
that defines the system environment, the jobs input files and run definitions, as well as 
information detailing the hypothesis or physics the job was intended to investigate.
Finally, once a week an rsync is performed on the entire archived database to ensure 
more than single redundancy for the runs.

In other words, the workflow is sufficiently advanced as to permit rapidly queuing 
a series of runs (in parallel) that are intended to investigate a variety of conditions or 
scenario parameters. This capabilty is necessary for the optimization campaign detailed 
in \ref{sec:future_work}, where running many concurent investigations will be required to
adaquetly sample the configuration space. 

\subsection{Simulation Geometry and Boundary Conditions}

check out val doc here
want a picture of the mesh too

explain how cell reynolds number is maintained for each grid

do you want boundary conditions? would have to mention both the lab and
the field and wind!

be sure to mention the 'sponge' layer. would also be nice to have references to papers on it!
