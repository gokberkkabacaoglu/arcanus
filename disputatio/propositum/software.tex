\section{Computational Methods and Software}
\label{sec:software}

\begin{itemize}
\item discretized equations, finite elements, stabilized
\item penalty formulation, also consider discussing sep model
\item software (GRINS+Libmesh)
\item verification of software
\item tool-chain, simulation machine and hardware
\item simulation geometry and boundary conditions for wind/thermal-only
\end{itemize}

\subsection{Equations (discretization and finite elements)}

weak form of equations
motivate finite elements

mention stabilized and reference stabilization
cite braack here
cite hughes papers

Refence the paper on sane stabilization parameters. 

\subsection{Modeling the vanes}

Talk about penalty method here
cite babuska

\subsection{Software}

mention the FEM element order, any information about stuff like that

The libMesh\cite{libmesh} finite element library
provides a wide set of tools with which to build a mesh-based
application. However, originally libMesh applications were required to
reimplement many kernels common to finite element applications,
including assembly loops, time integration schemes, etc. 

We therefore utilize the Finite Element GRINS library\cite{grins}.
GRINS was designed to support multiphysics finite element
applications, the reusability and extensibility of mathematical
modeling kernels, supporting interfaces to existing solver and
discretization libraries to enable modern solution strategies, while, at
the same time, retaining flexibility to effectively tackle the science
or engineering problem of focus. 

GRINS provides a platform that enables powerful numerical algorithms
such as adjoint-based AMR, adaptive modeling, sensitivity analysis,
and, eventually, enabling uncertainty quantification.

Thus, GRINS stands for, ``General Reacting Incompressible Navier-Stokes'',
which roughly encapsulates the physical regimes it was originally
designed to simulate. The remainder of this subsection is devoted to
discussing the underlying libraries used and the description of the
GRINS framework.  

PETSC\cite{petsc} trilinos\cite{trilinos}

The resulting system of ODEs are discretized using a theta-method. 
Unsteady solver, backward Euler. 

Released under LGPL2.1\cite{lgpl}. 

% GRINS also utilizes the fparser\cite{fparser}
% library to support both parsing and compilation of mathematical
% functions into high 
% performance kernels. This capability allows for easy specification of
% boundary conditions, initial conditions, or constitutive equations from an input file. 

% Currently, libMesh has been scaled tens of thousands of cores and has
% been run on over 100,000 cores on the BG/Q machine Mira at Argonne National
% Lab\cite{libmesh-scaling}

Open source, on github\cite{github}

In principle, alternative software libraries/frameworks such as
FEniCS\cite{fenics}, OpenFOAM\cite{openfoam}, etc. would likely be
capable of simulating this regime. 

%Tried to develop an OpenFOAM system.   

\subsection{Tool Chain and Simulation Custodianship}

Discuss how runs are performed, archived, etc. 

\subsection{Simulation Geometry and Boundary Conditions}

check out val doc here
want a picture of the mesh too

explain how cell reynolds number is maintained for each grid

do you want boundary conditions? would have to mention both the lab and
the field and wind!

be sure to mention the 'sponge' layer. would also be nice to have references to papers on it
