\section{Introduction / Executive Summary}

Renewable energy is critical to our environmental, economic, and
national security. Demand for energy is on the rise, as is our national
reliance on fossil fuel-based power plants for the bulk of our
electricity generation. There is a critical need for safe, clean, and
cost-effective alternatives to coal, such as wind, solar, hydroelectric,
and geothermal power\cite{arpa-e}. These technologies would reduce carbon dioxide
emissions and help position the U.S. as a leader in the global renewable
energy industry. 
%
% proposal
%
This proposal details a research plan perform a numerical investigation
for the design and optimization of a novel device for renewable, clean
energy generation. 

Much of the solar energy incident on the Earth's surface is absorbed
into the ground, which in turn heats the air layer above the surface.
This buoyant air layer contains considerable gravitational potential
energy. 
With nearly one-third of global land mass covered by deserts, there are huge
untapped regions for capturing solar heat (about 200 W/$\text{m}^2$ averaged over
a 24-hour day, and up to 1000 W/$\text{m}^2$ peak)\cite{something}.  The
available power is competitive in magnitude with worldwide power
generation from fossil sources. If successful, this could result in a
low-cost, scalable approach to electrical power generation that could
create a new class of renewable energy ideally suited for arid low-wind regions. 

How then, is one to efficiently extract this gravitational potential
energy and convert it into useable work? We turn to Nature to provide a 
guide, with the observation that objects already
exist that provide precisely this mechanism. Namely, the phenomena is
that of a naturally 
%
% are we certain this is baroclinic? montegomery might argue it is not.
% update: I think montgomery is wrong here, or at best nitty.  
% BAROCLINIC: essentially just that temp fronts exist
%
occurring ``dust devil'' with baroclinic generation of vorticity in a
vertically stratified, ground-heated air layer producing a coherent
columnar vortex. These ``dust-devil''s are naturally appearing in
regions as diverse as Arizona, to over water, to Mars\cite{mars}. They
are observed to occur over a wide range of length scales (between )with
large variations in 
velocities ()\cite{sinclair}. 

Thus, the basic idea behind this engineering approach is to convert the 
potential energy in this buoyant air layer to kinetic energy in an
anchored vortex, and to use that kinetic energy to drive a
vertical-axis turbine coupled with an electric generator in order to
produce electrical power. 
The Solar-Driven Vortex (SoV) phenomena has already been demonstrated in
an experimental setup by our partners at Georgia Tech. The simulation
effort utilizes Computational Fluid Dynamics (CFD) to simulate
this SoV. 

%This is a considerable effort. 

%
% TODO: add table of 'state of the art' or novel work performed
%


This proposal outlines a body of research to be performed to build
confidence in the simulation capability, optimize the apparatus, and 
probe the underlying physical dynamics of the system. 
The proposal is organized as follows. In section \ref{sec:physics}, we
begin with a discussion of the physical domain of interest. We then
review the presently understood dynamics of Dust-devils and similar
columnar vorticies. We then utilize this knowledge in section
\ref{sec:model}, in order to motivate a mathematical formulation of the
system of interest. In section \ref{sec:software}, we discuss the 
algorithms and software implementation in order to explicitly solve the
equations of interest and arrive at numerical solutions of our
phenomena. Section \ref{sec:validation} discusses the validation of
these results against existing experimental data and high fidelity
simulations.  Section \ref{sec:results} details the preliminary
predictions of system performance in the field, as well as the
present results of optimization of the apparatus. Finally, section
\ref{sec:future_work} contains a list of remaining tasks necessary to
fufill the goals of this research project. 


%details a short validation
%study performed by comparing between the available experimental
%measurements and the simulations results. 


%In order for these simulations to be generally useful, they must first
%be validated against existing experimental data and high fidelity
%simulations. These models will then explore regimes and scales where no
%experimental measurements presently exist. 
%Characterizing the
%uncertainty of predictions resulting from extrapolation is a critical
%component in enabling reliable assessments of field performance of the
%SoV, as it will guide the commercialization strategy of the product. 
