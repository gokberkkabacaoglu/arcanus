\section{Introduction / Executive Summary}

Renewable energy is critical to our environmental, economic, and
national security. Global demand for energy is projected to rise 56\% by
2040\cite{energy-outlook}, as is our national reliance on fossil
fuel-based power plants for the bulk of our electricity generation. 
There is a critical need for safe, clean, and
cost-effective alternatives to coal, such as wind, solar, hydroelectric,
and geothermal power. These technologies would reduce carbon dioxide
emissions and help position the U.S. as a leader in the global renewable
energy industry. 
% \cite{arpa-e}
% proposal
%
This proposal details a research plan perform a numerical investigation
for the design and optimization of a novel device for renewable, clean
energy generation. 

Much of the solar energy incident on the Earth's surface is absorbed
into the ground, which in turn heats the air layer above the surface.
This buoyant air layer contains considerable gravitational potential
energy. 
With nearly one-third of global land mass covered by deserts, there are huge
untapped regions for capturing solar heat (about 200 W/$\text{m}^2$ averaged over
a 24-hour day, and up to 1000 W/$\text{m}^2$ peak)\cite{Hoyt197827}.
The available power is competitive in magnitude with worldwide power
generation from fossil sources. If successful, this could result in a
low-cost, scalable approach to electrical power generation that could
create a new class of renewable energy ideally suited for arid low-wind regions. 

How then, is one to efficiently extract this gravitational potential
energy and convert it into usable work? We turn to Nature to provide a 
guide, with the observation that objects already
exist that provide precisely this mechanism. Namely, the phenomena is
that of a naturally 
%
% are we certain this is baroclinic? montegomery might argue it is not.
% update: I think montgomery is wrong here, or at best nitty.  
% BAROCLINIC: essentially just that temp fronts exist
%
occurring ``dust devil'' with baroclinic generation of vorticity in a
vertically stratified, ground-heated air layer producing a coherent
columnar vortex. These ``dust devil''s are ubiquitous, naturally appearing in
regions as diverse as Arizona, Siberia, over water, or even
Mars\cite{Sinclair1969,ROG:ROG1635,JGRE:JGRE1660}.  
% arizona, indiana, oregon, yukon, colorado
They are observed to occur over a wide range of length scales (1 - 30
meters) with large variations in velocities (1 to over 40
m/s)\cite{Sinclair1969}. 

Thus, the basic idea behind this engineering approach is to convert the 
potential energy in this buoyant air layer to kinetic energy in an
anchored vortex, and to use that kinetic energy to drive a
vertical-axis turbine coupled with an electric generator in order to
produce electrical power. 
The Solar-Driven Vortex (SoV) phenomena has already been demonstrated in
an experimental setup by our partners at Georgia Tech. However, in order to 
move beyond proof-of-concept, this effort utilizes Computational Fluid 
Dynamics (CFD) to simulate the SoV. This simulation capability provides 
fundamental insight into the 
driving dynamics of the system, generates high resolution data that is 
experimentally inaccessible (or at least, prohibitively difficult to gather) 
and can be utilized to rapidly optimize the geometry and configuration of 
the SoV apparatus. 

%This is a considerable effort. 

%
% TODO: add table of 'state of the art' or novel work performed
%

The objective of this project is to assess the technological feasibility of 
utilizing synthetic columnar vortices to generate usable energy. 
This proposal begins in section \ref{sec:physics} with a discussion of the 
naturally occurring phenomena. We 
review the presently understood dynamics of dust-devils and similar
columnar vortices, and how this informs the general concepts of system 
design for the synthetic counterparts we desire to generate. 
In section \ref{sec:mathmodel}, we outline a mathematical formulation of
the entire system. This leads naturally to section \ref{sec:software},
where we discuss the algorithms and software implementation used to
solve the equations of interest and arrive at numerical
solutions of our phenomena. Section \ref{sec:validation} discusses the
validation of these results against existing experimental data and high
fidelity simulations. Section \ref{sec:results} details the preliminary
predictions of system performance in the field, as well as detailing the 
several examples of a numerical optimization of the apparatus. Finally, with the 
preceding sections outlining the present simulation capabilities, 
section \ref{sec:proposed_work} proposes a course of investigation
designed to broadly probe the design space and in doing so, provide a
definitive assessment of the technological feasibility of the entire 
synthetic columnar vortex concept. 

%details a short validation
%study performed by comparing between the available experimental
%measurements and the simulations results. 

%In order for these simulations to be generally useful, they must first
%be validated against existing experimental data and high fidelity
%simulations. These models will then explore regimes and scales where no
%experimental measurements presently exist. 
%Characterizing the
%uncertainty of predictions resulting from extrapolation is a critical
%component in enabling reliable assessments of field performance of the
%SoV, as it will guide the commercialization strategy of the product. 
