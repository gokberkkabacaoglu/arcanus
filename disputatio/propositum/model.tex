\section{Mathematical Models}

\begin{itemize}
\item \st{unstable stratified boundary layers (raleigh number estimate)}
\item \st{justify incompressible N-S}
\item \st{justification of far-field eddy-viscosity model (M-O)}
\item modeling eddy-viscosity in device 
\item vane and turbine representation via penalty function // immersed boundary method
\item cone representation
\end{itemize}

%remember that \st{} is strikethrough
%
% should this all be math modeling?
%

Our aim is to simulate the formation of synthetic dust devils in the
field. This requires a model of the ambient conditions for a
representative case, such as Arizona, where experimental data is
available from test have been performed. Furthermore, for this to be
more generally useful in the prediction of flows in a variety of
conditions, we need a model generally applicable to any flow near the
surface of the earth.  

This section details an analysis of surface fluid mechanics, and
develops a mathematical model for turbulence in a thermally stratified
medium. 
% do we want this?
As we are utilizing a RANS model with spatially variable
diffusivity, we are particularly interested formulating a model for
these quantities. 

We are emulating the operation of the apparatus during the day. 
At these times, the atmospheric surface layer has the following character. 
Incident radiation from the sun largely does not interact with the
air, which is nearly transparent. Instead, this radiation is absorbed by
the ground, which causes a temperature rise. This results in a thermal
gradient between the hot ground and the cooler air. The warm ground
conducts heat to the air, causing an expansion and lowering the density
of the air. This reduced density air near the surface is driven upwards
by the force of buoyancy.  

For sufficiently large temperature gradients, these motions are
unstable, and as the warm air is driven upwards the flow will transition
to turbulence. For the typical use case we consider, namely Arizona in summer, 
Rayleigh numbers are typically between $10^{9} - 10^{11}$, and are therefore 
well in excess of the criterion for transition to a turbulent regime. The 
flow is that of an unstably statified fluid. 

\subsection{Equations of Fluid Motion}
%
% do I need to justify this more? These are pretty critical, after all
%

The equations describing fluid flow at
low Mach number with natural convection are, 
\begin{eqnarray*}
 \frac{\partial u}{\partial t} + u \cdot \nabla u =&
  -\frac{1}{\rho}\nabla P + \nu \nabla^2 u - g \frac{T'}{T_0}\\
 \nabla \cdot u =& 0 \\
 \rho c_p \frac{\partial T}{\partial t} =& \nabla \cdot ( k \nabla T).
\end{eqnarray*} 
Where we have made the assumptions that the temperature gradient is small in
comparison to the mean temperature of the region. These are the
Incompressible Navier-Stokes equations for free-convection (or
``Boussinesq'') coupled with the heat equation. 
%
% in full document be sure to mention that neglecting coriolis is legit
% below 50ms
%
%
As discussed above, we anticipate our flow to be
turbulent. Turbulence significantly alters the character of the flow,
and necessitates either resolving the resulting small scales or
providing a model that emulates their impact. In this case, we utilize 
essentally a Reynolds Averaged Navier-Stokes (RANS) model, where we
permit the viscosity and thermal conductivities to vary in space, and
decompose the flow into constant laminar and varying turbulent and vane
components,  
\begin{eqnarray*}
 \nu =& \nu_{l} + \nu_{T}(z) + \nu_{V}(r,z) \\
 K =& K_{l} + K_{T}(z) + K_{V}(r,z)
\end{eqnarray*}

This is an effective eddy viscosity model, and the subsequent two
sections will elaborate on the spatial dependence and character of
$\nu_T$,$K_T$,$\nu_C$ and $K_C$. 

\subsection{Viscosity Model}

We utilize the celebrated similarity model of Monin and Yaglom\cite{} as
a guide to the present development, which we outline below. This work is
extension of the mixing-length model of Prandtl, where the concepts of
gradient diffusion and mixing length were generalized to thermally
stratified flow.  
%
% justify prandlt assumption here
%

We begin by using dimensional analysis, and noting that the dynamics of
any mean quantity ($\bar f$) in a thermally stratified medium only depend on,

\begin{equation}
\bar f = f(z_0,\frac{g}{T_0},\rho_0,\nu,k,u^*,q)
\end{equation}

We expect that aside from very near the surface, the diffusivities $\nu$
and $k$ will be small compared to their turbulent counter-parts, $\nu_T$
and $K_T$. Likewise, if we define $z-z_0$ as an ``effective roughness
height'' or displacement distance, we can reasonably neglect $z_0$ from these
considerations. While the roughness height can be large (for instance in
a cornfield, where the roughness height could reasonably be several
meters), for our present study the expectation is that this roughness
height will be on the order of centimeters\cite{}. 

This leaves only five parameters: the distance from the ground, z; the
buoyancy coefficient, $\frac{g}{T_0}$; the density of the fluid,
$\rho_0$; a velocity scale, $u^*$; and the heat flux from the ground,
$q$. 
%
% add refence to dynamical and physical meteorology 
% 
These quantities depend on
four dimensions: length, time, temperature and mass. As a result,
Buckingham Pi theorem implies that only one dimension-less group can be
formed\cite{}.%munson 

This group is chosen to be,
\begin{equation}
 \xi = \frac{z}{L}.
\end{equation}
Here, $L$ is the famous, ``Monin-Obukhov'' length,
\begin{equation}
 L = -\frac{{u^*}^3}{\kappa \frac{g}{T_0} \frac{q}{c_p \rho_0}}
\end{equation}
where $\kappa$ is the (dimensionless) Von-Karman constant. Notice that
our interest lies in regimes where $L<0$, as $q<0$ (e.g. heat from the
ground into the fluid), which corresponds to the unstable stratification 
we expect during a sunny day. 

We are now in a position to state that the mean quantity has a
functional representation to the effect,
\begin{equation}
 \bar f = C \phi(\xi)
\end{equation}
with C a multiplicative constant with units of $\bar f$, and $\phi$ is a
function only of $\xi$. We are interested in the case where $\xi<0$, which
corresponds to heat flux from the ground into the air.  

The case where $\xi \to -\infty $ implies $\frac{z}{L} \to
-\infty $ and $z>>L$. This is most readily interpreted as the instance
where $u^* \to 0$, e.g. the case with no wind. For this case, the
function $\varphi_T$ must hold no dependence on $u^*$, and will approach
a constant value. Scaling analysis implies that the overall function
will not depend on $u^*$ only when the function $\varphi$ scales to the
$-\frac{4}{3}$ power. The resulting function appears as, 

\begin{equation}
 K_T = \frac{1}{C_T} \left( \frac{q}{c_p \rho_0} \frac{g}{T_0}
		     \right)^\frac{1}{3} z^{\frac{4}{3}}  \text{ 
for } z \gg L. 
\end{equation}

So long as the Prandtl number remains constant in space\cite{}, then
% todo: provide discussion as to why this is not an unreasonable expectation
identical arguments as to the asymptotic behaviour at large $\xi$ provide
the analogous result for the eddy viscosity's variation with respect to
distance from the ground,  
\begin{equation}
 \nu_T = \frac{1}{C_{\nu_T}} \left( \frac{q}{c_p \rho_0} \frac{g}{T_0}
			     \right)^\frac{1}{3} z^{\frac{4}{3}}  \text{ 
for } z \gg L. 
\end{equation}

These functions have been found to be broadly applicable, accurate and 
are easily instantiated in software. 

\subsection{Eddy Viscosity in Device}

The thermal and momentum diffusivities are also higher in the
device. The vanes generate shear and increase the turbulence. 

$K_{V}(r,z)$ and $\nu_{V}(r,z)$

Calibration problem. 

\subsection{Vane and Turbine Representation}

Cite babuska?

Provide motivation. -- rapidly prototype vanes
link to immersed boundary methods
Show how the forcing is rendered. 

\subsection{Cone Representation}

Surface is easy to define
