%\section{Summary and Remaining Work}
\section{Proposed Research Campaign}
\label{sec:proposed_work}

\begin{itemize}
\item \st{main event!}
\item ``proposed'' new runs and physical investigation
\item rough time line
\item discuss the program of runs designed to explore a wide configuration space
\item \st{calibrated turbine -- and what else?}
\item \st{make clear the objective is to assess feasibility, not to prove it can work}
\item make table of 'state of the art' or novel work expected to have been performed
\item what (fundamental) questions do you want to ask?
      \begin{itemize}
      \item Can we link phenomena to naturally occuring?
      \item By what geometries is the flow intensified, and why
      \end{itemize}
\end{itemize}

The objective of this research project is to provide a definitive
assessment of the  
technological feasibility of the entire synthetic columnar vortex
concept as a means of generating usable energy. In the 
previous sections we have briefed the reader on the present state of the
simulation capability. In doing so, we have discussed the physics that
influence dust devils formation, as well as our particular mathematical
models for the ambient conditions as well as the SoV vanes, cone and
turbine. We summarized the numerical discretizations used, the software
stack and the calibration, verification and validation of these
components. The purpose of these sections was to convince the reader of
two major points. The first point is that an accurate, verified and
validated simulation capability has been developed that can quickly
investigate a wide variety of system and scenario settings at a modest
computational cost. The second point is that we have developed
heuristics that permit optimization of any baseline SoV configuration to
a local maximum of energy production, as measured by kinetic energy flux
through the top of the SoV vanes.  

These two points justify the proposed course of work, which is to
broadly sweep through a large space of possible system configurations
and geometries, and in doing so, discover the globally optimal structure
of the SoV apparatus. Coupled with the scaling analysis presented in
section \ref{sec:physics}, we will then be able to predict the
conditions (if any) under which the SoV apparatus will be
technologically competitive with different methods of energy
generation. In other words, the proposed research is designed to assess
feasibility, and it is not expected that actual experimental validation
will accompany the computational results. Furthermore, it must also be
emphasized that feasibility here is focused on technological capability,
and will not include an economic assessment. In other words, it is
possible that the SoV will produce energy, but the design required to do
is prohibitively expensive, and therefore not economically competitive
with existing technologies. 

%
% outline parameter space, as we see it
%
We now outline a program of runs designed to explore a wide
configuration space. We begin by noting that we have three 
optimization domains, for the cone, turbine and vanes. Table
\ref{tab:vane} lists the proposed optimization parameters for the
vanes. We propose nine parameters to optimize for the
vanes. $\theta^{\text{t}}_{\text{min}}$ and
$\theta^{\text{b}}_{\text{min}}$ are the top and bottom tier minimum
angles. This is essentially the starting angle for the curved
vanes. Preliminary investigations have given no indication that the
bottom tier is receptive to anything but a fully radial angle
(e.g. $0^{\circ}$). For the top tier, however, a non-zero
$\theta_{\text{min}}$ has been found to increase the kinetic energy
flux. For both of the maximum angles we have reasonable starting
points. Previously, we have found that a larger angle at the bottom tier
is ideal, as it serves to ``spin-up'' a small core region.

We have less strongly informed prior expectations of reasonable values
for the rate of curvature, $\gamma$, the ratio of heights between the
top and bottom tiers ($H^b/H^t$) as well as the lengths of the
vanes. The generated flux is certainly sensitive to $\gamma$, but
optimal value is certainly not known. Likewise, while we have operated
with shorter vanes lengths on the top than the bottom, as well as a
shorter height, we are not certain that these configurations are
remotely optimal. Finally, for $D/H$, the ratio between the apparatus
diameter and total vane height, we have only operated near a ratio of
1.0. However, as noted in \cite{ROG:ROG1635}, ``Most dust devils are at 
least 5 times higher than they are wide''.  


%
% vane optimization
%
\large
\begin{center}
\begin{table}[h]
 \centering
  \begin{tabular}{| l | c | l |}
    \hline
    Parameter & Description & Range \\
    \hline
    $\theta^{\text{t}}_{\text{min}}$ & Starting, minimum angle of the
       top tier & ( 0 - $\theta^{\text{t}}_{\text{max}}$ ) \\
    $\theta^{\text{t}}_{\text{max}}$ & Ending, maximum angle of the top
       tier & ( 0 - 90 ) \\
    $\theta^{\text{b}}_{\text{min}}$ & Starting, minimum angle of the
       bottom tier & ( 0 - $\theta^{\text{b}}_{\text{max}}$ ) \\
    $\theta^{\text{b}}_{\text{max}}$ & Ending, maximum angle of the
       bottom tier & ( 0 - 90 ) \\
   $\gamma$ & Rate of curving & ( 0 - 3 ) \\
   $1 - (r_{\text{min}} / r_{\text{max}})^{\text{t}}$ & Length of the top
       tier vane & ( 0 - 1 ) \\
   $1 - (r_{\text{min}} / r_{\text{max}})^{\text{b}}$ & Length of the bottom
       tier vane & ( 0 - 1 ) \\
   $H^b/H^t$ & Ratio of heights between bottom and top tiers & ( 0 -
	   0.5 ) \\ 
   $D/H$ & Ratio of apparatus diameter and total vane height & ( 0.5 -
	   5.0 ) \\ 
    \hline
  \end{tabular}
  \caption{Vane Optimization Parameters.}
  \label{tab:vane}
\end{table}
\end{center}
\normalsize

Turbine

%
% turbine optimization
%
\large
\begin{center}
\begin{table}[h]
 \centering
  \begin{tabular}{| l | c | l |}
    \hline
    Parameter & Description & Range \\
    \hline
    $N_B$ & Number of blades & ( 1 - 12 ) \\
    $I$ & Moment of inertia & ( 1 - 12 ) \\
    $r_B/r_{\text{min}}^t$ & Radius of blade versus the inner radius of
       the top tier vanes & ( 0 - 1 ) \\
   $H_B/H$ & Height of the turbine blades versus system height & ( 0 - 1.2 ) \\
    \hline
  \end{tabular}
  \caption{Turbine Optimization Parameters.}
  \label{tab:turbine}
\end{table}
\end{center}
\normalsize

%
% cone optimization
%
\large
\begin{center}
\begin{table}[h]
 \centering
  \begin{tabular}{| l | c | l |}
    \hline
    Parameter & Description & Range \\
    \hline
    $H_C/D_C$ & Ratio of the height of the cone versus the cone diameter & ( 0 - 2.0 ) \\
    $D_{\text{C}}/D$ & Ratio of the cone diameter versus the system
       diameter & ( 0.5 - 1.5 ) \\
    $D_{\text{out}}/D_C$ & Ratio of the cone exit diameter versus the
       cone diameter & ( 0.25 - 1.0 ) \\ 
    \hline
  \end{tabular}
  \caption{Cone Optimization Parameters.}
  \label{tab:cone}
\end{table}
\end{center}
\normalsize

shed light on 
the mechanisms by which the apparatus configuration dictates the flow 


%
% how many runs are we capable of performing, realistically
%

%
% introduce concept of subdomains
%


%
% can we optimize? DAKOTA
%




In summary, in addition to the preliminary results, several tasks need
to be finished to fulfill the goals of this project:

\begin{itemize}
\item blah
\item yar
\end{itemize}

%
% discussion of risk mitigation 
%
\subsection{Risks and Mitigation}

code ready
expertise ready
plan outlined
ls5 delay?
more compute time
need more runs? this is a serious concern
do we need a comprehensive metric for success?
