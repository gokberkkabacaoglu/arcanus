%\section{Summary and Remaining Work}
\section{Proposed Research Campaign}
\label{sec:proposed_work}

% \begin{itemize}
% \item \st{main event!}
% \item ``proposed'' new runs and physical investigation
% \item rough time line
% \item discuss the program of runs designed to explore a wide configuration space
% \item \st{calibrated turbine -- and what else?}
% \item \st{make clear the objective is to assess feasibility, not to prove it can work}
% \item make table of 'state of the art' or novel work expected to have been performed
% \item \st{what (fundamental) questions do you want to ask?}
%   \begin{itemize}
%   \item \st{Can we link phenomena to naturally occuring?}
%   \item \st{By what geometries is the flow intensified, and why}
%   \end{itemize}
%   \item	consider discussing this is typical set for engineering (bob recommended)
% \end{itemize}

The objective of this research project is to provide a definitive
assessment of the  
technological feasibility of the entire synthetic columnar vortex
concept as a means of generating usable energy. In the 
previous sections we have briefed the reader on the present state of the
simulation capability. In doing so, we have discussed the physics that
influence dust devils formation, as well as our particular mathematical
models for the ambient conditions as well as the SoV vanes, cone and
turbine. We summarized the numerical discretizations used, the software
stack and the calibration, verification and validation of these
components. The purpose of these sections was to convince the reader of
two major points. The first point is that an accurate, verified and
validated simulation capability has been developed that can quickly
investigate a wide variety of system and scenario settings at a modest
computational cost. The second point is that we have developed
heuristics that permit optimization of any baseline SoV configuration to
a local maximum of energy production, as measured by kinetic energy flux
through the top of the SoV vanes.  

These two points justify the proposed course of work, which is to
broadly sweep through a large space of possible system configurations
and geometries, and in doing so, discover the globally optimal structure
of the SoV apparatus. Coupled with the scaling analysis presented in
section \ref{sec:physics}, we will then be able to predict the
conditions (if any) under which the SoV apparatus will be
technologically competitive with different methods of energy
generation. We will also be a position to investigate the physics of the
apparatus, and to assess how closely our synthetic dust devils mimic the
natural variety. In other words, the proposed research is designed to
assess feasibility, and it is not expected that actual experimental
validation will accompany the computational results. Furthermore, it
must also be emphasized that feasibility here is focused on
technological capability, and will not include an economic
assessment. In other words, it is possible that the SoV will produce
energy, but the design required to do is prohibitively expensive, and
therefore not economically competitive with existing technologies. 

%
% outline parameter space, as we see it
%
We now outline a program of runs designed to explore a wide
configuration space. We begin by noting that we have three 
optimization domains, for the cone, turbine and vanes. Table
\ref{tab:vane} lists the proposed optimization parameters for the
vanes. We propose nine parameters to optimize for the
vanes. $\theta^{\text{t}}_{\text{min}}$ and
$\theta^{\text{b}}_{\text{min}}$ are the top and bottom tier minimum
angles. This is essentially the starting (and minimum) angle for the curved
vanes. Preliminary investigations have given no indication that the
bottom tier is receptive to anything but a fully radial angle
(e.g. $0^{\circ}$). For the top tier, however, a non-zero
$\theta_{\text{min}}$ has been found to increase the kinetic energy
flux. For both of the maximum angles we have reasonable starting
points. Previously, we have found that a larger angle at the bottom tier
is ideal, as it serves to ``spin-up'' a small core region.

We have less strongly informed prior expectations of reasonable values
for the rate of curvature, $\gamma$, the ratio of heights between the
top and bottom tiers ($H^b/H^t$) as well as the lengths of the
vanes. The generated flux is certainly sensitive to $\gamma$, but
optimal value is certainly not known. Likewise, while we have operated
with shorter vanes lengths on the top than the bottom, as well as a
shorter height, we are not certain that these configurations are
remotely optimal. Finally, for $D/H$, the ratio between the apparatus
diameter and total vane height, we have only operated near a ratio of
1.0. However, as noted in \cite{ROG:ROG1635}, ``Most dust devils are at 
least 5 times higher than they are wide''. While increasing the height
of the apparatus would greatly increase the cost and inaccessibility of
the device for maintainance, this indicates that it is necessary to
examine the dependence of energy produce by the vortex at greater system
heights. We expect to increase beyond this aspect ratio of $\approx 1$
towards the naturally occuring ratio of 5. If this is found to greatly
enhance energy output, we will then consider even larger ratios. 

%
% vane optimization
%
\large
\begin{center}
\begin{table}[h]
 \centering
  \begin{tabular}{| l | c | l |}
    \hline
    Parameter & Description & Range \\
    \hline
    $\theta^{\text{t}}_{\text{min}}$ & Starting, minimum angle of the
       top tier & ( 0 - $\theta^{\text{t}}_{\text{max}}$ ) \\
    $\theta^{\text{t}}_{\text{max}}$ & Ending, maximum angle of the top
       tier & ( 0 - 90 ) \\
    $\theta^{\text{b}}_{\text{min}}$ & Starting, minimum angle of the
       bottom tier & ( 0 - $\theta^{\text{b}}_{\text{max}}$ ) \\
    $\theta^{\text{b}}_{\text{max}}$ & Ending, maximum angle of the
       bottom tier & ( 0 - 90 ) \\
   $\gamma^t$ & Rate of curving, vane top tier & ( 0 - 3 ) \\
   $\gamma^b$ & Rate of curving, vane bottom tier & ( 0 - 3 ) \\
   $1 - (r_{\text{min}} / r_{\text{max}})^{\text{t}}$ & Length of the top
       tier vane & ( 0 - 1 ) \\
   $1 - (r_{\text{min}} / r_{\text{max}})^{\text{b}}$ & Length of the
       bottom 
       tier vane & ( 0 - 1 ) \\
   $H^b/H^t$ & Ratio of heights between bottom and top tiers & ( 0 -
	   0.5 ) \\ 
   $D/H$ & Ratio of apparatus diameter and total vane height & ( 0.5 -
	   5.0 ) \\ 
    \hline
  \end{tabular}
  \caption{Vane Optimization Parameters.}
  \label{tab:vane}
\end{table}
\end{center}
\normalsize

The three cone parameters we propose to optimize are shown in table \ref{tab:cone}. 
We expect to optimize the cone after the bottom and top tiers are adjusted. 
For the cone, we expect the height, maximum diameter and inner exit diameter to 
all impact the flow. While it is not known what the ideal cone geometry will look like, 
our expectation is that the cone plays at least two important role. The first is acting
as a converging nozzle for the flow, increasing the vertical velocity as it exits 
out the top of the device. In the wind, the cone also acts as a shield, preventing
the high velocity freesteam flow from disrupting the vortex before it has run through the 
turbine. 

%
% cone optimization
%
\large
\begin{center}
\begin{table}[h]
 \centering
  \begin{tabular}{| l | c | l |}
    \hline
    Parameter & Description & Range \\
    \hline
    $H_C/D_C$ & Ratio of the height of the cone versus the cone diameter & ( 0 - 2.0 ) \\
    $D_{\text{C}}/D$ & Ratio of the cone diameter versus the system
       diameter & ( 0.5 - 1.5 ) \\
    $D_{\text{out}}/D_C$ & Ratio of the cone exit diameter versus the
       cone diameter & ( 0.25 - 1.0 ) \\ 
    \hline
  \end{tabular}
  \caption{Cone Optimization Parameters.}
  \label{tab:cone}
\end{table}
\end{center}
\normalsize

The turbine parameters to be optimized are shown in table \ref{tab:turbine}. 
The expectation is that the turbine optimization will be performed after the other
optimization efforts (top and bottom vanes, as well as cone). This is because the parameters 
of the turbine almost certainly are impacted by the geometric form of the dust devil. 
For instance, a wider dust devil may necessitate longer turbine blades. 

%
% turbine optimization
%
\large
\begin{center}
\begin{table}[h]
 \centering
  \begin{tabular}{| l | c | l |}
    \hline
    Parameter & Description & Range \\
    \hline
    $N_B$ & Number of blades & ( 1 - 12 ) \\
    $I$ & Moment of inertia & ( 1 - 12 ) \\
    $r_B/r_{\text{min}}^t$ & Radius of blade versus the inner radius of
       the top tier vanes & ( 0 - 1 ) \\
   $H_B/H$ & Height of the turbine blades versus system height & ( 0 - 1.2 ) \\
    \hline
  \end{tabular}
  \caption{Turbine Optimization Parameters.}
  \label{tab:turbine}
\end{table}
\end{center}
\normalsize


\subsection{Risks and Challenges to the Optimization Efforts}

A challenge inherent to this optimization effort is that while the
principle quantity of interest is the kinetic energy flux, we also seek
to use these runs to shed light on the mechanisms by which the apparatus
configuration dictates the flow. In other words, we are trying learn
more than \textbf{which} configurations optimize the flow, but also 
\textbf{why} they do so. 

% This presents an programmatic challenge, as
% optimization will involve additional analysis and postprocessing at each
% iteration.  

%
% how many runs are we capable of performing, realistically
%
An additional challenge is the computational expense of each run. Each
parameter exploration requires approximately 12 wall-clock hours to run
the simulation for a sufficient time to pass any initial transient in
the solution and then permit adequate statistical averaging at steady
state. Runs generally require approximately 264 processors on Lonestar
(for example) and therefore the cost of a forward run is $\approx 3,200$
core hours. Our overall compute budget will likely be between one hundred
thousand to one million core hours. In other words, our present
computational budge will support running between roughly 30 to 300
instances for our parameter sweeps. While this is not insubstantial, this
is not a sufficient number of evaluations to support formal
optimization algorithms. While we admittedly do not, \textit{a priori},
know the number of iterations necessary to solve the system, as it is
non-linear, our expectation is that thousands of forward solves would be
necessary. Utilizing higher order methods could reduce the number of
iterations, but gradient and derivative information is difficult to
access, as solving the adjoint problem is expensive for unsteady systems. 
Thus, while any individual run is relative inexpensive, we do not expect
to be able to mount an exhaustive campaign to formally optimize the
configuration given the dimension of the parameter space and the
structure of the problem. 

%
% introduce concept of subdomains
%
Our expectation is that we will proceed with optimization in a manner
similar to the examples outlined in section \ref{sec:results}, where new
parameter values are introduced, a run is performed, and then the output
is postprocessed and evaluated before the process begins again. Each
iteration is therefore more expensive, as it requires human
intervention between simulation runs. Nevertheless, it is not infeasible
to expect that several hundred runs can be performed. In general the
runs require twelve hours, which provides for a workflow loop between
runs and the evaluation of the output by an expert that roughly fits
into a daily routine.

Furthermore, the time requires to perform this simulation campaign could
be greatly reduced if several problems were to be undertaken in
parallel. This ``divide and conquer'' approach requires subdividing the
optimization effort into several ``subdomains''. 
The problem is nonlinear, and so some of the parameters are coupled and
cannot be easily optimized independent of each another, as adjustments
to one impact the desired value of the other. 
We propose to subdivide the vanes into upper and lower tiers, optimize
them individually, and then perform rudimentary sensitivity checks to
ensure that the coupled product does in fact represent a near optimal
flux output. We further expect that the remaining parameters in the
vanes, namely $H^b/H^t$, $D/H$ are amenable to optimization independent
of the particular parametric configuration of the vane tiers. 

The subdomains for the vanes are are summarized in Table
\ref{tab:opt}. This neatly divides the vane optimization effort into
three independent problems with 4 or fewer parameters each.

\begin{center}
\begin{table}[h]
 \centering
  \begin{tabular}{|l | l | l |}
   \multicolumn{3}{c}{Subdomains} \\
    \hline
   Top Tier & Bottom Tier & Misc. \\
   \hline
   $\theta^{\text{t}}_{\text{min}}$ & $\theta^{\text{b}}_{\text{min}}$ &
       $H^b/H^t$ \\
   $\theta^{\text{t}}_{\text{max}}$ & $\theta^{\text{b}}_{\text{max}}$&
	   $D/H$ \\ 
   $\gamma^t$ & $\gamma^b$ & \\
   $1 - (r_{\text{min}} / r_{\text{max}})^{\text{t}}$ & $1 -
       (r_{\text{min}} / r_{\text{max}})^{\text{b}}$ & \\
   \hline
  \end{tabular}
  \caption{Vane optimization parameters, divided into parallizable
 subdomains.} 
  \label{tab:opt}
\end{table}
\end{center}

\subsection{Proposed Timeline}



% %
% % discussion of risk mitigation 
% %

% code ready
% expertise ready
% plan outlined
% ls5 delay?
% more compute time
% need more runs? this is a serious concern
% do we need a comprehensive metric for success?


%
% timeline
%
A rough timeline of the proposed work is presented in table
\ref{tab:prop}. 



% 
% can we optimize? DAKOTA
% 
We propose to investigate formal optimization of these subproblems. 
We will consider sensitivity calcuations (forward, not adjoint) in order
to (possibly) reduce the problem dimension and inform our expectations
of system performance under adjustment. This feature is in development
in GRINS and should be available within the year. We are also interested
in utilizing either the DAKOTA\cite{adams2013dakota} or
TAO\cite{tao-user-ref} libraries on the bottom tier subproblem. These
libraries have suites of algorithms for non-linear optimization
problems. Evaluating the results of the most appropriate ``out of the
box'' available optimization algorithm on a small test problem will
provide an opportunity to ensure that our expectations on the number of
iterations necessary for formal optimization of our problem are correct. 
We note these two libraries because DAKOTA is well known as a library
with extensive ``black-box'' optimization support, and TAO is already
made available through pre-existing libMesh dependencies. 
%
% short summary of tasks
%
Finally, it is possible that this project will permit some glimpses 
at the fundamental processes underlying the naturally dust devil
phenomena. To accomplish this, the synthetic dust devils must be
compared to data from the naturally occuring variety,  with some
appropriate scaling. While it is not a significant component of the
proposal, this and several simple investigations into related phenomena
such as tornados and hurricanes will also  be investigated. A literature
review of the physics of general cyclonic structures and any observed
intensification mechanisms may provide hints of the geometries 
by which the flow is intensified, and why. 

%
% itemize tasks
%
% In summary, in addition to the preliminary results, several tasks will be
% accomplished in order to fulfill the goals of this project:

% \begin{itemize}
%  \item validate turbine
%  \item numerous optimization runs
%  \item best guess of ideal run (possible field run)
%  \item feasibility study?
% \end{itemize}

\begin{center}
\begin{table}
\caption{Timeline of Proposed Work. Bullets are dates of planned
 completion of deliverables. Black items are requisite, blue optional.}
\centering
\begin{minipage}[t]{.7\linewidth}
\color{black}
\rule{\linewidth}{1pt}
\ytb{Dec. 2015}{Algorithmic optimization subdomain problem proof-of-concept}
\ytl{Jan. 2016}{Conclude Parameter sweeps through all vane sub-domains}
\ytl{Feb. 2016}{Cone sub-domain optimization complete}
\ytl{Mar 2016}{Turbine sub-domain optimization complete}
\ytl{April 2016}{Coupling tests for parameters}
\ytb{May 2016}{Conclude Additional Parameter sweeps}
\ytl{June 2016}{Field configuration runs and predictions for
 experimental team}
\ytl{July 2016}{Comparisons between synthetic and natural dust devil physics}
\ytl{Aug  2016}{Dissertation detailing SoV feasibility assessment.}
\bigskip
\rule{\linewidth}{1pt}%
\end{minipage}%
\end{table}
\label{tab:prop}
\end{center}

% \subsection{Additional Investigations}

% % 
% % control for inter unit spacing
% % 
% In addition to the system configuration, it would be interest to consider the
% effect of local conditions on SoV performance. Characterizing the impact
% of variations in ambient conditions on the SoV will guide the
% commercialization strategy of the product, by determining optimal
% install locations across the country. It is therefore desirable to have
% models that are capable of accounting for variation in field conditions,
% such as solar input, cross-winds and topography. Furthermore, it is
% expected that large ``farms'' of SoVs (akin to the wind and solar farms
% for wind turbines and photovoltaics, respectively) may be used by
% commercial or utility-scale energy generation. In order for this to be
% effective,  the inter-unit spacing must also be optimized, as a single
% SoV collects from a large area. These computations will guide
% commercialization planning, where decision-makers will need to assess
% optimum unit size, spacing, and geographic location for utility-scale
% deployment.   
