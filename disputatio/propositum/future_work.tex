\section{Summary and Remaining Work}
\label{sec:future_work}

\begin{itemize}
\item \st{main event!}
\item ``proposed'' new runs and physical investigation
\item other ideas
\item rough time line
\item discuss the program of runs designed to explore a wide configuration space
\item calibrated turbine -- and what else?
\item make clear the objective is to assess feasibility, not to prove it can work
\item make table of 'state of the art' or novel work expected to have been performed
\item what (fundamental) questions do you want to ask?
      \begin{itemize}
      \item Can we link phenomena to naturally occuring?
      \item By what geometries is the flow intensified, and why
      \end{itemize}
\end{itemize}

The objective of this research project is to provide a definitive assessment of the 
technological feasibility of the entire synthetic columnar vortex concept as a 
means of generating usable energy. In the 
previous sections we have briefed the reader on the present state of the simulation capability. 
In doing so, we have discussed the physics that influence 
Dust-devils formation, as well as our particular mathematical models for the ambient conditions 
as well as the SoV vanes, cone and turbine. We summarized the numerical discretizations used, the software stack 
and the calibration, verification and validation of these components. 
The purpose of these sections was to convince the reader of two major points. 
The first point is that an accurate, verified and validated simulation capability has been developed
that can quickly investigate a wide variety of system and scenario settings 
at a modest computational cost. 
The second point is that we have developed heuristics that permit optimization of any baseline SoV
configuration to a local maximum of energy production, as measured by kinetic energy flux 
through the top of the SoV vanes. 

These two points justify the proposed course of work, which is to
broadly sweep through a large space of possible system configurations and geometries, and in doing so, 
discover the globally optimal structure of the SoV apparatus. Coupled with the scaling analysis presented 
in section \ref{sec:physics}, we will then be able to predict the conditions (if any) under which the SoV 
apparatus will be technologically competitive with different methods of energy generation. In other words, 
the proposed research is designed to assess feasibility, and it is not expected that actual 
experimental validation will accompany the computational results. Furthermore, it must also be emphasized 
that feasibility here is focused on technological capability, and will not include an economic assessment. 
In other words, it is possible that the SoV will produce energy, but the design required to do is 
prohibitively expensive, and therefore not economically competitive with existing technologies.

We now outline a program of runs designed to explore a wide configuration space.


In summary, in addition to the preliminary results, several tasks need to
be finished to fulfill the goals of this project:

\begin{itemize}
\item blah
\item yar
\end{itemize}
