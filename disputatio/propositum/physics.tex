\section{Physics of Dust-Devils}
\label{sec:physics}
\begin{itemize}
\item phenomenological description
\item regions/frequency of occurance
\item review of sinclair, kanak and renno and other pertinent literature
\item known physics of dust devils and other cyclonic structures
\item dust-devil/tornado/hurricane genesis
\item scaling discussion as estimate of energy
\item Dust-Devil Generation Concept
\item Motivation of computational modeling
\item draw regions
\item eyewall
\end{itemize}

In order to motivate how best to \textit{engineer} a synthetic
dust-devil, we first address what is known about the naturally occuring
phenomena. We therefore begin this section with a qualitative discussion
of dust-devils, followed by a review of the known physics and pertinent 
literature review. 


\subsection{Phenomenological Character of Dust-Devils}
\begin{itemize}
 \item chart from sinclair of frequency (they are driven by delta t)
 \item turning direction independent
 \item characteristic velocities and sizes
 \item genesis diagram (thermal plume) -- mention ambient angular vorticity
 \item structure diagram and description
\end{itemize}


%
% good place to ref jacobson2005fundamentals
%

\subsection{Estimate of Energy Scaling}

In this subsection, we attempt to provide a rough estimate of the energy
available to a Dust-Devil. There are two objectives behind this
analysis. The first is to provide justification for the concept of
extracting energy from these phenomena, with the reasoning that should
sufficient energy be available, then attemping to extract it might be
worthwhile. The second objective is to provide a simple analysis that
can serve as a measure of the efficiency of the generation process,
e.g. ``What fraction of the available energy are we extracting?''.  

At present, we consider only the energy flowing into the entrainment
region due to the ambient conditions, in particular, the incoming wind
and heat through the front hemisphere of a cylindrical region. We
consider a medium-sized (3m radius) dust-devil with an incoming
freestream velocity of 5 m/s. The surface temperature is 343 Kelvin,
with a specified inflow boundary layer bridging the ground temperature
to the ambient air conditions of 313 Kelvin. 
% cite this?
These numbers were chosen
based on information provided by the Georgia Tech field team in Arizona
during the summer of 2014.  

There are two forms of energy to consider: kinetic and enthalpy. We
begin by considering the kinetic energy flux through the front of the
apparatus. From the first law of thermodynamics we can express the
kinetic energy flux as a surface integral over the upstream face of the device, 
\begin{equation*}
\text{KE} = \int_{CS} \frac{\vec V^2}{2} \rho \vec V \cdot \hat n dA.
\end{equation*}
%
% could cite fluid dynamics book here
% pg. 239
%
We assume our freestream velocity has no components in y and z.
We assume that the variation in z for the streamwise velocity is only on
account of the thin boundary layer near the ground. We functionally
approximate this behavior using the common 7th power function for a
turbulent boundary layer,  
\begin{equation*}
  u(z) = U \text{ min }\left(\left(\frac{z}{\delta}\right)^7,1\right)
\end{equation*}
where U is the constant freestream velocity and $\delta$ the assumed
boundary layer thickness. Our integral can be solved to show, 
\begin{align*}
\text{KE} & = -R \rho U^3 \left[ z_{\text{max}} - \frac{10}{11}\delta.
\right]
\end{align*}

%%  & = \frac{1}{2} R \rho u^3 z_\text{max}
%%  \int^{\frac{3\pi}{2}}_{\frac{\pi}{2}} \text{cos}\theta d\theta \\
%%  & = -R \rho u^3 z_\text{max}. 
%% \end{align*}

The negative sign here indicates that the kinetic energy is flowing into
the surface, in opposition to the outward facing unit normal, $\hat
n$. Characteristic values for this analysis are, $u = 5$ m/s, $\rho =
1.225$ Kg/$m^3$, $R = 3$ m, and $z_{\text{max}} = 2.5$ m. The boundary
layer thickness is $\approx 10$ cm. This provides
an estimate of 1144.26 Watts as the incoming kinetic energy flux. Or,
approximately 1.14 kW.  

Now we estimate the gravitational potential energy flux by integrating 
the boussinesq term by the height of the vanes, 
\begin{align*}
  \text{Potential Energy Flux} & = \int_{-h}^0 u(z) \Delta \rho g z dz. \\
  & = \int_{-h}^0 u(z) \rho' g z^2 dz. 
\end{align*}
Where the substitution, $\Delta \rho = \rho' z$ was made. At 
this point we again separate the integral into two components, 
for the boundary layer and the constant freestream velocity region. 
\begin{align*}
  & = \rho' g \left[ \int_{-\delta}^{0} U \left( \frac{z}{\delta} \right)^7 z^2 dz 
      + \int_{-h}^{-\delta} U z^2 dz \right] \\
  & = -\rho' g U \left[ \frac{h^3}{3} - \frac{7}{30} \delta^3 \right].
\end{align*}
Furthermore, we
note that $\rho' = -\beta \rho_0 \Delta T$, resulting in, 
%
% cite monin-yaglom page 59
%
\begin{equation}
 \text{Power } = U \beta \rho_0 \Delta T g \left[ \frac{h^3}{3} -
					    \frac{7}{30} \delta^3
					   \right]. 
\end{equation}

Using $\rho_0 = 1.225$ Kg/$m^3$, $T_{\text{ref}}=313$m Kelvin, $\beta_T = 0.003194$
(This is just 1/$T_{\text{ref}}$), $g=9.81$ m/$s^2$, and a freesteam
velocity of five meters per second results in an
estimate of 30 Watts for the gravitational potential energy. 

This interesting result implies that the majority of the available
energy is attributable to kinetic energy, not the gravitational
energy. This is consistent with the results of Renno, who demonstrated
that the thermal energy present was not sufficient to account 
thermodynamic energy 
%
% finish me!
%
While the gravitational potential energy is a small fraction of the
energy available, that does not imply it is without significant impact. 

