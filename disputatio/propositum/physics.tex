\section{Physics of Dust-Devils}

In order to motivate how best to \textit{engineer} a synthetic
dust-devil, we first address what is known about the naturally occuring
phenomena. We therefore begin this section with a qualitative discussion
of dust-devils, followed by a review of the known physics and pertinent 
literature review. 

Our aim is to simulate the formation of synthetic dust devils in the
field. This requires a model of the ambient conditions for a
representative case, such as Arizona, where experimental data is
available from test have been performed. Furthermore, for this to be
more generally useful in the prediction of flows in a variety of
conditions, we need a model generally applicable to any flow near the
surface of the earth.  

This document details an analysis of surface fluid mechanics, and
develops a theory of turbulence in a thermally stratified medium. As we
are utilizing a RANS model with spatially variable diffusivity, we are
particularly interested formulating a model for these quantities. 

We are interested in the operation of the apparatus during the day. 
At these times, the atmospheric surface layer has the following character. 
Incident radiation from the sun largely does not interact with the
air, which is nearly transparent. Instead, this radiation is absorbed by
the ground, which causes a temperature rise. This results in a thermal
gradient between the hot ground and the cooler air. The warm ground
conducts heat to the air, causing an expansion and lowering the density
of the air. This reduced density air near the surface is driven upwards
by the force of buoyancy.  

For sufficiently large temperature gradients, these motions are
unstable, and as the warm air is driven upwards the flow will transition
to turbulence. 

\begin{itemize}
\item Qualitative discussion of Sinclair and physics of regime
\item phenomenological 
\item draw regions
\item eyewall
\end{itemize}
