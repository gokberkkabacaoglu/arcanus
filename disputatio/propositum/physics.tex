\section{Physics of Dust-Devils}
\label{sec:physics}

%% \begin{itemize}
%% \item \st{phenomenological description}
%% \item \st{regions/frequency of occurance}
%% \item \st{review of sinclair, kanak and renno and other pertinent literature}
%% \item \st{known physics of dust devils and other cyclonic structures}
%% \item \st{dust-devil/tornado/hurricane genesis}
%% \item \st{scaling discussion as estimate of energy}
%% \item \st{dust-Devil Generation Concept}
%% \item \st{motivation of computational modeling{
%% \item draw regions
%% %\item eyewall
%% \end{itemize}

In order to motivate how best to \textit{engineer} a synthetic
dust-devil, we first address what is known about the naturally occuring
phenomena. We therefore begin this section with a qualitative discussion
of dust-devils, followed by a review of the known physics and pertinent 
literature review. We then briefly outline how we intend to leverage
these physical processess as a method of usable energy generation. 


\subsection{Phenomenological Character of Dust-Devils}

%\begin{itemize}
%\item chart from sinclair of frequency (they are driven by delta t)
%\item turning direction independent
%\item characteristic velocities and sizes
  %\item genesis diagram (thermal plume) -- mention ambient angular vorticity
%\item structure diagram and description
%\end{itemize}

No rigorous defintion of a Dust devil exists. This cannot be attributed
to the rarity of the processes. Rather, the phenomena of dust devils are
remarkably ubiquitous. These whirlwinds have been
observed across a wide variety of terrains, climates and even on several
other planets\cite{Sinclair1969,Bluestein2004,JGR:JGR13978,JGRE:JGRE1660}. 
While it is difficult to pin down a precise definition, several features 
are characteristic of a dust devil. They are regions of
intense vorticity and rotation, coupled with strong upward motions. 
They are a self-sustaining vortex that maintains a funnel-like
chimney driven by hot air moving both upward and circularly. 
While they typically exist for only a few minutes, some have 
been observed to exist for significantly longer. While the velocities are 
typically several meters per second, 
%
% solen from: http://glossary.ametsoc.org/wiki/Fujita_scale
%
dust devils occasionally are strong enough to cause damage and injury,
with some reaching F1 on the Fujita Tornado intensity scale, with velocities 
between 33 and 49 $m/s$.
%
% F1 (moderate damage): 33-49 m s-1
%
% ``F1 - Surface of roofs peeled off; mobile homes pushed off foundations
% or overturned; moving autos pushed off road. ``
Diameters range from about one meter to greater than thirty.  Their
average height is about thirty meters, but a few have been observed 
as high as 1 km or more. There appears to be no preferred direction, and
they are observed to rotate anticyclonically as well as
cyclonically. Although the vertical velocity 
is predominantly upward, the flow along the a central axis of large dust devils
may be downward. 
%
% cite martian dust devils
%
%Moreover, actual dust devils have been photographed from orbit, with
%some of them as large as 1 to 2 kilometers across at their base and 10
%km tall. 
Similarly visibly structured phenomena have been spotted over water, in
intense forest fires, or even in cold or freezing environments. 
%
% good place to ref jacobson2005fundamentals
%
This is to say nothing of other similar cyclonic phenomena, such as
tornados and hurricanes. 

While the phenomena is pervasive, we can state that certain 
environmental conditions do impact the frequency of formation
of dust devils. Sinclair\cite{Sinclair1969} provides perhaps the most 
systematic investigation characterizing favorable conditions for formation. 

From these results we can conclude that dust devils are most likely 
to form at solar noon, the time of the greatest incident radiation 
on the ground. Furthermore, they are more likely to form in locations 
with a higher surface temperature. 
Moderate to high wind speeds (2-5 m/s) encourage dust 
devil genesis, but greater velocities (11 m/s) appear to impede formation. 
They are more likely to be observed in relatively flat locations, 
such as deserts. 

Actual measurements made inside a dust devil are limited. From what data
is available, the dust devil can be broken into two regions: a low surface 
layer and a higher invisid region. The low surface region is the 
principle location of radial inflow. 
It is at the top of this region that the flow 
reaches its peak velocity, with that peak dropping with increasing height. 
The strong radial and azimuthal velocities are drawn into a low pressure core 
where they gain vertical velocity and are lifted up. Earlier experimental and 
computational results have both observed a cooler downdraft in the very center 
of the dust devil. It is not clear what generates the azimuthal velocities. It is 
possible that ambient vorticity from objects is drawn into the vortex from far 
field, and which intensifies greatly owing to a $1/r^2$ dependence.

The higher region is characterized by a largely invisid potential flow region 
that characterized by warm air rising and circling around a cool, 
low pressure core. This region is typically many times larger in height than the
surface layer. While this region also has radial inflow, but it is 
significantly weaker than the lower region. Previous studies have found that
this region is relatively well described by a Rankine vortex model\cite{}. 
Both of these regions are sketched out as a simple 
cartoon in figure \ref{fig:cartoon}. 

  \begin{figure}[!htb]
    \begin{center}
     \includegraphics[width = 12 cm]{figs/lab_setup}
     \caption{Cartoon of the lower structure of a dust devil.}
     \label{fig:cartoon}
    \end{center}
  \end{figure}


\subsection{Estimate of Energy Scaling}

Here we provide a rough estimate of the energy
available to a Dust devil. There are two objectives behind this
analysis. The first is to provide justification for the concept of
extracting energy from these phenomena, with the reasoning that should
sufficient energy be available, then attemping to extract it might be
worthwhile. The second objective is to provide a simple analysis that
can serve as a measure of the efficiency of the generation process,
e.g. ``What fraction of the available energy are we extracting?''.  

At present, we consider only the energy flowing into the entrainment
region due to the ambient conditions, in particular, the incoming wind
and heat through the front hemisphere of a cylindrical region. We
consider a medium-sized (3m radius) dust devil with an incoming
freestream velocity of 5 m/s. The surface temperature is 343 Kelvin,
with a specified inflow boundary layer bridging the ground temperature
to the ambient air conditions of 313 Kelvin. 
% cite this?
These numbers were chosen
based on information provided by the Georgia Tech field team in Arizona
during the summer of 2014.  

There are two forms of energy to consider: kinetic and enthalpy. We
begin by considering the kinetic energy flux through the front of the
apparatus. From the first law of thermodynamics we can express the
kinetic energy flux as a surface integral over the upstream face of the device, 
\begin{equation*}
\text{KE} = \int_{CS} \frac{\vec V^2}{2} \rho \vec V \cdot \hat n dA.
\end{equation*}
%
% could cite fluid dynamics book here
% pg. 239
%
We assume our freestream velocity has no components in y and z.
We assume that the variation in z for the streamwise velocity is only on
account of the thin boundary layer near the ground. We functionally
approximate this behavior using the common 7th power function for a
turbulent boundary layer,  
\begin{equation*}
  u(z) = U \text{ min }\left(\left(\frac{z}{\delta}\right)^7,1\right)
\end{equation*}
where U is the constant freestream velocity and $\delta$ the assumed
boundary layer thickness. Our integral can be solved to show, 
\begin{align*}
\text{KE} & = -R \rho U^3 \left[ z_{\text{max}} - \frac{10}{11}\delta.
\right]
\end{align*}

%%  & = \frac{1}{2} R \rho u^3 z_\text{max}
%%  \int^{\frac{3\pi}{2}}_{\frac{\pi}{2}} \text{cos}\theta d\theta \\
%%  & = -R \rho u^3 z_\text{max}. 
%% \end{align*}

The negative sign here indicates that the kinetic energy is flowing into
the surface, in opposition to the outward facing unit normal, $\hat
n$. Characteristic values for this analysis are, $u = 5$ m/s, $\rho =
1.225$ Kg/$m^3$, $R = 3$ m, and $z_{\text{max}} = 2.5$ m. The boundary
layer thickness is $\approx 10$ cm. This provides
an estimate of 1144.26 Watts as the incoming kinetic energy flux. Or,
approximately 1.14 kW.  

We estimate the gravitational potential
energy flux by integrating the boussinesq term by the height of the vanes, 
\begin{align*}
  \text{Potential Energy Flux} & = \int_{-h}^0 u(z) \Delta \rho g z dz. \\
  & = \int_{-h}^0 u(z) \rho' g z^2 dz. 
\end{align*}
Where the substitution, $\Delta \rho = \rho' z$ was made. At 
this point we again separate the integral into two components, 
for the boundary layer and the constant freestream velocity region. 
\begin{align*}
  & = \rho' g \left[ \int_{-\delta}^{0} U \left( \frac{z}{\delta} \right)^7 z^2 dz 
      + \int_{-h}^{-\delta} U z^2 dz \right] \\
  & = -\rho' g U \left[ \frac{h^3}{3} - \frac{7}{30} \delta^3 \right].
\end{align*}
Furthermore, we
note that $\rho' = -\beta \rho_0 \Delta T$, resulting in, 
%
% cite monin-yaglom page 59
%
\begin{equation}
 \text{Power } = U \beta \rho_0 \Delta T g \left[ \frac{h^3}{3} -
					    \frac{7}{30} \delta^3
					   \right]. 
\end{equation}

Using $\rho_0 = 1.225$ Kg/$m^3$, $T_{\text{ref}}=313$m Kelvin, $\beta_T = 0.003194$
(This is just 1/$T_{\text{ref}}$), $g=9.81$ m/$s^2$, and a freesteam
velocity of five meters per second results in an
estimate of 30 Watts for the gravitational potential energy. 

This interesting result implies that the majority of the available
energy is attributable to kinetic energy, not the gravitational
energy. This is consistent with the results of Renno\cite{}, who demonstrated
that the thermal energy present was not sufficient to account 
thermodynamic energy 
%
% TODO: finish me!
%
While the gravitational potential energy is a small fraction of the
energy available, that does not imply it is without significant impact.
\todo{Finish discussion}%
\subsection{Dust Devil Generation Concept}

The preceeding discussion has provided justification for the idea of
using dust devils as a method of extracting ambient kinetic and
gravitational potential energy from the environment. This  
subsection now provides a brief discussion of how the physics of 
dust devils informs the generation of a synthetic variety. 

In contrast to the naturally occuring dust devils
our synthetic solar driven vortex, (SoV) design makes use of
control surfaces to dictate the
angle of incoming flow, and in doing so explicitly converts a portion of
the incoming radial flow in azimuthal velocity. These turning vanes also
serve as an anchor for the synthetic vortex, locking it into a small
region.  
  \begin{figure}[!htb]
    \begin{center}
     \includegraphics[width = 12 cm]{figs/lab_setup}
     \caption{Image of a possible two tier turning vane 
       configuration for generating synthetic dust devils.}
     \label{fig:cartoon}
    \end{center}
  \end{figure}

Our principle objective is to use a synthetic dust devil to produce 
useable work. In order to extract this energy, a turbine is placed 
at the top of the vanes, where the blades are driven by the dust devil's 
azimuthal and vertical velocities. 

While the turning vanes and turbine  
paradigm represents a reasonable starting point for design, the
parameter space of possible system configurations is large. It is
unclear how to engineer an effective SoV system. Important design
consideration include: 
\begin{itemize}
  \item How should the turning vanes be configured?
  \item How does the energy produced scale with system diameter?
  \item Could additional surfaces, such as a cone, capable of further 
    increasing energy output?
\end{itemize}

Questions such as these provide the principle impetus of using
computational fluid dynamics (CFD) in order to inform system design. The
parameter space of concievable system designs is far larger than can be
probed experimentally, and even if such a campaign were to be embarked
upon, it would be at significantly greater temporal and monitary
cost. The subseqent chapter will provide the mathematical basis by which
we model the system, so that we can then begin to discuss how we might
optimize it.  


