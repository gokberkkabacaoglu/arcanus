\begin{abstract}
\doublespacing

 Much of the solar energy incident on the Earth's surface is absorbed
 into the ground, which in turn heats the air layer above the surface.
 This buoyant air layer contains considerable gravitational potential
 energy. The energy in this layer can drive the formation of columnar vortices
 (``Dust Devils'') which  
 arise naturally in the atmosphere. A new energy harvesting approach
 makes use of this phenomenon by creating and anchoring the vortices
 artificially and extracting energy from them. In this research
 proposal, we explore the  characteristics of these vortices through
 numerical  simulation. Computational models of the turning vane system
 which generates the vortex and the turbine used to extract energy
 have been developed. 
 The formulation of these models and their validation
 against available experimental measurements will be discussed, as will
 the details of the columnar vortex structure and its interaction with
 the turbine. Preliminary results from these studies will also 
 be presented. 
 In addition, we introduce the computational models being used to
 optimize the system configuration in order to
 maximize the power extraction. These optimizations are designed to 
 assess the technological feasibility of the overall endeavor. 

 %This optimization and to characterize the effects of
 %environmental conditions such as cross winds and
 %topography. 
\end{abstract}

