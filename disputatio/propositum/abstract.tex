\begin{abstract}
\doublespacing

Buoyancy driven columnar vortices arise naturally in the atmosphere. A
 new energy harvesting approach makes use of this phenomenon by creating
 and anchoring the vortices artificially and extracting energy from
 them. In this talk, we explore the characteristics of these ``solar
 vortices'' through numerical simulation. Computational models of the
 turning vane system used to generate the solar vortex and the turbine
 used to extract energy have been developed. The formulation of these
 models and their validation against available experimental measurements
 will be discussed, as will the details of the columnar vortex structure
 and its interaction with the turbine. In addition, the computational
 models are being used to optimize the turning vane configuration and
 the turbine characteristics to maximize the power extraction, and to
 characterize the effects of environmental conditions such as cross
 winds and topography. Preliminary results from these studies will also
 be presented. 

\end{abstract}
