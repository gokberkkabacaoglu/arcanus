%%%%%%%%%%%%%%%%%%%%%%%%%%%%%%%%%%%%%%%%%%%%%%%%%%%%%%%%%%%%%%%%%%%%%%%%%%%%%%
\section{Motivation}
\label{sec:motivation}

A vehicle reentering Earth's atmosphere requires a thermal protection system
(TPS) to mitigate aerothermodynamic heating.  Gauging reentry heat load is
critical to mission success.  Undersizing a TPS at best destroys expensive
equipment and at worst causes loss of life.  Oversizing a TPS increases vehicle
weight and fuel costs and therefore reduces available payload.  Decision makers
need these heating predictions with quantified uncertainty so they may balance
reliability requirements against cost constraints.

Turbulence and laminar-turbulent transition enter critically into this balance.
Turbulence in the fluid boundary layer around a vehicle intensifies heating
because turbulent mixing enhances momentum, energy, and chemical species
transport to the TPS\@.  Recent coupled multiphysics studies by
\citet{Bauman2011Loose} and \citet{Stogner2011Uncertainty} showed that
ablative TPS predictions are highly sensitive to uncertainty in turbulence model
calibration parameters.  Further, while low-turbulence freestream conditions allow at
least the stagnation point region within the flow to be laminar, prediction
efforts often assume these boundary layers are fully turbulent.  Both
incorrectly applying turbulence models to laminar regions and neglecting the
downstream laminar-turbulent transition processes add markedly to heat load
uncertainty.  Transition models may be employed to relax this assumption by
accounting for transitional flow.  However, the extreme sensitivity of
transition phenomena to the upstream environment (see, for example,
\citet{Federov2011Transition}) brings with those models another uncertainty
penalty.

%%%%%%%%%%%%%%%%%%%%%%%%%%%%%%%%%%%%%%%%%%%%%%%%%%%%%%%%%%%%%%%%%%%%%%%%%%%%%%
\section{Objectives}

This work aims to reduce turbulence- and transition-driven uncertainty in
aerothermodynamic heating predictions for blunt-bodied reentry vehicles in two
ways.  The first way will reduce the uncertainty entering through the turbulence
model calibration parameters.  The second way will reduce the uncertainty
arising from incorrectly treating laminar regions as fully turbulent.

%\subsection[Producing High-Quality Data for\\Compressible Turbulence Model Calibration]
%           {Producing High-Quality Data for Compressible Turbulence Model Calibration}

First, we aim to use direct numerical simulation (DNS) of the compressible
Navier--Stokes equations to generate high-quality supersonic boundary layer data
for turbulence model calibration.  DNS was selected because the technique
produces data uncertainties limited only by the available computing resources.
We have designed and implemented a new, well-verified Fourier/B-spline
pseudospectral DNS code called Suzerain employing ``slow growth,'' a spatiotemporal boundary layer
homogenization approach by \citet{Topalian2011Slow, Topalian2014Temporal,
Topalian2014Spatiotemporal}, to efficiently generate turbulence statistics with
accurately quantified uncertainties.  The code is used to create a rich database
of compressible turbulence statistics for use by the reentry community.  In
addition to the long-lived, public datasets we generate, our modern DNS code can
serve others as a robust, extensible platform for computational turbulence
research.

%\subsection[Characterizing Turbulence-Sustaining Regions\\on Blunt-Bodied Reentry Vehicles]
%           {Characterizing Turbulence-Sustaining Regions on Blunt-Bodied Reentry Vehicles}

Second, we aim to detect which regions of an ablative thermal protection system
on a blunt-bodied vehicle can sustain turbulence.  Given the strength of the
favorable pressure gradients found in these flows, it is reasonable to expect
that a contiguous region extending some distance radially from the stagnation
point simply cannot maintain turbulence.  Rather than taking the classical
transition modeling approach and seeking where laminar-turbulent transition
occurs, this study instead aims to map where turbulence cannot survive.  The
spatiotemporal boundary layer DNS code is reused to
parametrically explore
pointwise flow conditions found within simulations like those of
\citet{Bauman2011Loose}.  Fully turbulent fields are initialized and evolved at
local conditions taken from such simulations.  We say the conditions cannot
sustain turbulence if the field relaminarizes.  By exploring this parameter
space, we aim to discover where turbulence models might not be employed when
engineering practitioners simulate these reentry flows.

%%%%%%%%%%%%%%%%%%%%%%%%%%%%%%%%%%%%%%%%%%%%%%%%%%%%%%%%%%%%%%%%%%%%%%%%%%%%%%
\section{Outline}

This work is organized as follows:

\autoref{sec:review} provides background on the uncertainties arising from
applying turbulence models within reentry applications, how calibration data
impacts these uncertainties, and evaluates potential sources for obtaining that
data.  It further discusses uncertainties arising from transition phenomena and
proposes a concrete scenario for study based on the Orion MPCV\@.

\autoref{sec:model} summarizes the mathematical models required to pursue
the aims of the thesis.  \autoref{sec:techniques} details the computational
techniques used to apply these models while \autoref{sec:software} describes
their software implementation.

\autoref{sec:bldata} presents new direct numerical simulations of
spatiotemporally homogenized turbulent boundary layers with features similar to
those found on the Orion MPCV thermal protection system.  It investigates
the character of the turbulence, presents Favre-averaged equation budgets,
and communicates the information necessary to use the data for turbulence
model calibration.

\autoref{sec:relam} detects turbulence-sustaining regions on the Orion MPCV
using spatiotemporally homogenized boundary layers.  The study methodology
is discussed followed by a collection of results corresponding to locations
on the MPCV thermal protection system.

Finally, \autoref{sec:conclusions} summarizes the conclusions of this thesis
and presents recommendations for future work.

%%%%%%%%%%%%%%%%%%%%%%%%%%%%%%%%%%%%%%%%%%%%%%%%%%%%%%%%%%%%%%%%%%%%%%%%%%%%%%
\section{Contributions}

% THIS WORK HAS MADE THE FOLLOWING CONTRIBUTIONS ???
% The contributions of this work include the following:
This work has made the following contributions:

\begin{enumerate}
    \item Creation of a well-verified, openly available pseudospectral
    code for the direct numerical simulation (DNS) of sub- through
    supersonic turbulent boundary layers using ``slow growth''
    homogenization techniques.

    \item Generation and characterization of metadata-rich DNS data,
    with well-quantified sampling uncertainty, for sub- and supersonic
    spatiotemporally homogenized turbulent boundary layers on cold,
    transpiring walls and subject to strong favorable pressure gradients.

    \item Design of a novel DNS experiment to determine where on a
    vehicle surface turbulence can be sustained without requiring the
    flight environment to be sufficiently well-understood that transition
    modeling can be reliably applied.

    \item Application of this novel DNS experiment to conditions
    from the NASA Orion Multi-Purpose Crew Vehicle ablative thermal
    protection system during atmospheric reentry from the International
    Space Station.
\end{enumerate}
