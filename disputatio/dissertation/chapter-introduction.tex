%\section{Introduction / Executive Summary}
\label{sec:intro}

%
%rhys used:
%          1) motivation
%          2) objectives
%          3) outline
%          4) contributions
%


\section{Motivation}

Renewable energy is critical to our environmental, economic, and
national security. Global demand for energy is projected to rise 56\% by
2040\cite{energy-outlook}, which will coincide with our national
reliance on fossil fuel-based power plants for the bulk of our
electricity generation. There is a critical need for safe, clean, and 
cost-effective alternatives to coal, such as wind, solar, hydroelectric,
and geothermal power. These technologies will simultaneously reduce
carbon dioxide emissions and help position the U.S. as a leader in the
global renewable energy industry. 
% \cite{arpa-e}
% proposal
%
This thesis documents a research effort that performed a numerical 
investigation and design optimization of a novel renewable energy concept. 

Much of the solar energy incident on the Earth's surface is absorbed
into the ground, which in turn heats the air layer above the surface.
This buoyant air layer contains considerable gravitational potential
energy. 
With nearly one-third of global land mass covered by deserts, there are huge
untapped regions for capturing solar heat (about 200 W/$\text{m}^2$
averaged over a 24-hour day, and up to 1000 W/$\text{m}^2$
peak)\cite{Hoyt197827}. The available power is competitive in magnitude
with worldwide power generation from fossil sources. If a technology
could effectively extract this energy, it would result in a low-cost,
scalable approach to electrical power generation that could create a new
class of renewable energy ideally suited for arid regions.  

How then, is one to efficiently extract this gravitational potential
energy and convert it into usable work? We turn to Nature to provide a 
guide, with the observation that there are natural objects that provide
precisely this mechanism. Namely, naturally occurring ``dust devils'' 
characterized by a vertically stratified, ground-heated air layer
that produces a coherent columnar vortex. These ``dust devil''s are
ubiquitous, naturally appearing in regions as diverse as Arizona,
Siberia, over water, or even
Mars\cite{Sinclair1969,ROG:ROG1635,JGRE:JGRE1660}.  
 % arizona, indiana, oregon, yukon, colorado
They are observed to occur over a wide range of length scales (1 - 30
meters) with large variations in velocities (1 to over 40
m/s)\cite{Sinclair1969}. 

The basic idea behind the proposed energy harvesting approach is to
convert the potential energy in this buoyant air layer to kinetic energy
in an anchored vortex, and to use that kinetic energy to drive a
vertical-axis turbine coupled with an electric generator  to
produce electrical power. 
The Solar-Driven Vortex (SoV) phenomena has been demonstrated in
an experimental laboratory by our partners at Georgia
Tech\cite{mark-thesis}. To move beyond proof-of-concept, Computational
Fluid Dynamics (CFD) was used to simulate the SoV. 

%These simulations have resulted in a
%greatly enhanced output of Power extracted by the SoV. 

\section{Objectives}

The objective of this thesis is to assess the technological feasibility
of using synthetic columnar vortices to generate usable energy. 
We considered feasibility in the context of technological capability,
not economic cost. The technological estimation is accomplished through
the use of CFD to exhaustively explore the predicted power extracted
over a wide range of system configurations. 

CFD was selected because the range of system configurations is far too
large (and prohibitively expensive in time and money) to construct and
test in the field. Additionally, the uncertainties in predictions
attributable to variations in the ambient conditions present in any field
condition are substantial. Instead, CFD permits rapidly iterating
through system design ideas with a precisely controlled, and consistent,
scenario. However, a challenge of this project is that this particular
system has never been simulated. Furthermore, existing models and
software capabilities are not sufficiently advanced as to reproduce the 
conditions needed for this campaign. 

Therefore, mathematical models that describe the ambient atmospheric 
conditions where dust devils typically form, have been produced. A novel 
representation of the SoV system geometry that is sufficiently flexible
to permit cost-effective iteration in designs has been developed. The
models have been instantiated in software and run on supercomputers. 
The output has been successfully validated against existing experimental
data. Furthermore, simulations have been performed to provide
fundamental insight into the driving dynamics of the system and
generated high resolution data, which is largely experimentally
inaccessible. This data has been used to rapidly optimize the geometry
and configuration of the SoV apparatus. These results have lead to a 
predicted configuration for experimental testing that generates several
kilowatts of power. 

% This document describes a course of investigation designed to broadly
% probe the design space and provide a definitive assessment of the
% technological feasibility of the entire synthetic columnar vortex concept. 
%
% all steps to this... math, software, etc. 

\section{Outline}

This dissertation is organized as follows. Chapter \ref{sec:physics} begins
with a discussion of the naturally occurring phenomenon, the presently
understood dynamics of dust-devils and similar columnar vortices, and
the implications for systems designed to generate their synthetic
counterparts.  
Chapter \ref{sec:mathmodel} outlines a mathematical model of
the entire system, and Chapter \ref{sec:software} discusses the
algorithms and software implementation used to simulate the
system. Chapter \ref{sec:validation} reviews the validation of
these resulting simulations against existing experimental data and high 
fidelity simulations.  
Chapter \ref{sec:results} examines the simulation results in detail, to
discern the physical processes driving the SoV. Chapter \ref{sec:field}
details final system design, as well as the predicted performance in the
field.  
%
%
%as well as  the  
%several examples of a numerical optimization of the apparatus. 
%
Finally, with the preceding sections outlining the present simulation
capabilities and predictions, Chapter \ref{sec:conclusions} concludes
with a discussion of the ultimate technological feasibility of the
venture and recommendations for future work. 

% want this?
%
%\section{Contributions}