%\section{Introduction / Executive Summary}
\label{sec:intro}

%
%rhys used:
%          1) motivation
%          2) objectives
%          3) outline
%          4) contributions
%


\section{Motivation}

Renewable energy is critical to our environmental, economic, and
national security. Global demand for energy is projected to rise 56\% by
2040\cite{energy-outlook}, as is our national reliance on fossil
fuel-based power plants for the bulk of our electricity generation. 
There is a critical need for safe, clean, and
cost-effective alternatives to coal, such as wind, solar, hydroelectric,
and geothermal power. These technologies will simultaneously reduce
carbon dioxide emissions and help position the U.S. as a leader in the
global renewable energy industry. 
% \cite{arpa-e}
% proposal
%
This proposal details a research plan to perform a numerical
investigation and design optimization of a novel renewable energy concept. 

Much of the solar energy incident on the Earth's surface is absorbed
into the ground, which in turn heats the air layer above the surface.
This buoyant air layer contains considerable gravitational potential
energy. 
With nearly one-third of global land mass covered by deserts, there are huge
untapped regions for capturing solar heat (about 200 W/$\text{m}^2$
averaged over a 24-hour day, and up to 1000 W/$\text{m}^2$
peak)\cite{Hoyt197827}. The available power is competitive in magnitude
with worldwide power generation from fossil sources. If a technology
could effectively extract this energy, it would result in a low-cost,
scalable approach to electrical power generation that could create a new
class of renewable energy ideally suited for arid regions.  

How then, is one to efficiently extract this gravitational potential
energy and convert it into usable work? We turn to Nature to provide a 
guide, with the observation that there are natural objects that provide
precisely this mechanism. Namely, naturally occurring ``dust devils'' 
characterized by a vertically stratified, ground-heated air layer
that produces a coherent columnar vortex. These ``dust devil''s are
ubiquitous, naturally appearing in regions as diverse as Arizona,
Siberia, over water, or even
Mars\cite{Sinclair1969,ROG:ROG1635,JGRE:JGRE1660}.  
 % arizona, indiana, oregon, yukon, colorado
They are observed to occur over a wide range of length scales (1 - 30
meters) with large variations in velocities (1 to over 40
m/s)\cite{Sinclair1969}. 

The basic idea behind the proposed energy harvesting approach is to convert the 
potential energy in this buoyant air layer to kinetic energy in an
anchored vortex, and to use that kinetic energy to drive a
vertical-axis turbine coupled with an electric generator  to
produce electrical power. 
The Solar-Driven Vortex (SoV) phenomena has been demonstrated in
an experimental setup by our partners at Georgia Tech. However, to 
move beyond proof-of-concept, Computational Fluid 
Dynamics (CFD) is needed to simulate the SoV. Such simulations will
provide fundamental insight into the 
driving dynamics of the system and generate high resolution data, which is
experimentally inaccessible, to be used to rapidly optimize the
geometry and configuration of the SoV apparatus. 

%This is a considerable effort. 

%
% TODO: add table of 'state of the art' or novel work performed
%

\section{Outline}

The objective of this project is to assess the technological feasibility of 
using synthetic columnar vortices to generate usable energy. 
This proposal begins in Section \ref{sec:physics} with a discussion of the 
naturally occurring phenomenon, the presently understood dynamics of
dust-devils and similar columnar vortices, and the implications for systems
designed to generate their synthetic counterparts. 
In Section \ref{sec:mathmodel}, we outline a mathematical model of
the entire system, and in Section \ref{sec:software}, we discuss the
algorithms and software implementation used to simulate the
system. Section \ref{sec:validation} discusses the 
validation of these results against existing experimental data and high
fidelity simulations. Section \ref{sec:results} details the preliminary
predictions of system performance in the field, as well as detailing the 
several examples of a numerical optimization of the apparatus. Finally, with the 
preceding sections outlining the present simulation capabilities, 
Section \ref{sec:proposed_work} proposes a course of investigation
designed to broadly probe the design space and provide a
definitive assessment of the technological feasibility of the entire 
synthetic columnar vortex concept. 

%details a short validation
%study performed by comparing between the available experimental
%measurements and the simulations results. 

%For these simulations to be generally useful, they must first
%be validated against existing experimental data and high fidelity
%simulations. These models will then explore regimes and scales where no
%experimental measurements presently exist. 
%Characterizing the
%uncertainty of predictions resulting from extrapolation is a critical
%component in enabling reliable assessments of field performance of the
%SoV, as it will guide the commercialization strategy of the product. 
