\label{sec:conclusions}

%
% be sure to mention that this thesis is a success of CFD/CAD
% computational optimization and computational design
%

%
% bob mentioned today that one should considered feasibility in the 
% context of just technological scaling, no cost. 
%

\section{Summary of the Present Work}

The objective of this thesis is to provide a definitive assessment of
the technological feasibility of the entire synthetic columnar vortex
concept as a means of generating usable energy. The previous sections
outlined the present state of the simulation capability. In doing so, we
have discussed the physics that influence dust devils formation, our
particular mathematical models for the ambient conditions, and the
formulation for the SoV vanes, cone and turbine. We summarized the
numerical discretizations used, the software stack and the calibration,
verification and validation of these components. The purpose of these
sections was to communicate two major points. The first is that an
accurate, verified and validated simulation capability has been
developed that can quickly investigate a wide variety of system and
scenario settings at a modest computational cost. The second point is
that we have developed heuristics that permit optimization of any
baseline SoV configuration to a local maximum of energy production, as
measured by kinetic energy flux through the top of the SoV vanes.  

These two points were capabilities developed to support the principle
objective of this work, which is to explore a large space of possible
system configurations and geometries to discover the globally optimal
structure of the SoV apparatus. Coupled with the scaling analysis
presented in Chapter \ref{sec:physics}, we are now able to predict
the conditions (if any) under which the SoV apparatus will be
technologically competitive with other sources of renewable energy.  

This has also permitted investigating the physics of the apparatus, to
assess how closely the synthetic dust devils mimic the natural
variety. 

%
% probably dont want this...
%
\section{System Feasibility Assessment}


This is designed to assess feasibility, and at this time no 
actual experimental validation accompanies the computational
results. Furthermore, it must also be emphasized that feasibility is
focused on technical viability, namely energy produced by the apparatus,
and does not include an economic assessment. In other words, it is
currently believe that that the SoV does produce usable energy, 
but the design required to do could be prohibitively expensive, and
therefore not economically competitive with existing technologies. 

This thesis is nevertheless a success with regards to the objectives of
development of a simulation capability. The optimization and 


physics hints that multiple tiers are necessary to make a sort of
continuous set of entrainment work

might mention that more expensive models are def possible\todo{multiple turbines}

\section{Future Work}


fundamental fluid structure? the failure of several models during the
optimization work performed here 

testbed for cyclonic phenomena

% \subsection{Additional Investigations}

% % 
% % control for inter unit spacing
% % 
In addition to the system configuration, it would be interest to
consider the effect of local conditions on SoV
performance. Characterizing the impact of variations in ambient
conditions on the SoV\todo{finish me}

% will guide the commercialization strategy of the
% product, by determining optimal install locations across the country. It
% is therefore desirable to have models that are capable of accounting for
% variation in field conditions, such as solar input, cross-winds and
% topography. Furthermore, it is expected that large ``farms'' of SoVs
% (akin to the wind and solar farms for wind turbines and photovoltaics,
% respectively) may be used by commercial or utility-scale energy
% generation. For this to be effective,  the inter-unit spacing must also
% be optimized, as a single SoV collects from a large area. These
% computations will guide commercialization planning, where
% decision-makers will need to assess optimum unit size, spacing, and
% geographic location for utility-scale deployment.   
