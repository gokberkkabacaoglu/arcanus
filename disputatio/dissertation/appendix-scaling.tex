\label{scaling}

As mentioned in Chapter~\ref{sec:physics}, the mechanical power
available for extraction is the flux of kinetic energy through a
vertical surface where one could attach a turbine, for instance, 

\begin{equation}
 P = \frac{\rho }{2} \int V_z (V_{\theta}^2 + V_z^2 ) dA. 
\end{equation}

Thus, with the velocity field we can determine the energy
flux. As in Figure~\ref{fig:sinclair_profile}, we can extract the
velocity fields from a naturally occurring dust devil as a guide of the 
representative power contained within. Concomitantly (as noted in
Sinclair) the azimuthal velocity closely follows a Rankene vortex model,   

\begin{equation}
 V_{\theta} = 
  \begin{cases}
   \frac{V_0 r}{R} \quad r < R \\
   \frac{V_0 R}{r} \quad r > R \\
  \end{cases}
\end{equation}

If we assume that $V_z$ adheres to Sinclair's observation that $V_z \approx
V_{\theta} \approx V_0$ for $r < R$ and $V_z=\frac{V_0 R}{r}$ for $r > R$,
the integral can be solved, 
\begin{eqnarray}
 P =& \frac{1}{2} \rho \int_0^{2\pi}\int_0^{\infty} V_z \, (V^2_z +
  V_{\theta}^2)\, dr \, d\theta \\ 
 =& \pi \rho \int_0^{\infty} V_z \, (V^2_z + V_{\theta}^2)\, dr \\
 =& \pi \rho \left( \int_0^R (V_0^3 + V_0 V_{\theta}^2)\,dr +
	      \int_R^{\infty} V_{\theta} (V_{\theta}^2 + V_{\theta}^2)\,dr 
	     \right) \\
 =& \pi \rho \left( \int_0^R (V_0^3)\, dr + \int_0^R (V_0 V_{\theta}^2)\,dr +
	      \int_R^{\infty} V_{\theta} (V_{\theta}^2 + V_{\theta}^2) \,dr 
		     \right) \\
 =& \pi \rho \left( \int_0^R (V_0^3)\, dr + \int_0^R (V_0 V_{\theta}^2)\,dr +
	      2 \int_R^{\infty} V_{\theta}^3 \,dr 
		     \right) \\
 =& \pi \rho \left( R\, V_0^3 + V_0 \int_0^R (\frac{V_0 r}{R})^2\,dr +
	      2 \int_R^{\infty} (\frac{V_0 R}{r})^3 \,dr 
		     \right) \\
 =& \pi \rho \left( R\, V_0^3 + V_0 \frac{1}{3} \frac{V_0^2 r^3}{R^2}\rvert_0^R -
	      2 \frac{1}{2} \frac{V_0^3 R^3}{r^2}\rvert_R^{\infty}
		      \right) \\
 =& \pi \rho \left( R\, V_0^3 + \frac{1}{3} V_0^3 R + V_0^3 R \right)\\
 =& \frac{7}{3}\pi \rho R\, V_0^3
\end{eqnarray}

%
%\bigintssss 
%

With, 
\begin{eqnarray}
 \rho \approx& 1.225 \,\text{kg}/\text{m}, \\
 R \approx& 5 \,\text{m}, \\
 V_0 \approx& 10 \,\frac{\text{m}}{\text{s}},
\end{eqnarray}
we arrive at our estimate of 45 kW.

%
% >>> (10*10*10*3.14*7/3. * 1.225*5)/1000.
% 44.87583333333334
%