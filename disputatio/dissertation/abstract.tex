\begin{abstract}
\doublespacing

 Much of the solar energy incident on the Earth's surface is absorbed
 into the ground, which in turn heats the air layer above the surface.
 This buoyant air layer contains considerable gravitational potential
 energy. The energy in this layer can drive the formation of columnar
 vortices (``Dust Devils'') which arise naturally in the atmosphere. A
 new energy harvesting approach 
 makes use of this phenomena by creating and anchoring the vortices
 artificially and extracting energy from them. In the research
 proposed here, we will explore the  characteristics of these vortices
 through numerical simulation. Computational models of the turning vane
 system which generates the vortex and the turbine used to extract
 energy have been developed and are presented here. These models have
 been tested against available experimental measurements. 
 Preliminary results from these studies are also
 presented, as well as initial details of the columnar vortex structure. 
 In addition, we introduce the approach used to optimize the
 system configuration to maximize the power extraction. The objective of
 this work is to explore a wide variety of configurations to assess the
 technological feasibility of the overall endeavor.   

% focus on: 
%
% 1) what is being proposed
%
% 2) how it will be done
%


\end{abstract}

