\label{sec:model}
This chapter summarizes the nondimensional mathematical models used in the
present work.  First the governing Navier--Stokes are shown.  Reynolds
and Favre averaging is briefly defined followed by the form of the
Favre-averaged Navier--Stokes equations used.  Lastly, the spatiotemporal
homogenization forcing terms due to \citep{Topalian2014Spatiotemporal}
are presented.  One can find the underlying derivations in
Appendix~\ref{sec:derivation}.

\section{The Governing Navier--Stokes Equations}
\label{sec:goveqn}

The flow physics are modeled using the unsteady, three-dimensional,
compressible Navier--Stokes equations.  These continuum equations arise from
applying conservation of mass, momentum, and energy to a Newtonian, perfect
gas.  The model assumes that the first viscosity~$\mu$ obeys a power law in
temperature~$T$, the other viscosity~$\lambda$ is a constant multiple of $\mu$,
heat conduction through the gas obeys Fourier's law, and momentum and thermal
diffusivity are related by a constant Prandtl number.  For simplicity,
aerothermochemical effects are neglected.

The governing equations may be written in nondimensional form as
\begin{subequations}
\label{eq:nondim_model}
\begin{align}
  \label{eq:nondim_continuity}
  \frac{\partial\!}{\partial\!t}\rho{}
 =
 &- \nabla\cdot\rho{}u
  + \Ssd_\rho
  \\
  \label{eq:nondim_momentum}
  \frac{\partial\!}{\partial\!t}\rho{}u
 =
 &- \nabla\cdot(u\otimes\rho{}u)
  - \frac{1}{\Mach^{2}} \nabla{} p
  + \frac{1}{\Reynolds} \nabla\cdot\tau
  + f
  + \Ssd_{\rho{} u}
  \\
  \frac{\partial\!}{\partial\!t} \rho{}E
 =
 &- \nabla\cdot\rho{}Eu
  + \frac{1}{\Reynolds\,\Prandtl\,\left( \gamma - 1 \right)}
    \nabla\cdot\mu\nabla{} T
\notag\\ \label{eq:nondim_energy}
 &- \nabla\cdot{} p u
  + \frac{\Mach^{2}}{\Reynolds} \nabla\cdot\tau{} u
  + \Mach^{2} f \cdot{} u
  + q_{b}
  + \Ssd_{\rho{} E}
\end{align}
along with the constitutive relationships
\begin{align}
  \label{eq:nondim_constitutive}
  p &= \left(\gamma-1\right) \left(
    \rho{}E - \frac{\Mach^{2}}{2}\rho{}u^{2}
  \right)
  &
  T &= \gamma{} \frac{p}{\rho}
  &
  a &= \sqrt{T}
  &
  h &= \frac{T}{\gamma-1}
\end{align}
\begin{align}
  \mu &= T^{\beta}
  &
  \lambda &= \left(\alpha-\frac{2}{3}\right)\mu
  &
  \tau &=  \mu\left(\nabla{}u+\trans{\nabla{}u}\right)
         + \lambda\left(\nabla\cdot{}u\right) I
\end{align}
where the nondimensional free parameters
\begin{align}
  \Reynolds &= \frac{\rho_{0}u_{0}l_{0}}{\mu_{0}}
  &
  \Mach &= \frac{u_{0}}{a_{0}}
  &
  \Prandtl &= \frac{\mu_{0}C_{p}}{\kappa_{0}}
\end{align}
\end{subequations}
are the Reynolds, Mach, and Prandtl numbers, respectively.  Other free
parameters include the ratio of specific heats $\gamma$ and the viscosity power
law exponent $\beta$.  The von~K\'arm\'an relationship for the
Knudsen number becomes
\begin{align}
  \Knudsen &= \frac{\Mach}{\Reynolds}\sqrt{\frac{\gamma\pi}{2}}
\end{align}
where the present continuum assumptions are justified when
${\Knudsen\ll{}1}$.  The nondimensionalization requires some dimensional
reference density~$\rho_{0}$, length~$l_{0}$, velocity~$u_{0}$, and
temperature~$T_{0}$.  Other references quantities are defined as follows:
\begin{subequations}
\label{eq:basic_references}
\begin{align}
  t_{0} &= \frac{l_{0}}{u_{0}}
  &
  a_{0} &= \sqrt{\gamma{}RT_{0}}
  &
  p_{0} &= \rho_{0} a_{0}^{2}
  &
  E_{0}, H_{0}, h_{0} &= a_{0}^{2}
  \\
  \mu_{0},\lambda_{0} &= \mu\!\left( T_{0} \right)
  &
  \tau_{0} &= \frac{\mu_{0}u_{0}}{l_{0}}
  &
  f_{0} &= \frac{\rho_{0}u_{0}}{t_{0}}
  &
  q_{0} &= \frac{\rho_{0}a_{0}^{2}}{t_{0}}
\end{align}
\begin{align}
  {\Ssd_{\rho}}_0 &= \frac{\rho_{0}}{t_0}
  &
  {\Ssd_{\rho u}}_0 &= \frac{\rho_{0} u_0}{t_0}
  &
  {\Ssd_{\rho E}}_0 &= \frac{\rho_{0} E_0}{t_0}.
\end{align}
\end{subequations}
The terms $f$ and $q_b$ accommodate problem-specific momentum and total energy
forcing.  When employed, boundary layer homogenization is accomplished
through slow growth terms $\Ssd_\rho{}$, $\Ssd_{\rho{} u}$, and $\Ssd_{\rho{}
E}$ which take forms similar to the right hand side of~\eqref{eq:tsg_general}.

The bulk viscosity,
\begin{align}
  \mu_{B} &= \lambda + \frac{2}{3}\mu,
\end{align}
and the deviatoric component of the strain rate tensor,
\begin{align}
  S &= \varepsilon - \frac{1}{3} \trace\left(\varepsilon\right) I
     = \frac{1}{2}\left(\nabla{}u + \trans{\nabla{}u}\right)
     - \frac{1}{3}\left(\nabla\cdot{}u\right)I,
\end{align}
alternatively may be used to write $\tau$ as
\begin{align}
  \tau &= 2 \mu S + \mu_B  \left( \nabla\cdot{}u \right) I.
\end{align}
The final free parameter $\alpha$ then controls the bulk viscosity according to
\begin{align}
\mu_{B} &= \alpha \mu.
\end{align}
Setting $\alpha=0$ recovers Stokes' hypothesis.  The kinematic and bulk
kinematic viscosities
\begin{align}
 \nu &= \frac{\mu}{\rho} & \nu_{B} &= \frac{\mu_{B}}{\rho}
\end{align}
will be used at times to simplify notation.  This completes the description of
the model which is said to be ``closed'' because knowing $\rho$, $u$, and $E$
permits advancing that state in time.

\section{The Favre-Averaged Navier--Stokes Equations}
\label{sec:statevo}

As turbulence is chaotic, reporting a statistical description of
its behavior is essential.
With only additional modest mathematical assumptions, the above instantaneous
model may be manipulated to describe the evolution of mean quantities.
%
Notationally, the expectation or ``Reynolds average'' of a generic flow
variable $q$ is written~$\bar{q}$.  The density-weighted expectation
or ``Favre average'' is defined by
\begin{align}
  \tilde{q} &= \overline{\rho{}q}/\bar{\rho}.
\end{align}
Fluctuations about the mean and the density-weighted mean are
denoted
\begin{align}
  q'  &\equiv q - \bar{q},
  &
  q'' &\equiv q - \tilde{q},
\end{align}
respectively.
Reynolds averaging commutes with differentiation under mild smoothness
assumptions.  Here the common convention that taking Favre fluctuations,
$\left(\cdot\right)''$, has higher precedence than differentiation,
$\nabla\left(\cdot\right)$, has been adopted.  Additional background on these
two averaging approaches can be found in
\autoref{sec:averagingtechniques}.

Assuming that all required expectations are finite and that Reynolds averaging
commutes with differentiation whenever necessary, the model of
\autoref{sec:goveqn} gives rise to the unsteady Favre-averaged
Navier--Stokes (FANS) equations:
\begin{subequations}
\label{eq:fans_all}
\begin{align}
    \frac{\partial\!}{\partial\!t}\bar{\rho}
=
 &- \nabla\cdot\bar{\rho}\tilde{u}
  + \overline{\Ssd_{\rho{}}}
\label{eq:fans_mass}
\\
    \frac{\partial\!}{\partial\!t}\bar{\rho}\tilde{u}
=
 &- \nabla\cdot(\tilde{u}\otimes\bar{\rho}\tilde{u})
  - \frac{1}{\Mach^2}\nabla{}\bar{p}
  + \nabla\cdot\left(
        \frac{\bar{\tau}}{\Reynolds}
      - \bar{\rho} \widetilde{u''\otimes{}u''}
    \right)
  + \bar{f}
  + \overline{\Ssd_{\rho{} u}}
\label{eq:fans_mom}
\\
  \frac{\partial\!}{\partial\!t} \bar{\rho}\tilde{E}
=
 &- \nabla\cdot\bar{\rho}\tilde{H}\tilde{u}
  + \Mach^{2} \nabla\cdot\left(
        \left(
            \frac{\bar{\tau}}{\Reynolds}
          - \bar{\rho} \widetilde{u''\otimes{}u''}
        \right) \tilde{u}
      - \frac{1}{2}\bar{\rho}\widetilde{{u''}^{2}u''}
      + \frac{\overline{\tau{}u''}}{\Reynolds}
    \right)
\notag\\
 &+ \frac{1}{\gamma-1} \nabla\cdot\left(
      \frac{%
         \bar{\mu} \widetilde{\nabla{}T}
       + \bar{\rho} \widetilde{\nu'' \left(\nabla{}T\right)''}
      }{\Reynolds\Prandtl}
      - \bar{\rho} \widetilde{T''u''}
    \right)
\notag\\
 &+ \Mach^{2} \left(
        \bar{f}\cdot\tilde{u}
      + \overline{f\cdot{}u''}
    \right)
  + \bar{q}_b
  + \overline{\Ssd_{\rho{} E}}
\label{eq:fans_energy}
\\
    \frac{\partial\!}{\partial\!t}\bar{\rho}k
=
 &- \nabla\cdot\bar{\rho}k\tilde{u}
  - \bar{\rho} \widetilde{u''\otimes{}u''} : \nabla\tilde{u}
  - \frac{\bar{\rho} \epsilon}{\Reynolds}
  + \nabla\cdot\left(
        -\frac{1}{2}\bar{\rho} \widetilde{{u''}^{2}u''}
      + \frac{\overline{\tau{}u''}}{\Reynolds}
    \right)
\notag\\
 &+ \frac{1}{\Mach^2} \left(
        \bar{p}\nabla\cdot\overline{u''}
      + \overline{p' \nabla\cdot{}u''}
      - \frac{1}{\gamma} \nabla\cdot\bar{\rho} \widetilde{T''u''}
    \right)
  + \overline{f\cdot{}u''}
  + \overline{\Ssd_{\rho{} u}\cdot{}u''}.
\label{eq:fans_tke}
\end{align}
The equations are augmented by the following nondimensional relationships:
\begin{align}
  \bar{p} &= \frac{\bar{\rho} \tilde{T}}{\gamma}
&
   \bar{\rho}\tilde{\nu} =
   \bar{\mu}
&= \overline{T^\beta}
&
  k &= \frac{1}{2}\widetilde{{u''}^2}
&
  \bar{\rho} \epsilon &= \overline{\tau : \nabla{}u''}
\end{align}
\begin{align}
  \tilde{E}
&=
  \frac{\tilde{T}}{\gamma\left(\gamma-1\right)}
  + \Mach^2 \left( \frac{1}{2}\tilde{u}^2 + k
  \right)
&
  \tilde{H}
&=
  \tilde{E} + \frac{\tilde{T}}{\gamma}
&
  \tilde{h} &= \frac{\tilde{T}}{\gamma-1}
\end{align}
\begin{align}
   \tilde{S}
&=
     \frac{1}{2}\left(
       \widetilde{\nabla{}u} + \trans{\widetilde{\nabla{}u}}
     \right)
   - \frac{1}{3}\left(\widetilde{\nabla\cdot{}u}\right) I
\end{align}
\begin{align}
   \bar{\tau}
&=  2 \bar{\mu}\tilde{S}
  + 2 \bar{\rho} \widetilde{\nu''S''}
  + \alpha \bar{\mu} \widetilde{\nabla\cdot{}u} I
  + \alpha \bar{\rho} \widetilde{\nu''\left(\nabla\cdot{}u\right)''} I.
\end{align}
\end{subequations}
Beyond references~\eqref{eq:basic_references}, this nondimensionalization
additionally selects:
\begin{align}
  k_0 &= u_{0}^2
&
  \epsilon_0 &= \frac{u_{0}^2}{t_0}.
\end{align}

Several correlations affect the evolution
of mean quantities: the Reynolds stress,
$-\bar{\rho}\widetilde{u''\otimes{}u''}$; the Reynolds heat flux, $\bar{\rho}
\widetilde{h''u''} = \bar{\rho} \widetilde{T''u''}/\left(\gamma - 1\right)$;
turbulent production, $- \bar{\rho} \widetilde{u''\otimes{}u''} :
\nabla\tilde{u}$; turbulent dissipation, $\bar{\rho} \epsilon/\Reynolds$; turbulent
transport, $-\frac{1}{2}\bar{\rho}\widetilde{{u''}^{2}u''}$; turbulent work,
$\overline{\tau{}u''}/\Reynolds$; and the two forcing-velocity correlations,
$\overline{f\cdot{}u''}$ and $\overline{\Ssd_{\rho{} u}\cdot{}u''}$.  The
Reynolds stress and heat flux augment the viscous stress and heat flux,
respectively.  The production term generates the turbulent kinetic energy
density~$k$ from the interaction of fluctuations with mean gradients while the
dissipation term destroys $k$.  The turbulent transport and work terms represent
transport of the $k$ and viscous stress work due to turbulent velocity
fluctuations, respectively.  The commonly encountered pressure--velocity
correlation, $\overline{p'u''}$, does not appear in the $k$~equation because an
exact ideal gas relationship for the turbulent mass flux discussed by
\citet[p.~216]{Lele1994Compressibility},
\begin{align}
    \label{eq:lelecompresibility}
    \overline{u''} &= \frac{\widetilde{T''u''}}{\tilde{T}}
                    - \frac{\overline{p'u''}}{\bar{p}},
\end{align}
has been used to eliminate it.

The FANS equations may be expressed equivalently using only Reynolds averaging
and therefore are often called the compressible Reynolds-averaged Navier--Stokes
(RANS) equations.  Notice that no new constitutive assumptions have been
employed to produce this FANS formulation--- caveat integrability and smoothness
requirements they are as exact a description of flow physics as the governing
Navier--Stokes equations.  Several common simplifications, none of which has
been made above, along with the correlations they implicitly neglect are
documented in Appendix~\ref{sec:derivation_FANS}.

The FANS equations are ``unclosed'' because knowing $\bar{\rho}$, $\tilde{u}$,
$\tilde{E}$, and $k$ does not permit advancing that state in time.  Advancing a
solution requires:
\begin{subequations}
\begin{align*}
&\bar{\rho}
&
&\tilde{u}
&
&\tilde{E}
&
&\bar{\mu}
&
&\bar{f}
&
&\bar{q}_b
&
&k
&
&\epsilon
&
&\overline{u''}
&
&\symmetricpart{\widetilde{\nabla{}u}}
\end{align*}
\begin{align*}
&\overline{f\cdot{}u''}
&
&\overline{\tau{}u''}
&
&\overline{p'\nabla\cdot{}u''}
&
&-\widetilde{u''\otimes{}u''}
&
&-\frac{1}{2}\widetilde{{u''}^{2}u''}
\end{align*}
\begin{align*}
&\widetilde{T''u''}
&
&\widetilde{\nu''S''}
&
&\widetilde{\nu''\left(\nabla\cdot{}u\right)''}
&
&\widetilde{\nu''\left(\nabla{}T\right)''}
\end{align*}
\begin{align*}
&\overline{\Ssd_{\rho{}}}
&
&\overline{\Ssd_{\rho{} u}}
&
&\overline{\Ssd_{\rho{} E}}
&
&\overline{\Ssd_{\rho{} u}\cdot{}u''}.
\end{align*}
\end{subequations}
In many circumstances, the mean state is known \emph{a priori} to be independent
of time and of a lower spatial dimensionality than the instantaneous state.

Experimentally obtained estimates of the reduced set of these quantities
required to ``close'' a particular problem are referred to as
``statistics'' in the turbulence community.  For example, channel flows are
characterized by statistics varying only in the wall-normal direction.
Spatially evolving boundary layers possess statistics that vary in both the
streamwise and wall-normal direction.  Homogenization, as summarized in the
following section, trades the boundary layer's streamwise statistical evolution
for a dependence on a collection of auxiliary closure assumptions and modeling
parameters.


\section[Spatiotemporal Homogenization Permitting an Inviscid Base Flow]
        {Spatiotemporal Homogenization\\Permitting an Inviscid Base Flow}
\label{sec:imposing_fpg}

\citet{Topalian2014Spatiotemporal} recently postulated a spatiotemporal
homogenization formulation for simulating the fast evolution of a homogenized
flow defect relative to some prescribed, spatially developing inviscid base
flow.
%
This section states the forcing terms in sufficient detail to reproduce
the simulation results in the present work.  The construction
of this spatiotemporal model appears in \autoref{sec:slowgrowthmodels} for
completeness.

The nondimensional, conserved spatiotemporal forcing entering
into~\eqref{eq:nondim_model} is
\begin{subequations}
\label{eq:spatiotemporal_cons_forcing}
\begin{align}
    \Ssd_{\rho}     &= \Ssd_{\rho,xt},
&   \Ssd_{\rho u_i} &= \rho \Ssd_{u_i,xt} + u_i \Ssd_{\rho,xt},
&   \Ssd_{\rho E}   &= \rho \Ssd_{E,xt}   + E   \Ssd_{\rho,xt}
\end{align}
where, fixing a temporal growth rate $\operatorname{gr}_{t_0}\!\left(
\Delta \right)$, the primitive constituents are:
\label{eq:spatiotemporal_prim_forcing}
\begin{align}
    \Ssd_{\rho,xt} &= \tilde{u}\left( \rho \right)_{x_0}
                    + \rho \left( \tilde{u} \right)_{x_0}
\\
    \Ssd_{u_i,xt}  &= \tilde{u} \left( \tilde{u}_i \right)_{x_0}
                    + \frac{\delta_{ix} \left(\bar{p}\right)_{x_0}}
                           {\Mach^2 \bar{\rho}}
                    + u_i^{\prime\prime}\left[
                        - \operatorname{gr}_{t_0}\!\left(A_u^A\right)
                        + \frac{y \operatorname{gr}_{t_0}\!\left(\Delta\right)}
                               {\sqrt{\widetilde{u_k^{\prime\prime}u_k^{\prime\prime}}}}
                          \frac{\partial\!\sqrt{\widetilde{u_k^{\prime\prime}u_k^{\prime\prime}}}}
                               {\partial\!y}
                      \right]
\\
    \Ssd_{E  ,xt}  &= \tilde{u} \left( \tilde{E}   \right)_{x_0}
                    + \frac{\bar  {p}}{\bar{\rho}} \left( \tilde{u} \right)_{x_0}
                    + \frac{\tilde{u}}{\bar{\rho}} \left( \bar  {p} \right)_{x_0}
                    + E^{\prime\prime}\left[
                        - \operatorname{gr}_{t_0}\!\left(A_E^A\right)
                        + \frac{y \operatorname{gr}_{t_0}\!\left(\Delta\right)}
                               {\sqrt{\widetilde{E^{\prime\prime}E^{\prime\prime}}}}
                          \frac{\partial\!\sqrt{\widetilde{E^{\prime\prime}E^{\prime\prime}}}}
                               {\partial\!y}
                      \right].
\end{align}
\end{subequations}
These terms are considerably more complex than their temporal
predecessors~\eqref{eq:temporalhomogenization}.
Subscripts $t_0$ and $x_0$ indicate forcing arising from temporal or spatial
homogenization, respectively.  The former terms are gathered inside brackets
in \eqref{eq:spatiotemporal_prim_forcing}.  \citeauthor{Topalian2011Slow}
modeled the latter terms as
\begin{subequations}
\label{eq:spatiotemporal_prim_model}
\begin{align}
\left(\rho\right)_{x_0} &=
    \frac{\rho}{\bar{\rho}} \left(
        - \frac{\partial\!{\rho}_I}{\partial\!x_0}
        - \bar{\rho}_D \operatorname{gr}_{x_0}\!\left({\bar{\rho}_D^A}\right)
        + y \operatorname{gr}_{x_0}\!\left(\Delta\right)
          \frac{\partial\!\bar{\rho}_D}{\partial\!y}
    \right)
\\
\left(\tilde{u}_i\right)_{x_0} &=
    - \frac{\partial\!{u}_{i,I}}{\partial\!x_0}
    - \tilde{u}_{i,D} \operatorname{gr}_{x_0}\!\left(\tilde{u}_{i,D}^A\right)
    + y \operatorname{gr}_{x_0}\!\left(\Delta\right)
      \frac{\partial\!\tilde{u}_{i,D}}{\partial\!y}
\\
\label{eq:sp_barp_model}
\left(\bar{p}\right)_{x_0} &=
    - \frac{\partial\!{p}_I}{\partial\!x_0}
    - \bar{p}_D \operatorname{gr}_{x_0}\!\left(\bar{p}_{D}^A\right)
    + y \operatorname{gr}_{x_0}\!\left(\Delta\right)
      \frac{\partial\!\bar{p}_D}{\partial\!y}
\\
\left(\tilde{E}\right)_{x_0} &=
    - \frac{\partial\!{E}_I}{\partial\!x_0}
    - \tilde{E}_D \operatorname{gr}_{x_0}\!\left(\tilde{E}_{D}^A\right)
    + y \operatorname{gr}_{x_0}\!\left(\Delta\right)
      \frac{\partial\!\tilde{E}_D}{\partial\!y}
\end{align}
\end{subequations}
which must be computed against a base flow satisfying the steady Euler
equations.  That is, in conjunction with the instantaneous Favre-averaged state,
pointwise inviscid data
\begin{equation}
\label{eq:spatiotemporal_cons_baseflow}
\begin{alignedat}{5}
    &\rho_I\!\left(y\right)                                &\qquad &\rho                                u_I\!\left(y\right)  &\qquad  &\rho                                v_I\!\left(y\right)  &\qquad  &\rho                                E_I\!\left(y\right)  &\qquad  &p_I\!\left(y\right)                               \\
    \frac{\partial\!}{\partial\!y}&\rho_I\!\left(y\right)  &\qquad \frac{\partial\!}{\partial\!y}&\rho  u_I\!\left(y\right)  &\qquad  \frac{\partial\!}{\partial\!y}&\rho  v_I\!\left(y\right)  &\qquad  \frac{\partial\!}{\partial\!y}&\rho  E_I\!\left(y\right)  &\qquad  \frac{\partial\!}{\partial\!y}&p_I\!\left(y\right) \\
    \frac{\partial\!}{\partial\!x}&\rho_I\!\left(y\right)  &\qquad \frac{\partial\!}{\partial\!x}&\rho  u_I\!\left(y\right)  &\qquad  \frac{\partial\!}{\partial\!x}&\rho  v_I\!\left(y\right)  &\qquad  \frac{\partial\!}{\partial\!x}&\rho  E_I\!\left(y\right)  &\qquad  \frac{\partial\!}{\partial\!x}&p_I\!\left(y\right)
\end{alignedat}
\end{equation}
must be specified to define the mean primitive viscous flow defects
\begin{align}
    \bar{\rho}_D &= \bar{\rho} - \rho_I
&   \tilde{u}_{i,D} &= \tilde{u}_i - u_{i,I}
&   \tilde{E}_D &= \tilde{E} - E_I
&   \bar{p}_D &= \bar{p} - p_I
\end{align}
entering into~\eqref{eq:spatiotemporal_prim_model}.  Nonzero streamwise
derivatives in the inviscid base flow data, for example $p_I$ entering
into~\eqref{eq:sp_barp_model}, are what permit the model to impose
pressure-gradient-like conditions while retaining streamwise periodicity in the
fast time solution.  A semi-analytical procedure to generate the base flow
data~\eqref{eq:spatiotemporal_cons_baseflow} necessary for the present work
is the subject of Appendix~\ref{sec:radialflow}.

The two parameters
\begin{align}
    \operatorname{gr}_{t_0}\!\left(\Delta\right) &= \left.\left(
        -\frac{\epsilon}{\Delta} \frac{\partial\!\Delta}{\partial\!t_s}
    \right)\right|_{t_s = t_0}
    &
    \operatorname{gr}_{x_0}\!\left(\Delta\right) &= \left.\left(
        -\frac{\epsilon}{\Delta} \frac{\partial\!\Delta}{\partial\!x_s}
    \right)\right|_{x_s = x_0}
\end{align}
represent the growth rate of a characteristic length scale $\Delta$ at some
fixed slow time $t_0$ or some fixed slow location $x_0$ for small homogenization
parameter $\epsilon$.  In practice,
$\operatorname{gr}_{t_0}\!\left(\Delta\right)$ is a constant supplied to target
some desired boundary layer thickness with $\epsilon$ indirectly fixed.
The inviscid base flow streamwise velocity controls the second parameter per
\begin{equation}
    \operatorname{gr}_{x_0}\!\left(\Delta\right)
    =
    \frac{\operatorname{gr}_{t_0}\!\left(\Delta\right)}
         {u_{I,w}}
\end{equation}
where the subscript $w$ denotes wall data taken from $y=0$.  The wall reference
is chosen as no freestream limit exists for flows experiencing nonzero pressure
gradients.

Expressions~\eqref{eq:spatiotemporal_prim_model} include constants governing the
growth rates for the amplitude of the mean flow defect, denoted
$\operatorname{gr}_{x_0}\!\left(q_D^A\right)$ for
$q\in\left\{\bar{\rho},\tilde{u},\tilde{v},\tilde{w},\tilde{E},\bar{p}\right\}$.
In scenarios with an isothermal wall, known boundary state in conjunction with
the inviscid base flow~\eqref{eq:spatiotemporal_cons_baseflow} informs these
quantities.  By definition,
\begin{align}
    \label{eq:gramp_mean}
    \operatorname{gr}_{x_0}\!\left(q_D^A\right)
    &=
    \left.
    \frac{1}{q_D^A}
    \frac{\partial\!q_D^A}{\partial\!x_s}
    \right|_{x_s=x_0}
    =
    \left.
    \frac{1}{q_{w} - q_{I,w}}
    \left(
          \frac{\partial\!q_{w}  }{\partial\!x_s}
        - \frac{\partial\!q_{I,w}}{\partial\!x_s}
    \right)
    \right|_{x_s=x_0}.
\end{align}
%
For convenience, $\left.\left(\partial\!q / \partial\!x_s\right)\right|_{x_s = x_0}$ is
henceforth abbreviated as $\partial\!q / \partial\!x_s$.
%
From~\eqref{eq:nondim_constitutive}, uniform wall temperature $T_w$, and the
isobaric assumption $\partial\!\bar{p} / \partial\!_y \approx 0$,
\begin{align}
    \bar{\rho}_w
    &= \frac{\gamma \bar{p}_{w}}{T_w}
    \approx \frac{\gamma p_{I,w}}{T_w}.
\end{align}
Taking the slow spatial derivative under these assumptions,
\begin{align}
    \frac{\partial\!\bar{\rho}_w }{\partial\!x_s}
    &\approx \frac{\gamma}{\bar{T}_w} \frac{\partial\!p_{I,w}}{\partial\!x_s}.
\end{align}
Therefore,
\begin{align}
    \label{eq:gramp_mean_rho}
    \operatorname{gr}_{x_0}\!\left(\bar{\rho}_{D}\right)
    &\approx
    \frac{1}{\frac{\gamma p_{I,w}}{T_w} - {\rho}_{I,w}}
    \left(
          \frac{\gamma}{T_w} \frac{\partial\!{p   }_{I,w} }{\partial\!x_s}
        -                    \frac{\partial\!{\rho}_{I,w} }{\partial\!x_s}
    \right)
    =
    \frac{%
          T_w    \frac{\partial\!\rho_{I,w}}{\partial\!x_s}
        - \gamma \frac{\partial\!   p_{I,w}}{\partial\!x_s}
    }{%
          T_w \rho_{I,w} - \gamma p_{I,w}
      }.
\end{align}
%
Consider the wall-normal momentum growth rate at a no-slip wall,
\begin{align}
    \operatorname{gr}_{x_0}\!\left(\overline{\rho v}_D^A\right)
    &=
    \frac{\frac{\partial\!}{\partial\!x_s} {\rho v}_{I,w}}{{\rho v}_{I,w}}.
\end{align}
Any nonzero wall blowing velocity $v_w$ has been neglected because
mimicking~\eqref{eq:gramp_mean_rho},
\begin{align}
    \label{eq:gramp_mean_rhov_alt}
    \underset{\text{rejected}}
             {\operatorname{gr}_{x_0}\!\left(\overline{\rho v}_D^A\right)}
    &\approx
    \frac{1}{{\rho v}_{I,w} - \frac{\gamma p_{I,w}}{T_w} v_w}
    \left(
          \frac{\partial\!}{\partial\!x_s}         {\rho v}_{I,w}
        - \frac{\gamma v_w}{T_w} \frac{\partial\!}{\partial\!x_s} {p}_{I,w}
    \right)
\\\notag
    &=
    \frac{%
          T_w \frac{\partial\!}{\partial\!x_s} {\rho v}_{I,w}
        - \gamma v_w \frac{\partial\!}{\partial\!x_s} {p     }_{I,w}
    }{%
        T_w {\rho v}_{I,w} - \gamma v_w p_{I,w}
    }
    ,
\end{align}
behaves oddly on two accounts.  First, from it one
recovers~\eqref{eq:gramp_mean_rho} whenever the base flow is designed with
transpiration as then both $v_{I,w} = v_w \neq 0$ and $\frac{\partial\!v_{I,w}}{\partial\!
x_s} = 0$ hold.  Second, whenever $v_{I,w} = 0$ its limiting $v_w \to 0$
behavior is broken in the sense that one recovers
$\left(\frac{\partial\!{p}_{I,w}}{\partial\!x_s}\right) / p_{I,w}$ for any $v_w
\neq 0$ but not when $v_w = 0$.
%
Consequently, the velocity growth rates also ignore blowing and are:
\begin{align}
    \operatorname{gr}_{x_0}\!\left(\tilde{u}_D^A\right)
    &=
    \frac{%
        1
    }{%
        {u}_{I,w}
    }
    \frac{\partial\!{u}_{I,w}}{\partial\!x_s}
    &
    \operatorname{gr}_{x_0}\!\left(\tilde{v}_D^A\right)
    &=
    \frac{%
        1
    }{%
        {v}_{I,w}
    }
    \frac{\partial\!{v}_{I,w} }{\partial\!x_s}
    &
    \operatorname{gr}_{x_0}\!\left(\tilde{w}_D^A\right)
    &=
    \frac{%
        1
    }{%
        {w}_{I,w}
    }
    \frac{\partial\!{w}_{I,w}}{\partial\!x_s}.
\end{align}
%
The specific energy mean defect growth rate is
\begin{align}
    \operatorname{gr}_{x_0}\!\left(\tilde{E}_D^A\right)
    &=
    \frac{%
        \frac{\partial\!{E}_{I,w}}{\partial\!x_s}
    }{%
        {E}_{I,w} - E_w
    }
\end{align}
where wall blowing is now neither problematic nor neglected
so~\eqref{eq:nondim_constitutive} fixes
\begin{align}
    E_w &= \frac{T_w}{\gamma \left(\gamma - 1\right)}
         + \frac{\Mach^2}{2} v_w^2.
\end{align}
%
Finally, whenever growth rates are uninformed or ill-defined according to these
arguments, they are taken to be zero.  Therefore,
\begin{align}
    \operatorname{gr}_{x_0}\!\left(\bar{p}_D^A\right) &= 0,
&
    \operatorname{gr}_{t_0}\!\left(A_u^A\right) &= 0,
&
    \operatorname{gr}_{t_0}\!\left(A_E^A\right) &= 0.
\end{align}
Other cases necessitating this final clause include the thermodynamic growth
rates when $\left| 1 - T_w / T_{I,w} \right| < 1\%$ and the wall-normal and
spanwise
% momentum and %%% NOT USED HERE
velocity rates when the base flow at the wall is trivial
in those directions.
