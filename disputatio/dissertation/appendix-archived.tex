\label{sec:archiving}
\todo{fix me}

Two general types of data have been captured from each completed
simulation and archived on the Corral\footnote{%
    \texttt{lonestar:/corral-tacc/utexas/pecos/turbulence/thesis-rhys-chapter\{5,6,7\}/}
}
and Ranch\footnote{%
    \texttt{rhys@ranch:\{cevisslam,channels\}/}
}
resources at the Texas Advanced Computing Center\footnote{%
    \url{http://www.tacc.utexas.edu/}
}
(TACC).  These complete archives will be made available on request.  Reduced
data, as described towards the end of this appendix, will be made available
online\footnote{\url{http://turbulence.ices.utexas.edu/}} for general
consumption.

The first type of data archived describes the software environment used for the
simulations.  From each production batch job the information shown in
\autoref{tbl:executiondetails} was preserved.
%
In addition to providing an execution record and raw data for performance
variability investigations, these details permit determining what, if any,
portions of the software stack may have changed between any two given batch
jobs.

\begin{table}
\centering
\caption[Execution details captured from each production batch job]{%
  Software and hardware execution details captured from every production batch
  job as human-readable text files.  Files named like \texttt{*.dat} provide time
  measured relative to a wall clock, the simulation physics, and time step
  number.\label{tbl:executiondetails}
}
\begin{small}
\begin{tabular}{p{0.20\textwidth}|p{0.70\textwidth}}
Filename & Contents \\ \hline \hline
\texttt{bc.dat}       & Trace of conserved state behavior at boundaries \\
\texttt{binary}       & Absolute path to the compiled Suzerain binary \\
\texttt{cpuinfo}      & \texttt{/proc/cpuinfo} from MPI rank zero \\
\texttt{dependencies} & Runtime-resolved shared library dependencies \\
\texttt{environment}  & Environment variables in effect at runtime \\
\texttt{kernel}       & \texttt{/proc/kernel} from MPI rank zero \\
\texttt{log.dat}      & Complete execution log according to Suzerain \\
\texttt{meminfo}      & \texttt{/proc/meminfo} from MPI rank zero \\
\texttt{output}       & Complete execution log according to the batch system \\
\texttt{qoi.dat}      & Trace of scalar quantities of interest like $\Reynolds[\theta]{}$ \\
\texttt{state.dat}    & Trace of mean and fluctuating conserved state \\
\texttt{version}      & Suzerain version information from the compiled binary
\end{tabular}
\end{small}
\end{table}

The second type of data archived contains instantaneous physics.  This data
consists of complete instantaneous field snapshots taken periodically during each
simulation run along with a variety of scenario parameters and descriptive grid
statistics.  This data is stored in HDF5~\citep{hdf5} files via the ESIO
library~\citep{ESIOweb}.  These snapshots are Suzerain restart
files.  \autoref{tbl:restartfile} describes a subset of the data captured.  An
effort was made to preserve the discrete operator details so that others might
post-process the fields using consistent numerics but without needing access to
a B-spline package.  While fields are stored as Fourier and B-spline
expansion coefficients for efficiency and operational flexibility, Suzerain can convert this data to physical space if
necessary.

\begin{table}
\centering
\caption[Instantaneous fields and other details comprising a restart file]{%
  A small subset of the details comprising a Suzerain restart file.
  HDF5 comments in the file provide operational context.  For example,
  information on field storage ordering is provided in the comments of
  \texttt{/rho}, \texttt{/rho\_E}, etc.\label{tbl:restartfile}
}
\begin{small}
\begin{tabular}{p{0.29\textwidth}|p{0.65\textwidth}}
HDF5 Dataset & Contents \\ \hline \hline
\texttt{alpha                 } & Ratio of bulk to dynamic viscosity \\
\texttt{beta                  } & Temperature power law exponent \\
\texttt{breakpoints\_y        } & Breakpoint locations used for wall-normal B-spline basis \\
%\texttt{bulk\_rho             } & Bulk density target \\
%\texttt{bulk\_rhou            } & Bulk streamwise momentum target \\
\texttt{collocation\_points\_x} & Collocation points for the dealiased, streamwise X direction \\
\texttt{collocation\_points\_y} & Collocation points for wall-normal discrete operators \\
\texttt{collocation\_points\_z} & Collocation points for the dealiased, spanwise Z direction \\
\texttt{DAFx                  } & Dealiasing factor in streamwise X direction \\
\texttt{DAFz                  } & Dealiasing factor in spanwise Z direction \\
\texttt{Dy0T                  } & Transpose of banded, wall-normal Y collocation mass matrix \\
\texttt{Dy1T                  } & Transpose of banded, wall-normal Y first derivative \\
\texttt{Dy2T                  } & Transpose of banded, wall-normal Y second derivative \\
\texttt{evmagfactor           } & Safety factor in $\left(0,1\right]$ used to adjust time step aggressiveness \\
\texttt{gamma                 } & Ratio of specific heats \\
%\texttt{Gy0T                  } & Transpose of banded, wall-normal Y Galerkin mass matrix \\
\texttt{htdelta               } & Wall-normal breakpoint hyperbolic tangent stretching \\
%\texttt{integration\_weights  } & Integrate by dotting B-spline coefficients against weights \\
\texttt{knots                 } & Knots used to build B-spline basis \\
\texttt{k                     } & Wall-normal B-spline order (4 indicates piecewise cubic) \\
\texttt{kx                    } & Wavenumbers in streamwise X direction \\
\texttt{kz                    } & Wavenumbers in spanwise Z direction \\
\texttt{Lx                    } & Nondimensional grid length in streamwise X direction \\
\texttt{Ly                    } & Nondimensional grid length in wall normal Y direction \\
\texttt{Lz                    } & Nondimensional grid length in spanwise Z direction \\
\texttt{Ma                    } & Mach number \\
\texttt{Nx                    } & Global logical extents in streamwise X direction \\
\texttt{Ny                    } & Global logical extents in wall-normal Y direction \\
\texttt{Nz                    } & Global logical extents in spanwise Z direction \\
\texttt{Pr                    } & Prandtl number \\
\texttt{Re                    } & Reynolds number \\
\texttt{rho                   } & Nondimensional density \\
\texttt{rho\_E                } & Nondimensional total energy \\
\texttt{rho\_u                } & Nondimensional streamwise momentum \\
\texttt{rho\_v                } & Nondimensional wall-normal momentum \\
\texttt{rho\_w                } & Nondimensional spanwise momentum \\
\texttt{t                     } & Simulation physical time \\
\end{tabular}
\end{small}
\end{table}

During simulation execution, \emph{in situ} instantaneous mean samples
of various quantities as a function of wall-normal position are taken
more frequently than full restart checkpoints.  The samples are stored
in separate HDF5 files sharing much with Suzerain's restart files.
The quantities thus sampled are a superset of information necessary to
compute the instantaneous Favre-averaged Navier--Stokes residuals per
\autoref{sec:statevo}.

After a simulation completes, all such samples and associated instantaneous
residuals are aggregated into a single ``reduced data'' HDF5 file.  This reduced
data permits third parties to easily access many first and second order
turbulence statistics without requiring them to post-process many gigabytes of
raw field data on a dedicated cluster environment.  In addition, the
autoregressive uncertainty estimates described in \autoref{eq:uqaccounting} are
included for each Reynolds-averaged scalar.  This reduced data is easily
downloadable and may be imported into common software like GNU~Octave,
\textsc{Matlab}\textsuperscript{\textregistered}, \textit{Mathematica}\textsuperscript{\textregistered}, or
Python in a single command.
