\label{sec:archiving}

%Do you want vane forcing functions defined here?

Two general types of data have been captured from each completed
simulation and archived on 
Ranch\footnote{npm7@ranch:/home2/00000/npm7/sov\_huge\_backup}
resources at the Texas Advanced Computing Center\footnote{%
    \url{http://www.tacc.utexas.edu/}
}
(TACC).  These complete archives will be made available on request. 

The first type of data archived describes the software environment used for the
simulations.  From each production batch job the information shown in
\autoref{tbl:executiondetails} was preserved.
%
In addition to providing an execution record and raw data for performance
variability investigations, these details permit determining what, if any,
portions of the software stack may have changed between any two given batch
jobs.


\begin{table}
\centering
\caption[Execution details captured from each production batch job]{%
  Software and hardware execution details captured from every production batch
  job as human-readable text files.  Files named like \texttt{*.dat} provide time
  measured relative to a wall clock, the simulation physics, and time step
  number.\label{tbl:executiondetails}
}
\begin{small}
\begin{tabular}{p{0.20\textwidth}|p{0.70\textwidth}}
Filename & Contents \\ \hline \hline
\texttt{bc.dat}       & Trace of conserved state behavior at boundaries \\
\texttt{binary}       & Absolute path to the compiled Suzerain binary \\
\texttt{cpuinfo}      & \texttt{/proc/cpuinfo} from MPI rank zero \\
\texttt{dependencies} & Runtime-resolved shared library dependencies \\
\texttt{environment}  & Environment variables in effect at runtime \\
\texttt{kernel}       & \texttt{/proc/kernel} from MPI rank zero \\
\texttt{log.dat}      & Complete execution log according to Suzerain \\
\texttt{meminfo}      & \texttt{/proc/meminfo} from MPI rank zero \\
\texttt{output}       & Complete execution log according to the batch system \\
\texttt{qoi.dat}      & Trace of scalar quantities of interest like $\Reynolds[\theta]{}$ \\
\texttt{state.dat}    & Trace of mean and fluctuating conserved state \\
\texttt{version}      & Suzerain version information from the compiled binary
\end{tabular}
\end{small}
\end{table}

The second type of data archived contains instantaneous physics.  This data
consists of complete instantaneous field snapshots taken periodically during each
simulation run along with a variety of scenario parameters and descriptive grid
statistics.  

\begin{table}
\centering
\caption[Instantaneous fields and other details comprising a restart file]{%
  A small subset of the details comprising a Suzerain restart file.
  HDF5 comments in the file provide operational context.  For example,
  information on field storage ordering is provided in the comments of
  \texttt{/rho}, \texttt{/rho\_E}, etc.\label{tbl:restartfile}
}
\begin{small}
\begin{tabular}{p{0.29\textwidth}|p{0.65\textwidth}}
HDF5 Dataset & Contents \\ \hline \hline
\texttt{alpha                 } & Ratio of bulk to dynamic viscosity \\
\texttt{t                     } & Simulation physical time 
\end{tabular}
\end{small}
\end{table}

This data is cumbersome and not easily imported into common software like GNU~Octave,
\textsc{Matlab}\textsuperscript{\textregistered}, \textit{Mathematica}\textsuperscript{\textregistered}, or
Python in a single command. Rather, paraview provides the best means to visualize and explore these 
datasets. 
