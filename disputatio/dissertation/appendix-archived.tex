\label{sec:archiving}

%Do you want vane forcing functions defined here?

The entirety of data used in this document have been captured and
are on the tape archival system
Ranch\footnote{\normalsize npm7@ranch:/home2/00000/npm7/sov\_huge\_backup} 
at the Texas Advanced Computing Center\footnote{% 
    \normalsize \url{http://www.tacc.utexas.edu/}
}
(TACC).  These complete archives will be made available on request. 

The files are stored in a format identical to that of the SVN archive
located at,
\url{https://svn.ices.utexas.edu/repos/pecos/solar_vortex/}. The
organization of the repository bears some discussion. The root level,
contains the folders: {\it documents}, {\it grids},{\it input}, {\it
postproc}, and {\it single\_shot\_input}. {\it documents} contains
quarterly reports and model documentation in LaTeX and MSFT word format.  
{\it grids} contains the raw gridgen files used to generate meshes for
the gridded vanes. {\it postproc} contains the files used to perform
temporal averaging, as well as paraview and python scripts used to
visualize the fields and generate images of the simulations. {\it
single\_shot\_input} is a deprecated set of input files from the
earliest investigations into the SoV. These input files represent an
older format where all the definitions and file settings were contained
in a single input file. Due to the volume and complexity of input
required for these simulations, these older files are cumbersome and
 difficult to read. Finally, {\it input} is the directory that
contains the input files and the output of the simulations (on
Ranch, not on SVN). 

The {\it input} directory is broken into four directories. The first,
{\it field}, contains all the physical investigations for the SoV Field
tests with the virtual vanes, typically steady, but some unsteady
virtual vane investigations are also contained here. {\it gridded} are
directories that contain the input files and simulation output from the
gridded runs, and {\it laboratory} contains the table-top laboratory
runs. Finally, {\it opt} contains the optimization runs where runs where
rapidly iterated with perturbed system parameters. These were entirely
``Steady'' virtual vane cases. 

All these directories then have a common structure. They have a {\it
common} directory that contains all the sub-input files, and then a
unique problem folder that details the unique run and the output
of this file. For instance, the a problem folder might be entitled,
``field\_2016\_august\_3m'' for the 2016 August field test conducted
with a three meter per second wind velocity. Inside each problem
directory, there are two files, an ``initial.in'' and a ``gold.in''. The
initial file starts a run, even if steady, typically with enhanced (and
likely un-physical) viscosity, to help the solver converge. Subsequent
runs are restarted from this base state but with the viscosity model
detailed in Chapter~\ref{sec:mathmodel}. No results from this initial
solve are quoted in this document. In some cases for complicated
geometries, multiple initial steps were required, in which the viscosity
was stepped down from the high initial state to the model derived
values. 

In addition to the input files, each of these directories contains two
directories, {\it gold} and {\it output}. After each run, all of the
output files are moved into {\it output} where they are saved in a
directory labeled by the unique slurm id for that particular run. In
addition, if the run exited successfully, then the output files
necessary to restart the run are saved in the {\it gold} directory. 
This is handled automatically, after the completion of a job, by custom
bash scripts attached to the slurm scheduler. These scripts are
available in the {\it common} directory. 

Each input file must specify a path to files contained in {\it common}
to define the problem run. Thus, for instance, to specify a viscosity
model, the line, 
\begin{verbatim}
[include ../common/viscosity/visc_mo_steady_super.in]
\end{verbatim}
would run the job with the viscosity model defined in the file,
``visc\_mo\_steady\_super.in''.  
The {\it common} contained every file needed to specify a problem, so
that different cases share common files. This ensures that a single file
cannot have an incorrectly specified parameter in it, and permits
regression testing old results to ensure that output has not changed. 
The {\it common} directory is organized as follows. The available models
to include are specified in Table~\ref{tbl:run}. 

\begin{table}
\centering
\caption{%
  The available model classes for each SoV run. Not all model classes
 are necessary to perform a run.\label{tbl:run}
}
\begin{small}
\begin{tabular}{p{0.20\textwidth}|p{0.70\textwidth}}
Directory Name & Purpose \\ \hline \hline
 \texttt{bc}       & Boundary Conditions \\
 \texttt{drag}       & Drag Model (optional) \\
 \texttt{forcing}       & Surface Roughness Forcing \\
 \texttt{ic}       & Initial Conditions \\
 \texttt{qoi}       & Quantities of Interest to be measured, such as
     kinetic energy flux \\
 \texttt{scripts}       & Scripts used to invoke job, not used by input
     files directly \\
 \texttt{turbine}       & Turbine run definition \\
 \texttt{vanes}       & Virtual vane definition \\
 \texttt{visc}       & Viscosity Model \\
\end{tabular}
\end{small}
\end{table}


\begin{table}
\centering
\caption[Instantaneous fields and other details comprising a restart file]{%
  The data comprising a restart file, stored in EXODUSII
 format.\label{tbl:EXO}
}
\begin{small}
\begin{tabular}{p{0.29\textwidth}|p{0.65\textwidth}}
EXODUSII Dataset & Contents \\ \hline \hline
\texttt{T                     } & The 3d temperature field, in Kelvin \\
\texttt{u                     } & Streamwise velocity component \\ 
\texttt{v                     } & Spanwise velocity component \\ 
\texttt{w                     } & Vertical velocity component \\ 
\texttt{p                     } & Pressure field \\ 
\texttt{k                     } & Thermal conductivity field \\ 
\texttt{mu                    } & Kinematic Viscosity field \\ 
\texttt{vel\_penalty\_        } & Virtual Vane Forcing Field \\ 
\texttt{vel\_source\_        }  & Surface Roughness Forcing Field 
\end{tabular}
\end{small}
\end{table}

A snapshot of an input file is provided in Table~\ref{tbl:EXO}. 
Due to the exodusII format, this data is cumbersome and not easily
imported into common software like GNU~Octave,
\textsc{Matlab}\textsuperscript{\textregistered},
\textit{Mathematica}\textsuperscript{\textregistered}, or Python in a
single command. Rather, paraview provides the best means to visualize 
and explore these datasets, and was the main post-processing software
used in this thesis. A hand written python reader was developed and can
be provided upon request. 

The github hashes of the latest (and believed to be backward compatible)
GRINS and libMesh versions used in this document were, 
\begin{verbatim}
  GRINS Version: 5373d0fc001ea98c715638851e4e3b0e7f96cc95                                                  
  libMesh Version: cd139a10cef2cf603f85f64a11c10d6bbe4d6780
\end{verbatim}
%
While built from master in the development branch, these should
correspond closely to versions v0.7.0 in GRINS and libMesh v1.0.0.  

%Finally, the raw .tex of this document and the accompanying figures was
%is publicly available at,
%\url{https://github.com/nicholasmalaya/arcanus/disputatio/dissertation}.  