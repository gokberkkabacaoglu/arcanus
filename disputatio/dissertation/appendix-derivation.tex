%\section{Stabilization and Weak Formulation}
\label{app:stab}

This appendix details the weak formulation of the Navier Stokes
equations instantiated in the software GRINS, and provides a
derivation of the $\tau$ stabilization terms. 

In brief, our process is the following: 

\begin{itemize}
 \item Cast Navier Stokes + Boussinesq equations into weak form
 \item Prepare as an operator $Lc=f$
 \item Calculate Fr\'echet derivative
 \item Separate into differential (P) and constant (Z) components,
       $L'[c] = P + Z$
 \item Choose stabilization operator such that $S = -P^*$
 \item Then stabilization has form, $a_h(c,\phi) = a(c,\phi) + \langle
       Lc,S\phi \rangle_\tau$
\end{itemize}

This is essentially the least-squares stabilization proposed by Hughes
and extended to natural convection by Becker and Braack. 

%
% start the real work
%

\section{Weak Formulation of the Equations of Interest}

We begin with the incompressible Navier-Stokes equations with Bousinesq
buoyancy,
\begin{align}
 \nabla \cdot u &= \, 0 \label{eq_cont}\\
 \frac{\partial u}{\partial t} + u \cdot \nabla u &= -\frac{1}{\rho}
 \nabla p + \nu \nabla^2 u + g \frac{T'}{T_0} \label{eq_mom}\\
 \rho c_p \frac{\partial T}{\partial t} + u \cdot \nabla T &= \nabla
 \cdot (k \nabla T) \label{eq_energy}
\end{align}
e.g. the continuity, momentum and energy equations, respectively. Our 
state vector is $c =  \left[p,u,T \right]$. To cast these into
weak form we multiply by appropriate test 
functions $\phi = \left[q,v,w \right] \in H^1_0(\Omega)$ and integrate over
the domain, $\Omega \in \mathbb{R}^n$. Our system of equations now
appears as, 
\begin{align}
  \bigintsss_\Omega q \nabla \cdot u \, dx &= 0 \\
 \bigintsss_\Omega \dot u \cdot v \, dx +
 \bigintsss_\Omega  (u \cdot \nabla) \, u \cdot v \, dx &=
 \bigintsss_\Omega \frac{p}{\rho} \nabla \cdot v \, dx - \nu \bigintsss_\Omega \nabla u \cdot \nabla v
 \,dx + \bigintsss_\Omega g \frac{T'}{T_0} \cdot v \, dx \\ 
 \rho c_p \bigintsss_\Omega \dot T \cdot w \, dx + \bigintsss_\Omega (u
 \cdot \nabla) T \cdot w \, dx  &= -\bigintsss_\Omega (k \nabla T) \cdot
 \nabla w \, dx
\end{align}

where an ``over-dot'' denotes time diffentiation, e.g. $\dot u =
\frac{\partial u}{\partial t}$. Note that both the pressure term as well
as the viscous term were integrated by parts to reduce the required
order of the solution on those state variables.  

We define the inner product by the shorthand notation $(u,v) =
\bigintsss_\Omega u\cdot v dx $, giving our equations the form,  
\begin{align}
 (\nabla \cdot u, q) &= 0 \\
 (\dot u,v) + (u \cdot \nabla u, v) - (p,\nabla \cdot v) + \nu (\nabla
 u, \nabla v) &= (g \frac{T'}{T_0},v) \\
 \rho c_p (\dot T,w) + (u \cdot \nabla T,w) + (k \nabla T,\nabla w) &= 0.
\end{align}

This defines our weak form operator $a(c,\phi)$. Our full equations will
also include a stabilization term such that,  
\begin{equation}
 a_h(c,\phi) = a(c,\phi) +  \langle Lc,S\phi \rangle_\tau. 
\end{equation}

The subsequent section will define the operators L and S, so that we
might then fully define the stabilization term $\langle Lc,S\phi
\rangle_\tau$. 

%
% section
%
\section{The Stabilization Operators L and S}


To form the stabilization terms, 

\begin{equation}
 \langle Lc,S\phi \rangle_\tau
\end{equation}

we must define the operators L and S. The operator L is simply the PDEs
in Equations \ref{eq_cont} - \ref{eq_energy} written in operator form. S is
defined as the negative adjoint of the differential terms in L, e.g.
\begin{align}
 L'[c] = P + Z \\
 S = -P^*. 
\end{align}
Where P are the differential terms, and Z the constant terms. 

Our objective is now to construct the adjoint operator of L. This is
accomplished using the Fr\'echet derivative, which defines the
functional derivative on L. In general this is accomplished by taking
the first variation of a function $\Pi(u)$ around a base state, $u$,
\begin{equation}
 \delta\, \Pi(u) = \lim_{\epsilon \to 0} \frac{\Pi(u+\epsilon \hat u) -
  \Pi(u)}{\epsilon} =
  \frac{\partial \Pi(u +\epsilon \hat u)}{\partial \epsilon}
  \bigg|_{\epsilon = 0}
\end{equation}
for all $\hat u$ and $\epsilon > 0$ with $u + \epsilon \hat u \in
H^1_0(\Omega)$. This is recognizable as the G\^{a}teaux
derivative of the functional. 


We now consider the first variation of state for the momentum equation
term by term. The convective term is, 
\begin{align}
 \frac{\partial}{\partial \epsilon} (u + \epsilon \hat u) &\cdot \nabla
  (u + \epsilon \hat u) \\
 = \lim_{\epsilon \to 0} \hat u &\cdot \nabla (u + \epsilon \hat u) \\
 = \hat u & \cdot \nabla u \\
 = - u &\cdot \nabla \hat u
\end{align}
and the viscous term is, 
\begin{align}
 \frac{\partial}{\partial \epsilon} \nabla^2 (u + \epsilon \hat u) \\
 = \nabla^2 \hat u
\end{align}
while the buoyancy term is, 
\begin{align}
 \delta \left(-g \frac{T'}{T_0}\right) &= \delta \left( -g
 \frac{T-T_0}{T_0} \right) \\
 &= -g \frac{\partial}{\partial \epsilon} \left( \frac{T-T_0+\epsilon
 \hat T}{T_0} \right) \\
 &= -g \left( \frac{\hat T}{T_0} \right) 
\end{align}


%Lagrangian is therefore, $\mathcal{L}$
and thus the full adjoint equation for momentum appears as
\begin{align}
 - u &\cdot \nabla \hat u - \nabla^2 \hat u = -\frac{1}{\rho} \nabla p. 
\label{eq_adjmom}
\end{align}

The continuity equation is straightforward, 
\begin{align}
 \frac{\partial}{\partial \epsilon} \nabla \cdot (u + \epsilon \hat
 u) = 0 \\
\nabla \cdot \hat u = 0.
\label{eq_adjcont}
\end{align}

Finally, consider the convective term of the energy equation, 
\begin{align}
 \frac{\partial}{\partial \epsilon} u \cdot \nabla (T + \epsilon \hat T)
 = u \cdot \nabla \hat T
\end{align}
and the thermal diffusion term, 
\begin{align}
 \frac{\partial}{\partial \epsilon} \cdot (-k \nabla (T + \epsilon \hat
 T)) = \nabla \cdot (-k \nabla \hat T).
\end{align}
The full adjoint energy equation is therefore, 
\begin{align}
 u \cdot \nabla \hat T + \nabla \cdot (k \nabla \hat T) =0.
\label{eq_adjen}
\end{align}

We are now in a position to define the matricies L and S. L comes
directly from the PDEs in Equations \ref{eq_cont} - \ref{eq_energy} and is
defined as thus, 

\begin{equation}
\renewcommand\arraystretch{2}
 L = 
  \begin{pmatrix}
    0 & \nabla \cdot () & 0   \\
    \nabla \,() & u \cdot \nabla() - \nu \nabla^2() & -g \frac{()}{T_0}  \\
    0 & 0 & u \cdot \nabla() - \nabla \cdot (k \nabla() \,)
  \end{pmatrix}.
\end{equation}

While the S matrix is constructed from Equations
\ref{eq_adjmom}, \ref{eq_adjcont}, and \ref{eq_adjen} must be, 
\begin{equation}
\renewcommand\arraystretch{2}
 S = -P^* = 
  \begin{pmatrix}
    0 & \nabla \cdot () & 0   \\
    \nabla \,() & u \cdot \nabla() + \nu \nabla^2() &  -g \frac{()}{T_0}  \\
    0 & 0 & u \cdot \nabla() + \nabla \cdot k \nabla()
  \end{pmatrix}.
\end{equation}

%
% tau!
%
\section{Tau Stabilization Terms}

Finally, we may now form the $\tau$ stabilization terms, 
\begin{equation}
 \langle Lc,S\phi \rangle_\tau. 
\end{equation}
Where the operator $ \langle \cdot,\cdot \rangle_\tau. $ is shorthand 
and denotes
\begin{equation}
 \langle u,v \rangle_\tau = \sum_K \tau_K (u,v)_K.
\end{equation}
Where $K$ denotes the FEM cells. Now, through what Becker and Braack
contemptibly referred to as ``elementary calculus'', we now discover our
stabilization terms,  
\begin{align*}
 \langle Lc,S\phi \rangle_\tau = \sum_K \{ \quad &\tau_p (\nabla \cdot u,
 \nabla \cdot v) \\
 +\quad &\tau_u \,(\nabla p + u \cdot \nabla u - \nu \nabla^2 - g \frac{T'}{T_0},
 \nabla q) \\
 +\quad &\tau_u \, (\nabla p + u \cdot \nabla u - \nu \nabla^2 - g \frac{T'}{T_0},
 \nabla u \cdot \nabla v + \nu \nabla^2 v) \\
 +\quad &\tau_T \, (u \cdot T - \nabla \cdot (k \nabla T), \nabla u \cdot \nabla
 w + \nabla \cdot (k \nabla w)) \}.
\end{align*}

