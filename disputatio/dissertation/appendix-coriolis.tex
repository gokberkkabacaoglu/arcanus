\label{appendix:coriolis}

This appendix briefly details the arguments of Monin and
Obukhov\cite{monin1954basic} estimating the impact of the Coriolis
force, and justifying the neglect of this force in our equations of
motion (Equation~\ref{eq:ns}, in particular). 

To begin, they assumed statistically stationary, spatially homogeneous,
neutrally stratified flow that varied only in height, z. The Reynolds
equations for the wind-velocity direction can be simplified to be,  
\begin{equation}
 \frac{\partial \overline{\rho u'w'}}{\partial z} = -\frac{\partial
  \bar P}{\partial x} + \rho l \bar v, 
\end{equation}
where $\overline{\rho u'w'}$ is the turbulent momentum flux, $\frac{\partial
\bar P}{\partial x} $  the mean pressure gradient, and $\bar v$ the
averaged wind velocity. Dividing by density and integrating over height,
z, 
\begin{equation}
 \int_0^H \frac{\partial \overline{u'w'}}{\partial z} dz = \int_0^H
  \left( -\frac{1}{\rho} \frac{\partial \bar P}{\partial x} + l \bar v
  \right) dz.  
\end{equation}
Assuming constant density and replacing with the turbulent shear stress,
$\tau = \overline{\rho u'w'}$, the left integrand can be solved,  
\begin{equation}
\frac{\tau(0) - \tau(H)}{\rho} = \int_0^H
  \left( \frac{1}{\rho} \frac{\partial \bar P}{\partial x} - l \bar v
  \right) dz.  
\end{equation}
Careful readers should note the sign change in the equation above. 