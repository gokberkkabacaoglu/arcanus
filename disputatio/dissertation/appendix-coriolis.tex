\label{appendix:coriolis}

This appendix briefly details the arguments of Monin and
Obukhov\cite{monin1954basic} estimating the impact of the Coriolis
force on the dynamics of flow near the surface, in the so-called Eckman
layer. This section provides justification for the neglect of this force
in our equations of motion (Equation~\ref{eq:ns}, in particular). 

To begin, assume statistically stationary, spatially homogeneous,
neutrally stratified flow that varied only in height, z. The Reynolds
equations for the wind-velocity direction can be simplified to be,  
\begin{equation}
 \frac{\partial \overline{\rho u'w'}}{\partial z} = -\frac{\partial
  \bar P}{\partial x} + \rho f \bar v, 
\end{equation}
where $\overline{\rho u'w'}$ is the turbulent momentum flux, $\frac{\partial
\bar P}{\partial x} $  the mean pressure gradient, $f$ is the Coriolis
frequency (often called the Coriolis parameter) and $\bar v$ the
averaged wind velocity. Dividing by density and integrating over height,
z,  
\begin{equation}
 \int_0^H \frac{\partial \overline{u'w'}}{\partial z} dz = \int_0^H
  \left( -\frac{1}{\rho} \frac{\partial \bar P}{\partial x} + f \bar v
  \right) dz.  
\end{equation}
Assuming constant density and replacing with the turbulent shear stress,
$\tau = \overline{\rho u'w'}$, the left integrand can be solved,  
\begin{equation}
\frac{\tau(0) - \tau(H)}{\rho} = \int_0^H
  \left( \frac{1}{\rho} \frac{\partial \bar P}{\partial x} - f \bar v
  \right) dz.  
\end{equation}
Careful readers should note the sign change in the equation above. The
right hand side is bounded by, 
\begin{equation}
\int_0^H  \left( \frac{1}{\rho} \frac{\partial \bar P}{\partial x} - f \bar v
  \right) dz < \int_0^H  \left( \frac{1}{\rho} \frac{\partial \bar
  P}{\partial x}  \right) dz,
\end{equation}
as the Coriolis effect opposes the action of the pressure
gradient. Substituting for the friction velocity, $u_* =
\sqrt{\frac{\tau}{\rho}}$, our inequality has the form,
\begin{equation}
u^2_*(H) - u^2_*(0)  < \int_0^H  \left( \frac{1}{\rho} \frac{\partial
				  \bar P}{\partial x}  \right) dz. 
\label{eq:cor_f}
\end{equation}
Consider a pressure wind velocity scale, 
\begin{align}
 v_p = \frac{1}{\rho f} \frac{\partial \bar P}{\partial x}  \\
 f v_p = \frac{1}{\rho} \frac{\partial \bar P}{\partial x} 
\end{align}
which, when used in Equation~\ref{eq:cor_f}, greatly simplifies our
inequality to, 
\begin{equation}
u^2_*(H) - u^2_*(0)  < f v_p \, H. 
\label{eq:cor_main}
\end{equation}
As we only seek to estimate the region where the change attributable to
the Coriolis effect is less than some tolerance, we bound
the difference as,
\begin{align}
 \frac{u^2_*(H) - u^2_*(0)}{u^2_*(0)} \leq a, \\
 u^2_*(H) - u^2_*(0)\leq a u^2_*(0),
\label{eq:cor_ineq}
\end{align}
where the tolerance $a$ is selected to be 20\%. Combining
Equations~\ref{eq:cor_ineq} and \ref{eq:cor_main}, the height H at which
the Coriolis force meets our tolerance is found to be,
\begin{align}
 a \, u^2_*(0) \leq H f v_p \\
 \boxed{\frac{a u^2_*(0)}{f v_p} \leq H}.
\end{align}
Monin and Obukhov further estimated the values of the inputs to this
equation as $\frac{u_*}{v_p} = 0.05$, $v_p \approx 10$ meters/second, $f
\approx 10^{-4}$ 1/seconds and $a=0.20$, which results in an H of 50
meters. Thus, the dynamics of flow below this height are estimated to
have a less than 20\% impact on account of the Coriolis effect, which we
further neglect fully in the simulations presented in this document. 

Incidentally, this argument may also been seen as evidence for why dust
devils have no preferred direction of rotation. The Coriolis effect is
too small to impose a direction, and only (very mildly) intensifies
cyclonic dust devils while modestly weakening the anti-cyclonic variety.