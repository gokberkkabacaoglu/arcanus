\label{sec:mms}

A time-varying manufactured solution is presented for the nondimensional
governing equations summarized in \autoref{sec:goveqn}.  That is, forcing terms
$Q_{\rho}$, $\vec{Q}_{\rho{}u}$, and $Q_{\rho{}E}$ are added to the mass,
momentum, total energy equations so that their solution matches a prescribed
form.  This particular manufactured solution is the nondimensional analogue of
that presented in \citet{Ulerich2012MMS}.  The solution was used to verify the
software described in \autoref{sec:software}.  Additional background on
manufactured solutions can be found in \citet{MASA}.

%\section{Manufactured Solution}
%\label{sec:solution}

For $\phi\in\left\{\rho, u, v, w, T\right\}$ analytical solutions are selected
with the form
\begin{alignat}{20}%\label{eq:solution}
  \phi\!\left(x, y, z, t\right)
  &= &&a_{\phi{}0}  &&          &&             &&               &&  &&            &&       &&          &&             &&                &&  &&            &&        &&\cos\Bigl(&&f_{\phi{}0 } &&t &&+ &&g_{\phi{}0 }&&\Bigr) \notag\\
  &+ &&a_{\phi{}x } &&\cos\Bigl(&&b_{\phi{}x } &&2\pi x L_x^{-1}&&+ &&c_{\phi{}x }&&\Bigr) &&          &&             &&                &&  &&            &&        &&\cos\Bigl(&&f_{\phi{}x } &&t &&+ &&g_{\phi{}x }&&\Bigr) \notag\\
  &+ &&a_{\phi{}xy} &&\cos\Bigl(&&b_{\phi{}xy} &&2\pi x L_x^{-1}&&+ &&c_{\phi{}xy}&&\Bigr) &&\cos\Bigl(&&d_{\phi{}xy} &&2\pi y L_y^{-1} &&+ &&e_{\phi{}xy}&&\Bigr)  &&\cos\Bigl(&&f_{\phi{}xy} &&t &&+ &&g_{\phi{}xy}&&\Bigr) \notag\\
  &+ &&a_{\phi{}xz} &&\cos\Bigl(&&b_{\phi{}xz} &&2\pi x L_x^{-1}&&+ &&c_{\phi{}xz}&&\Bigr) &&\cos\Bigl(&&d_{\phi{}xz} &&2\pi z L_z^{-1} &&+ &&e_{\phi{}xz}&&\Bigr)  &&\cos\Bigl(&&f_{\phi{}xz} &&t &&+ &&g_{\phi{}xz}&&\Bigr) \notag\\
  &+ &&a_{\phi{}y } &&\cos\Bigl(&&b_{\phi{}y } &&2\pi y L_y^{-1}&&+ &&c_{\phi{}y }&&\Bigr) &&          &&             &&                &&  &&            &&        &&\cos\Bigl(&&f_{\phi{}y } &&t &&+ &&g_{\phi{}y }&&\Bigr) \notag\\
  &+ &&a_{\phi{}yz} &&\cos\Bigl(&&b_{\phi{}yz} &&2\pi y L_y^{-1}&&+ &&c_{\phi{}yz}&&\Bigr) &&\cos\Bigl(&&d_{\phi{}yz} &&2\pi z L_z^{-1} &&+ &&e_{\phi{}yz}&&\Bigr)  &&\cos\Bigl(&&f_{\phi{}yz} &&t &&+ &&g_{\phi{}yz}&&\Bigr) \notag\\
  &+ &&a_{\phi{}z } &&\cos\Bigl(&&b_{\phi{}z } &&2\pi z L_z^{-1}&&+ &&c_{\phi{}z }&&\Bigr) &&          &&             &&                &&  &&            &&        &&\cos\Bigl(&&f_{\phi{}z } &&t &&+ &&g_{\phi{}z }&&\Bigr) \notag
\end{alignat}
where $a$, $b$, $c$, $d$, $e$, $f$, and $g$ are constant coefficient
collections indexed by $\phi$ and one or more directions.  To aid in providing
reusable, physically realizable coefficients for Cartesian domains of arbitrary
size, domain extents $L_x$, $L_y$, $L_z$ have been introduced.
%
Partial derivatives $\phi_{t }$, $\phi_{x }$, $\phi_{y }$, $\phi_{z }$,
$\phi_{xx}$, $\phi_{xy}$, $\phi_{xz}$, $\phi_{yy}$, $\phi_{yz}$, and $\phi_{zz}$
may be computed directly from the chosen solutions.

%Computer algebra systems
%like SymPy~\citep{SymPy} can both compute these derivatives and output C code
%for computing these values at some $x$, $y$, $z$, and $t$.

%\section{Semi-Discrete Manufactured Forcing}

The above solutions are plugged into the model from
\autoref{sec:goveqn} and solved for the forcing terms $Q_{\rho}$,
$\vec{Q}_{\rho{}u}$, and $Q_{\rho{}E}$.  However, solving for these complete
terms entirely within the context of a computer algebra system causes an
explosion of terms.  As the fully expanded forcing terms are too large
to be usable in any meaningful way, they are not shown.
%
Instead, starting from the solution and its the analytic derivatives, basic
calculus followed by algebraic operations performed in floating point are used
to obtain the necessary forcing \emph{at runtime}.
%
Computing the forcing terms looks as follows:
\lstinputlisting[language=python]{mms/forcing.py}
%
The errors arising in this
process behave like standard floating point truncation issues.  Refer to
\citet{Ulerich2012MMS} for a more extended discussion of this approach.

%Many of the
%computations are independent of the constitutive relations used and could be
%employed for other manufactured solutions.


%\section{Suggested Coefficients for Isothermal Channels and Flat Plates}

Employing the manufactured solution requires fixing the more than two hundred
coefficients appearing in the model and chosen solution forms.  Selecting usable
values is not difficult but it is time consuming.
Reasonable coefficient choices for testing channel and flat plate codes
are therefore presented.

In both geometries the streamwise, wall-normal, and spanwise directions are
labeled $x$, $y$, and $z$ respectively.  Both $x$ and $z$ are periodic while
$y\in\left\{0,L_y\right\}$ is not.  Transient tests should likely take place
within the duration $0\leq{}t\leq{}1/10$ nondimensional time units as the time
phase offsets (for example, $g_{Tyz}$) have been chosen for appreciable transients to
occur throughout this time window.

For isothermal channel flow code verification we recommend testing using
\begin{equation*}
  b_{\rho{}y} =
  b_{u{}y}    =
  b_{v{}y}    =
  b_{w{}y}    =
  b_{T{}y}    = \frac{1}{2}
\end{equation*}
and the coefficients given in \autoref{tbl:mmscoeff}.  With these choices
the manufactured solution satisfies isothermal, no-slip conditions at $y = 0,
L_y$.  For isothermal flat plate code verification we recommend testing using
\begin{equation*}
  b_{\rho{}y} =
  b_{u{}y}    =
  b_{v{}y}    =
  b_{w{}y}    =
  b_{T{}y}    = \frac{1}{4}
\end{equation*}
and the coefficients given in \autoref{tbl:mmscoeff}.  With these choices the
manufactured solution satisfies an isothermal, no-slip condition at $y = 0$.

\begin{table}[p]
\caption[Manufactured solution coefficient recommendations]
        {Manufactured solution coefficient recommendations.
         Unlisted coefficients should be set to zero.\label{tbl:mmscoeff}}
\allowdisplaybreaks
\begin{multicols}{4}
\begin{small}
\begin{align*}
\alpha    &= 0          \\
\beta     &= 2/3        \\
\gamma    &= {1.4}  \\
\Mach     &= {1.15} \\
\Prandtl  &= {0.7}  \\
\Reynolds &= 100 \\
L_x       &= 4 \pi \\
L_y       &= 2 \\
L_z       &= 4 \pi / 3 \\
\intertext{}
a_{\rho{}0}  &= 1 \\
a_{\rho{}xy} &= 1 / 11 \\
b_{\rho{}xy} &= 3 \\
d_{\rho{}xy} &= 3 \\
f_{\rho{}xy} &= 3 \\
g_{\rho{}xy} &= \pi / 4 \\
a_{\rho{}y} &= 1 / 7 \\
b_{\rho{}y} &= \text{\emph{See \textsection\ref{sec:mms}}} \\
f_{\rho{}y} &= 1 \\
g_{\rho{}y} &= \pi / 4 - 1 / 20 \\
a_{\rho{}yz} &= 1 / 31 \\
b_{\rho{}yz} &= 2 \\
d_{\rho{}yz} &= 2 \\
f_{\rho{}yz} &= 2 \\
g_{\rho{}yz} &= \pi / 4 + 1 / 20 \\
\intertext{}
a_{uxy} &= 37 / 251 \\
b_{uxy} &= 3 \\
c_{uxy} &= - \pi / 2 \\
d_{uxy} &= 3 \\
e_{uxy} &= - \pi / 2 \\
f_{uxy} &= 3 \\
g_{uxy} &= \pi / 4 \\
a_{uy} &= 1 \\
b_{uy} &= \text{\emph{See \textsection\ref{sec:mms}}} \\
c_{uy} &= -\pi / 2 \\
f_{uy} &= 1 \\
g_{uy} &= \pi / 4 - 1 / 20 \\
a_{uyz} &= 41 / 257 \\
b_{uyz} &= 2 \\
c_{uyz} &= - \pi / 2 \\
d_{uyz} &= 2 \\
e_{uyz} &= - \pi / 2 \\
f_{uyz} &= 2 \\
g_{uyz} &= \pi / 4 + 1 / 20 \\
\intertext{}
a_{vxy} &= 3 / 337 \\
b_{vxy} &= 3 \\
c_{vxy} & = - \pi / 2 \\
d_{vxy} & = 3         \\
e_{vxy} & = - \pi / 2 \\
f_{vxy} &= 3 \\
g_{vxy} &= \pi / 4 \\
a_{vy} &= 2 / 127 \\
b_{vy} &= \text{\emph{See \textsection\ref{sec:mms}}} \\
c_{vy} &= - \pi / 2 \\
f_{vy} &= 1 \\
g_{vy} &= \pi / 4 - 1 / 20 \\
a_{vyz} &= 5 / 347 \\
b_{vyz} &= 2 \\
c_{vyz} &= -\pi / 2 \\
d_{vyz} &= 2 \\
e_{vyz} &= -\pi / 2 \\
f_{vyz} &= 2 \\
g_{vyz} &= \pi / 4 + 1 / 20 \\
\intertext{}
a_{wxy} &= 11 / 409 \\
b_{wxy} &= 3 \\
c_{wxy} &= -\pi / 2 \\
d_{wxy} &= 3 \\
e_{wxy} &= -\pi / 2 \\
f_{wxy} &= 3 \\
g_{wxy} &= \pi / 4 \\
a_{wy} &= 7 / 373 \\
b_{wy} &= \text{\emph{See \textsection\ref{sec:mms}}} \\
c_{wy} &= - \pi / 2 \\
f_{wy} &= 1 \\
g_{wy} &= \pi / 4 - 1 / 20 \\
a_{wyz} &= 13 / 389 \\
b_{wyz} &= 2 \\
c_{wyz} &= - \pi / 2 \\
d_{wyz} &= 2 \\
e_{wyz} &= - \pi / 2 \\
f_{wyz} &= 2 \\
g_{wyz} &= \pi / 4 + 1 / 20 \\
\intertext{}
a_{T0} &= 1 \\
a_{Txy} &= 1 / 17 \\
b_{Txy} &= 3 \\
c_{Txy} &= - \pi / 2 \\
d_{Txy} &= 3 \\
e_{Txy} &= - \pi / 2 \\
f_{Txy} &= 3 \\
g_{Txy} &= \pi / 4 \\
a_{Ty} &= 1 / 13 \\
b_{Ty} &= \text{\emph{See \textsection\ref{sec:mms}}} \\
c_{Ty} &= - \pi / 2 \\
f_{Ty} &= 1 \\
g_{Ty} &= \pi / 4 - 1 / 20 \\
a_{Tyz} &= 1 / 37 \\
b_{Tyz} &= 2 \\
c_{Tyz} &= - \pi / 2 \\
d_{Tyz} &= 2 \\
e_{Tyz} &= - \pi / 2 \\
f_{Tyz} &= 2 \\
g_{Tyz} &= \pi / 4 + 1 / 20 \\
\end{align*}
\end{small}
\end{multicols}
\end{table}

