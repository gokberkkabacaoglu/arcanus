\documentclass{article}
\usepackage{hyperref}
\usepackage{amsmath,amssymb}
\usepackage{graphicx}
\usepackage{caption}
\usepackage{subcaption}
\usepackage{color}
\usepackage[section]{placeins}
%\renewcommand{\thesubsection}{\thesection.\alph{subsection}}
\usepackage{listings}
%\renewcommand{\thesection}{\alph{subsection}}

\title{Combustion Theory Final Take Home Exam} 
\author{Nicholas Malaya\\ Department of Mechanical Engineering \\
University of Texas at Austin}  
\date{}

\begin{document}
\maketitle
\newpage


\section*{Turbulent diffusion flames}

In this problem, I want you to assume that the system is turbulent and
that you know the turbulence mass diffusivity ($D_T$) is 10 times the
laminar value. Assume that the fluctuation squared of the mixture
fraction is equal to the gradient of the mean mixture fraction squared
multiplied by the characteristic diffusion length scale squared, i.e.
\begin{equation}
 \bar{z'z'} = \frac{1}{2} (\nabla \bar z) \frac{D_T L_x}{u}
\end{equation}

\subsection*{a) Write down the solution for the mean mixture fraction field
  with the turbulent diffusivity.}

We start with the steady state species equation, 
\begin{equation}
 \rho u \frac{\partial Y_i}{\partial x} = \rho D \frac{\partial^2
  Y_i}{\partial y^2} \pm \omega_i. 
\end{equation}
We note that, 
\begin{equation}
 \omega = \frac{\omega_i}{\nu_i W_i} = \frac{\omega_F}{\nu_F W_F} =
  \frac{\omega_O}{\nu_O W_O}. 
\end{equation}
This hints at a conserved scalar form of the species equation, where
with, 
\begin{equation}
 \beta  = \frac{\omega_F}{\nu_F W_F} - \frac{\omega_O}{\nu_O W_O}
\end{equation}
then our reaction is decoupled from the convection-diffusion of a
conserved scalar quantity. In particular, 
\begin{equation}
 \mathcal{L}(\beta)  = \mathcal{L}\left(\frac{\omega_F}{\nu_F W_F} -
				   \frac{\omega_O}{\nu_O W_O} \right)
 \Rightarrow \rho u \frac{\partial \beta_i}{\partial x} - \rho D \frac{\partial^2
  \beta_i}{\partial y^2} = 0. 
\end{equation}
Now, we construct z, 
\begin{equation}
 z = \frac{\beta -\beta_{O}}{\beta_F - \beta_{O}}
\end{equation}
Here, the boundary conditions are that $z=1$ for all $y>0$ and $x<0$
(e.g. the fuel reserve) and $z=0$ for $y<0$ and $x<0$ (e.g. the oxygen
reserve). 

Thus, we are solving, 
\begin{equation}
\rho u \frac{\partial z}{\partial x} - \rho D \frac{\partial^2
  z}{\partial y^2} = 0. 
\end{equation}
and, 
\begin{equation}
\rho u \frac{\partial \bar z}{\partial x} - \rho D_T \frac{\partial^2
  \bar z}{\partial y^2} = 0. 
\end{equation}
Where the first equation is from the laminar flow, and the latter case
is the favre-averaged mean field.  

%
%
%
%
\subsection*{b) Plot the fluctuation and mean value of the mixture fraction
  at 5 cm, 30 cm, and 50 cm.}

The mean value mixture fraction was described above. The fluctuation $\bar{z'z'} 
must also be determined. Normally, this would require solving another differential equation, 
and potentially using submodels for the scalar dissipation rate as well. However, we were given a
simplified expression (model) for the variance, namely,
\begin{equation}
 \bar{z'z'} = \frac{1}{2} (\nabla \bar z) \frac{D_T L_x}{u}
\end{equation}


%
%
%
%
\subsection*{c) Plot the PDF of the mixture fraction at two points, $y=0$ cm,
$x=30$ and at $y=15$ cm, $x=30$ cm.}



%
%
%
%
\subsection*{d) Plot the laminar and mean turbulent temperature
distributions at $x=30$ cm.}

\subsection*{e) Discuss the results.}

A few initial comments. The entire formulation utilized chemical equilibrium models 
to find the long time stable solution. The model cannot predict extinction or ignition. 
This also assumes that the chemical time scales are much smaller than the turbulence 
time scales, e.g. that the Damkohler number is large. 

Finally, we either ``turned-on'' or ``turned-off'' the turbulence (laminar). In reality, the flow
might be intermittent or laminar away from the mixing layer at $y=0$ and turbulent near it. 
So a mixing of the models might be more appropriate. 

Given the assumptions, we would have to be careful using the results of 
this model in a predictive context. Those caveats aside, 

\newpage

Thank you for the class! 
\vspace{1in}
\newline
References:

``Turbulent Combustion'', Norbert Peters

``Combustion Physics'', Chung K. Law


\end{document}
