\documentclass{article}
\usepackage{hyperref}
\usepackage{amsmath,amssymb}
\usepackage{graphicx}
\usepackage{caption}
\usepackage{subcaption}
\usepackage{color}
\usepackage[section]{placeins}
\renewcommand{\thesubsection}{\thesection.\alph{subsection}}
\usepackage{listings}

\title{\bf{Combustion Problem Set}}
\author{Nicholas Malaya\\ Department of Mechanical Engineering \\
University of Texas at Austin}  
\date{}

\begin{document}
\maketitle

\newpage
\section*{Problem 6.14}

If we assume a near-unity Lewis number then,
\begin{equation}
\frac{c_p (T_\infty - T_s)}{q_v} = \frac{Y_{i,s} - Y_{i,\infty}}{1-Y_{i,s}}.
\end{equation}
$C_p,T_\infty, q_v$ are given, and $Y_i,\infty=0$. Thus, 
\begin{equation}
\frac{c_p (T_\infty - T_s)}{q_v} = \frac{Y_{i,s}}{1-Y_{i,s}}
\end{equation}
and, 
\begin{equation}
\frac{c_p (T_\infty - T_s)}{q_v} (1-Y_{i,s}) = Y_{i,s}.
\end{equation}
Distributing the temperature term, moving $Y_{i,s}$ to the RHS, and then
solving for $Y_{i,s}$, 
\begin{equation}
\frac{a}{1-a} = Y_{i,s}
\end{equation}
where $a=\frac{c_p (T_\infty - T_s)}{q_v}$. However, we do not know what
$T_s$ is. To solve for $T_s$, we use Clausius-Clapwell:
\begin{equation}
 p = p_{\text{ref}}\text{ exp}\left[-\frac{\bar q_v}{R^0}\left(\frac{1}{T_s}-\frac{1}{T_{\text{ref}}}\right)\right]
\end{equation}
To solve for $T_s$, we take the logarithm of the equation, and perform
several manipulations to arrive at:
\begin{equation}
 T_s = \left(\frac{1}{T_{\text{ref}}} - 1 \right)
  \left(\frac{1}{\text{ln}p - \text{ln }p_{\text{ref}}}\right)
  \frac{\hat q_v}{R^0}. 
\end{equation}
We can now solve for $T_s$, which will then give us $Y_{i,s}$. We also
need the non-dimensional vaporization rate, $\hat m_v$. This is
described by equation 6.4.6 in Law's as, 
\begin{equation}
 B^{\text{dry}}_{m,v} = \frac{Y_{i,s}}{1-Y_{i,s}.}
\end{equation}

%
% ------------------------------------------------------------
%
%
%
%
\newpage
\section*{Problem 6.15}

The $D^2$ law is:
\begin{equation}
 \tau_v = r^2/ K_v.
\end{equation}
The radius size that will be completely burned during the entire stroke
will therefore be, 
\begin{equation}
 r = \sqrt{\tau_v K_v}
\end{equation}
We know that $\tau_v$ is equal to the time for one stroke in the engine,
which is running at 1000 rpm. Thus, $\tau_v$:
\begin{equation}
T = \left(\frac{1 \text{ minute}}{1000}\right) \left(\frac{60
     \text{ sec}}{\text{ minute}}\right).
\end{equation}
$K_v$ is more complicated:
\begin{equation}
K_v = \frac{2 (\lambda/c_p)}{p_l} ln(1+B_{h,v}).
\end{equation}
Where, $B_{h,v}$ is, 
\begin{equation}
\frac{c_p (T_\infty - T_s)}{q_v}
\end{equation}
At this point we have all we need to solve these equations, except that
$\lambda$ was not given. This requires that we assume then that the
Pecle number is $\approx 1$, at which point,
\begin{equation}
(\lambda/c_p) = (\rho D)
\end{equation}
and,
\begin{equation}
 \lambda = (\rho D)c_p
\end{equation}
The evidence for the pecle number being 1 is present in Law on page 208,
but I admittedly do not follow his argument. I would be interested in
hearing more about how this assumption is reasonable.
\newline
Plugging in all the given scenario parameters, we find that the radius
is 
\end{document}
