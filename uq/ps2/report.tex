\documentclass{article}
\usepackage{hyperref}
\usepackage{amsmath,amssymb}
\usepackage{graphicx}
\usepackage{caption}
\usepackage{subcaption}
\usepackage[section]{placeins}
\renewcommand{\thesubsection}{\thesection.\alph{subsection}}
\usepackage{listings}

\title{\bf{CSE397: Group Assignment \#2}}
\author{Benjamin Crestel, Nicholas Malaya, Prabhakar Marepalli, Ajay Vadakkepatt}

 %\\ Institute for Computational Engineering and Sciences \\ University of Texas at Austin} 
\date{}

\begin{document}
\maketitle

\newpage
\section*{Problem Statement}

\section{Uncertainty in the data}
\begin{enumerate}
 \item Ambient (and intermittent) winds
 \item Roughness of basketball and bowling ball (were they unusually
       rough?) (Bowling ball has holes in it)
 \item $C_d$ is varying in time, but we are not accounting for that.
 \item Dimension and weight of the balls:
\begin{itemize}
 \item Basket ball: Circumference and weight don't appear to correspond
       to any official values (Wikipedia + Google). That would indicates
       these characteristics were measured. 
 \item Bowling ball: We can assume the bowling ball was also measured.
\end{itemize}

We are only dealing with measurement errors here. We can model both with
       additive Gaussian noise of mean zero and standard deviation
       defined by a 95\% confidence interval. We could assume the
       circumference is correct within $\pm 1$ cm with 95\%
       confidence. Whereas the weight is accurate within 0.1 oz. 
 \item Height of the drop: This is represented by a uniform distribution
       between 34.5m and 35.5m. 
 \item Position of the operator: The position is very uncertain. But it
       doesn't matter in the outcome of the experiment. As long as the
       operator is far enough to capture the whole drop sequence without
       having to move (which he is at 22.4m away from the action), it is
       not going to have an incidence on the results. 
 \item Temperature: Mass density of the air depends on the
       temperature. Both experiments were performed in two different
       days. So most likely the temperature was different which affected
       the mass density of the air. We can assume the temperature was
       between 15C~(1.2250~kg/m$^3$) and 25C~(1.2041~kg/m$^3$) with 95\%
       confidence. We will directly translate this uncertainty into the
       mass density. 
 \item Video acquisition: The time framing is assumed to be exactly
       accurate as this is based on the iPad internal clock and there is
       no reason to assume it was defective. However the distance of the
       ball is very uncertain for different reasons. 
\begin{itemize}
 \item Reference distance: The distances are all relative to a reference
       distance. The experimentalists chose the height of the bridge as
       their reference distance. But on top of the large uncertainties
       on that exact distance (see above), we have uncertainties coming
       from the definition of that length in Logger Pro. 
 \item Fixed position of the camera: The acquisition of the motion of
       the ball is done by selecting the ball (or the top of the ball to
       more accurate) on the screen at each frame you want to
       use. However, based on online videos, Logger Pro doesn't adjust
       the position of the object relative to its environment. So the
       position of the object is relative of the sides of the image (ie,
       the camera is assumed to be fixed in Logger Pro). Hence if the
       camera operator shifts the camera during the experiment he shifts
       the absolute position of the ball by the same amount. If he
       rotates the iPad, he shifts the absolute position of the ball by
       the distance between the camera and the ball times the rotation
       angle. That would indicate that the distance from the ball is
       actually relevant. Unless we adopt some general treatment as
       advised in the next section. 
\end{itemize}
 \item Gravity of Earth: From Wikipedia (actually using Wolfram Alpha's
       city data), we found the gravity at Vancouver to be
       $9.81339$~m/s$^2$. The uncertainty will be estimated below. 
\end{enumerate}

\section{Uncertainty on the position of the ball}

Working at the continuous level for now, we are interested in
characterizing the uncertainty on the measured position of the ball
$\tilde{h}(t)$.  
We model this uncertainty with additive noise $\varepsilon(t)$ such that 
\begin{align*}
 \tilde{h}(t) = h(t) + \varepsilon(t).
\end{align*}
Assuming we can estimate the mean and covariance matrix of that random
variables, the principle of maximum entropy relative to a uniform
distribution would lead us to using a Gaussian distribution for that
noise. 
In the question we are advised to use the following form for the
mean~$\mu(t)$ and covariance matrix~$R(t,t')$ of the noise 
\begin{align*}
 \mu(t) = \alpha t , \quad R(t,t') = \sigma^2 e^{-|t-t'|/\tau} .
\end{align*}

In practice for each experiment we are given a discrete set of times~$\{
t_i \}_i$ when position measurements occurred. This leads to the
following discrete representation of uncertainty for the measured vector
position 
\begin{align*}
 \{ \tilde{h} \} = \{ h \} + \{ \varepsilon \},
\end{align*}
where
\begin{align*}
 \{ \varepsilon \} \sim \mathcal{N}\left( \{ \mu \} ,  R  \right),
\end{align*}
with
\begin{align*}
 \mu_i = \alpha t_i, \quad R_{ij} = \sigma^2 e^{-|t_i-t_j|/\tau}.
\end{align*}

\section{Prior on $C_d$, the coefficient of drag}

The coefficient of drag is a non-dimensionalization of the force that
results from a scaling analysis. 
\begin{equation*}
 \frac{F_D}{\frac{1}{2} \rho V^2 A } = C_D(Re).
\end{equation*}
Thus, the coefficient depends only on the Reynolds number of the
flow. Let's estimate the Reynolds number to give us a sense of the range 
we believe may be possible in this particular experiment. 

The Reynolds number is defined as, 
\begin{equation*}
 Re = \frac{V L}{\nu}
\end{equation*}
Let's estimate the characteristic velocity (V), and length (L) as well
as the viscosity, $\nu$. An upper bound for the velocity would be from
the calculating the velocity in the absense of drag. Here, $V = g
t_{\text{final}}$, and with $t_{\text{final}} \approx 3 $ seconds, our V
is $\approx 29.4 $ m/s. The characteristic length is the ball diameter,
which for the basketball is 23.33 cm. Finally, the kinematic viscosity
of air is $15.68 * 10^{-6} \text{m}^2/s$ according to
\url{http://www.engineeringtoolbox.com/air-properties-d_156.html}. Thus,
our $\text{Re} \approx 437,000$, which means the flow is certainly
turbulence. Furthermore, we expect roughness effects to become
significant around $Re \approx 10^5$. Given that we do not know, with
confidence, if the balls are rough or smooth, our prior must be
sufficiently broad to capture the possibilty of either condition. 

Looking at a NASA chart for the measured drag coefficient on a sphere, 
at low reynolds numbers (when we release the ball) we expect that our
drag coefficient might be as large as 2. Should the sphere be smooth,
the drag coefficient could be as small as 0.1 at the end. We also know
that our prior should not be capable of negative values, that would be
aphysical. For these reasons, we decided to define $C_d$ as
\begin{align*}
 C_d \sim \mathcal{U}(0,2).
\end{align*}


\section{Probabilistic model}

\begin{enumerate}
 \item The coefficient $\alpha$ defined above is said to be ``of order
       $0.5$~m/s with uncertainty of the same order''.  
 Since we could not exclude negative values of $\alpha$ we can choose
       $\alpha$ to be Gaussian with mean $0.5$~m/s and $95\%$~confidence
       interval given by the interval of uncertainty. Here the
       uncertainty is said to be ``of the same order'' as the mean. In
       order to be conservative, we'll pick our $67\%$~confidence
       interval to be $[-1,1]$~m/s.  
\[ \alpha \sim \mathcal{N}(0.5, 1). \]
 \item For $\sigma$, we know it has to be non-negative. We can therefore
       assume that $\sigma$ is uniformly distributed over the interval
       $[0, 0.2]$~m. 
\[ \sigma \sim \mathcal{U}(0, 0.2) . \] 
 \item The uncertainty on the height will impact the
       measurement. Assuming for a second that we have no errors due to
       the handling of the iPad, the height we measure is given
       proportionally to the reference height of the bridge and the
       ratio of the pixels for each object, that is 
 \begin{align*}
  \tilde{h} = \frac{p_1}{p_2} \tilde{H},
 \end{align*}
where $\tilde{H}$ is the measured height of the bridge. As discussed
       earlier, the height of the bridge is a random variable sampled
       from a uniform distribution over $[34.5, 35.5]$~m. 
For convenience we re-write $\tilde{H}$ as
\begin{align*}
 \tilde{H} = 35 + \tilde{\eta},
\end{align*}
where $\tilde{\eta}$ is sampled from a uniform distribution over $[-0.5,
       0.5]$~m. Plugging this into the equation for our measured height
       of the ball, we get 
\begin{align*}
 \tilde{h} = \frac{p_1}{p_2} 35 + \frac{p_1}{p_2} \tilde{\eta}. 
\end{align*}
Let's define the exact height of the ball as 
\begin{align*}
 h \equiv \frac{p_1}{p_2} 35.
\end{align*}
We then get
\begin{align*}
 \tilde{h} = h \left( 1 + \frac{\tilde{\eta}}{35} \right).
\end{align*}
That is the uncertainty due to the height of the bridge acts on the
       measured height of the ball through a multiplicative noise.  
Using the notations we defined in the previous questions for the
       uncertainty due to the handling of the iPad, we can write the
       measured height of the ball as 
\begin{align*}
 \tilde{h} = \left( h + \varepsilon \right) (1 + \eta),
\end{align*}
where $\varepsilon$ is a multivariate Gaussian with mean and covariance
       matrix defined in earlier and  
\begin{align*}
 \eta \sim \mathcal{U}\left( \frac{-0.5}{35}, \frac{+0.5}{35} \right) .
\end{align*}
Maybe a more informative way to formulate that problem would be to
       define the total height of the bridge directly as a random
       variable. This leads to the following data reduction model for
       the measured height of the ball 
\begin{align*}
 \tilde{h} = ( h  + \varepsilon ) \frac{H}{35},
\end{align*}
where
\begin{align*}
 H \sim \mathcal{U} \left( 34.5, 35.5 \right).
\end{align*}
\end{enumerate}


\section{Likelihood and MCMC computations}

We will infer (at most) for the following parameters: $\alpha, \sigma,
C_d, H$. The likelihood gives us the probability that the results from
our simulation~$h_c$ match the data~$h_d$ we have. 
\[ \pi\left( h_c = h_d \, | \, \alpha, \sigma, C_d, H \right) .\] 
In the hypothetical case we'd know all parameters exactly ($\alpha,
\sigma, C_d, H$), the data would be connected to the results of our
simulation through the following relation 
\begin{align*}
 h_d = \left( h^\text{ode} + \varepsilon \right) \frac{H}{35} , 
\end{align*}
where $h_c$ is given by
\begin{align*}
 h_c = h^\text{ode} \frac{H}{35}.
\end{align*}
For convenience, we re-write this relation as
\begin{align*}
\left( h_d - h_c \right)\frac{35}H  = \varepsilon.
\end{align*}
We already detailed the distribution of the random variable
$\varepsilon$. It is Gaussian with mean and covariance given by the time
measurements of each experiment and random variables $\alpha$ and
$\sigma$ (which are assumed to be sampled when evaluating the
likelihood). Calling $\mu$ and $R$ the mean and covariance matrix
corresponding to the right format of $h$ (ie, stacking information from
all 4 experiments 'on top of each other'), we get the following
distribution for the likelihood, 
\begin{align*}
 \pi(h_c = h_d \, | \, \alpha, \sigma, C_d, H) = \frac1{\sqrt{(2\pi)^n |R|}} \exp \left\{ - \frac12 \left(  h_d \frac{35}H - h^\text{ode} - \mu \right)^T \cdotp R^{-1} \cdotp \left(  h_d \frac{35}H - h^\text{ode} - \mu \right) \right\} ,
\end{align*}
where $n = t_f^1 \times t_f^2 \times t_f^3 \times t_f^4$ is the number
of data (ie, the dimension of $h_d$) and $h_d$ is defined as 
\begin{align*}
 h_d = \begin{bmatrix}
h_1(t_1^1) \\
h_1(t_2^1) \\
\vdots \\
h_1 (t_f^1) \\
h_2(t_1^2) \\
\vdots \\
h_2(t_f^2) \\
\vdots \\
h_4(t_f^4) 
\end{bmatrix}.
\end{align*}
Note that the times used in $h_d$ (and hence $h_c, \mu, R$) can be a
subset of the times used in the actual experiments. For instance, the
last measurements look highly suspicious and should be probably excluded
from our inference. 
$\mu$ and $R$ are defined as follows 
\begin{align*}
 \mu = \alpha
 \begin{bmatrix}
  t_1^1 \\
  t_2^1 \\
  \vdots \\
  t_f^1 \\
  t_1^2 \\
  \vdots \\
  t_f^4
 \end{bmatrix}, \quad R = 
 \begin{bmatrix}
  R_1 & 0 & 0 & 0 \\
  0 & R_2 & 0 & 0 \\
  0 & 0 & R_3 & 0 \\ 
  0 & 0 & 0 & R_4
 \end{bmatrix}
\end{align*}
where
\begin{align*}
 R_i = \sigma^2
 \begin{bmatrix}
  1 & e^{-|t_1^i - t_2^i|/\tau} & e^{-|t_1^i - t_3^i|/\tau} &  \ldots & e^{-|t_1^i - t_{f-1}^i|/\tau} & e^{-|t_1^1-t_f^i|/\tau} \\
\vdots & 1 & e^{-|t_2^i - t_3^i|/\tau} &   \ldots & e^{-|t_2^i - t_{f-1}^i|/\tau} & e^{-|t_2^i - t_f^i|/\tau} \\
\\
\vdots &  &  \ddots & &  & \vdots \\
\\
& \text{symm} & & & 1 & e^{-|t_{f-1}^i - t_f^i|/\tau}  \\
\\
& & \ldots & & & 1
 \end{bmatrix}
\end{align*}
Because of the very specific form of our covariance matrix (block-diagonal), we can simplify the expression for the likelihood as follows
\begin{align*}
 & \pi(h_c = h_d \, | \, \alpha, \sigma, C_d, H) =  \ldots \\
 & \frac1{\sqrt{(2\pi)^n \prod_{i=1}^4|R_i|}}  \exp \left\{ - \frac12 \sum_{i=1}^4 \left(  h_d \frac{35}H - h^\text{ode} - \mu \right)_i^T \cdotp R_i^{-1} \cdotp \left(  h_d \frac{35}H - h^\text{ode} - \mu \right)_i \right\} ,
\end{align*}
where 
\begin{align*}
 \left(  h_d \frac{35}H - h^\text{ode} - \mu \right)_i
\end{align*}
corresponds to the data, computed results and mean for the $i^\text{th}$ experiment. This formulation clarifies the fact that we work experiment per experiment as each experiment is independent from the others.





\end{document}
