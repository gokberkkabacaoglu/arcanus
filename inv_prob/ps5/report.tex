\documentclass[11pt]{article}
\usepackage{graphicx,amsmath,amsfonts,amssymb,graphicx} 
\usepackage[varg]{txfonts}
\usepackage{enumerate}
\usepackage{hyperref}
\usepackage{listings}
\usepackage{color}
\urlstyle{tt}

\usepackage{geometry}
\geometry{%
  letterpaper,
  lmargin=2cm,
  rmargin=2cm,
  tmargin=2cm,
  bmargin=2cm,
  footskip=12pt,
  headheight=12pt}
  
\usepackage{lastpage}
\usepackage{fancyhdr}
%\pagestyle{fancy}
%\headheight 35pt

\def\squarebox#1{\hbox to #1{\hfill\vbox to #1{\vfill}}}
\def\qed{\hspace*{\fill}
        \vbox{\hrule\hbox{\vrule\squarebox{.667em}\vrule}\hrule}}
\newenvironment{solution}{\begin{trivlist}\item[]{\bf Solution:}}
                      {\end{trivlist}}
\lstset{
	language=MATLAB,              % choose the language of the code ("language=Verilog" is popular as well)
   tabsize=3,							  % sets the size of the tabs in spaces (1 Tab is replaced with 3 spaces)
	basicstyle=\tiny,               % the size of the fonts that are used for the code
	numbers=left,                   % where to put the line-numbers
	numberstyle=\tiny,              % the size of the fonts that are used for the line-numbers
	stepnumber=2,                   % the step between two line-numbers. If it's 1 each line will be numbered
	numbersep=5pt,                  % how far the line-numbers are from the code
	%backgroundcolor=\color{mygrey}, % choose the background color. You must add \usepackage{color}
	%showspaces=false,              % show spaces adding particular underscores
	%showstringspaces=false,        % underline spaces within strings
	%showtabs=false,                % show tabs within strings adding particular underscores
	frame=single,	                 % adds a frame around the code
	tabsize=3,	                    % sets default tabsize to 2 spaces
	captionpos=b,                   % sets the caption-position to bottom
	breaklines=true,                % sets automatic line breaking
	breakatwhitespace=false,        % sets if automatic breaks should only happen at whitespace
	%escapeinside={\%*}{*)},        % if you want to add a comment within your code
	%commentstyle=\color{BrickRed}   % sets the comment style
}                    
\begin{document}

\title{\bf{CSE397: Assignment \#5}}
\author{Nicholas Malaya \\ Department of Mechanical Engineering \\
Institute for Computational Engineering and Sciences \\ University of
Texas at Austin} \date{} 
\maketitle
\newpage

\subsection*{Problem 1: An inverse problem for Burgers' Equation}

\begin{enumerate}
\item[(1)] Derive a weak form. Use integration-by-parts on the viscous
	   term to derive the weak form of Burgers' equation. 

\begin{solution}
Starting with Burgers' equation in strong form, 
\begin{align}
 u_t + u u_x - \nu u_{xx} = f, 
\end{align}
we multiply by a test function, $p(t,x)$ and integrate over time and space, 
\begin{align}
 \int_0^T \int_0^L (u_t p + u u_x p - \nu u_{xx} p - f p) dx dt = 0. 
\end{align}
Now, we integrate the viscous term by parts, to move a derivative of x
 onto the test function, 
\begin{align}
 \int_0^T \int_0^L (u_t p + u u_x p + \nu u_{x} p_x - f p) dx dt -
 \int_0^T \nu u_x p \bigg|_{x=0}^{x=T} dt = 0. 
\end{align}
Notice that we have introduced a boundary term (and changed the sign of
 the convective operator) by doing this. However,
 $p(0,x)=0$, and therefore this term vanishes. Thus, 
\begin{align}
 \int_0^T \int_0^L (u_t p + u u_x p + \nu u_{x} p_x - f p) dx dt = 0
\end{align}
is our weak form.  
\end{solution}

\item[(2)] Using the Lagrangian functional, derive expressions for the
	   adjoint equation and for the gradient of J with respect to
	   $\nu$. Give weak and strong forms of these equations. 
\begin{solution}

The Largrangian functional is given by: 
\begin{equation}
\mathcal{L} = \int_0^T \int_0^L (u_t p + u u_x p + \nu u_{x} p_x - f p) dx dt
+ \frac{1}{2}\int_{T_1}^T\int_0^L(u-u^{\text{obs}})^2dxdt  
+ \frac{\beta}{2} \int_0^L \nu_x \nu_x dx  
\label{lagrangian}
\end{equation}
Varying this with respect to the test function, $p$, will yield (1) back
 in weak and strong form. Varying with respect to $u$ will return the
 weak form of the adjoint equation which holds for all $\tilde{u}$ which
 obey the same boundary conditions as $u$ (we must integrate $\tilde u_t$
 and $\tilde u_x$ by parts to accomplish this):   
\begin{equation}
\delta_u \mathcal{L} = \int_0^T\int_0^L\left(-\tilde{u} p_t +
					u\tilde{u}_x p + \tilde{u}u_x p  
		 - \nu\tilde{u} p_{xx}\right)dxdt  
+ \int_{T_1}^T\int_0^L(u-u^{\text{obs}})\tilde{u}dxdt = 0.
\end{equation}
As one would expect, this moves backward in time. 


 Now, in order to solve for the strong form, we now need to combine the
 integration from $T_1$ to $T$ with the integral from zero to $T$. In
 order to do this, we multiply the second integral by a function that is
 zero below $T_1$, and 1 past it. This is actually the ``Heavyside
 function''. The resulting equation is then, 
\begin{equation}
\delta_u \mathcal{L} = \int_0^T\int_0^L \left( -\tilde{u} p_t -
					\tilde{u}(u p)_x + \tilde{u}u_x p  
		 - \nu\tilde{u} p_{xx}
+ H(t-T_1)(u-u^{\text{obs}})\tilde{u} \right)dxdt = 0.
\end{equation}
Now, we move $\tilde u$ out of the parenthesis, 
\begin{equation}
\delta_u \mathcal{L} = \int_0^T\int_0^L \tilde u \left( - p_t -
					(u p)_x + u_x p  
		 - \nu p_{xx}
+ H(t-T_1)(u-u^{\text{obs}}) \right) dxdt = 0.
\end{equation}
Remember that $\tilde u$ is arbitrary. Everything in the parenthesis above must
 therefore be equal to zero, and our strong form is:
\begin{equation}
- p_t - (u p)_x + u_x p - \nu p_{xx} + H(t-T_1)(u-u^{\text{obs}}) = 0.
\end{equation}

\end{solution}
\end{enumerate}

Finally, the variation with respect to $\nu$ is, 
\begin{equation}
 \delta_\nu \mathcal{L} = \frac{\beta}{2} \int^L_0 \nu_x \tilde \nu_x dx +
  \int_0^T \int_0^L \tilde \nu u_x p_x dx dt. 
\end{equation}
This is the weak form of the gradient. In order to arrive at the strong
form, we perform the simple judo kata of moving the differential
operator from $\tilde v_x$, 
\begin{equation}
  \int_0^T \int_0^L \tilde \nu u_x p_x dx dt -\frac{\beta}{2} \int^L_0
   \nu_{xx} \tilde \nu dx.
\end{equation}
Next, move $\beta$ into the integral, and pull out the time integration, 
\begin{equation}
 \int_0^T \left[ 
	   \int_0^L \tilde \nu u_x p_x dx -\frac{\beta}{2} \int^L_0
	   \nu_{xx} \tilde \nu dx 
       \right] dt
\end{equation}
finally, combine the spatial integration operators, 
\begin{equation}
 \int_0^T \left[ 
	   \int_0^L \tilde \nu \left(u_x p_x - \frac{\beta}{2} \nu_{xx}
			       \right) dx \right] dt.
\end{equation}
Now, we have isolated $\tilde \nu$, which can be arbitrary. Thus, the
terms in the parenthesis must be equal to zero, and the strong form
appears as: 
\begin{equation}
 u_x p_x - \frac{\beta}{2} \nu_{xx} = 0.
\end{equation}
%
%
%
%
%
%

\newpage
\subsection*{Problem 2}
\textbf{Inverse elliptic parameter estimation, continued from assignment
4.} We solve 
the inverse problem for the advection-diffusion equation on $\Omega =
[0, 1] \times [0, 1]$: 
\begin{equation}
\min_a J(m) = \frac{1}{2}\int_\Omega(u-u^{\text{obs}})^2dx +
 \frac{\beta}{2}\int_\Omega (\nabla m \cdot \nabla m)dx \tag{8}
 \label{2cost} 
\end{equation}
where $u$ is the solution of
\begin{align}
-\nabla\cdot(m\nabla u) + v\cdot\nabla u &= f \text{ in } \Omega
 \\ 
u &= 0 \text{ on } \partial \Omega 
\end{align}
with the advective velocity $v = (v_1 , v_2 )$, regularization parameter
$\beta > 0$ and measurement data $u^{\text{obs}}$ , which are obtained by
solving the state equation with $m(x, y) = 2$ for $(x - 0.5)^2 + (y -
0.5)^2 \le 0.04$ and $m(x, y) = 8$ otherwise (and adding noise). 
\begin{enumerate}
\item[(a)]Find an optimal regularization parameter $\beta$ found from the
	  discrepancy criterion.

\begin{solution}
yar
\end{solution}


\item[(b)]Extend the COMSOL implementation elliptic\_GN\_ip of the
	  Gauss-Newton method using the advection velocity
	  $v=(30,0)$. Report the number of Gauss-Newton and of overall
	  CG iterations for a discretization of the domain with $10
	  \times 10$, $20 \times 20$, $40 \times 40$ and $80 \times 80$
	  linear finite elements and give the number of unknowns used to
	  discretize the coefficient function $m$ (which is called a in
	  the implementation) for each of these meshes. Discuss how the
	  number of iterations changes as more parameters are used. 

\begin{solution}
yar
\end{solution}

\item[(c)]To avoid over-solving of the CG system in early Newton steps,
	  where we are still far away from the solution and thus cannot
	  benefit from the fast local convergence properties of Newton's
	  method, the implementation uses the stopping
	  criterion. Discuss the behavior.  


\begin{solution}
yar
\end{solution}

\item[(d)]The ill-posedness of inverse problems is closely related to
	  the spectrum (i.e., the eigenvalues) of the involved
	  operators. Compute the eigenvalues of the reduced Hessian at
	  the solution of the problem for a mesh with $20 \times 20$
	  elements. Since the Hessian is not explicitly available and
	  can only be applied to vectors, there are 2 possibilities to
	  access its eigenvalues: 
\begin{enumerate}
\item[$\bullet$] Build the Hessian matrix explicitly by applying it to
	     unit vectors and compute the eigenvalues of the resulting
	     matrix. 
\item[$\bullet$] Luckily, there are iterative methods to compute
	     eigenvalues of a matrix, which only requires the
	     application of the matrix to vectors. Compute the largest
	     100 eigenvalues using an iterative method. 
\end{enumerate}
Plot the spectrum of the reduced Hessian with and without regularization 
	  and discuss the result. 

\item[(e)]\textit{Optional:}Replace the Tikhonov regularization by total
	  variation regularization and report the results for different
	  meshes. 
\end{enumerate}

\newpage
\subsection*{Code}
%Here is the code for part c:
%\lstinputlisting{code/elliptic_sd_ip_adv_TV.m}
\end{document}