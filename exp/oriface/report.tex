\documentclass{article}
\usepackage{amsmath,amssymb}
\usepackage{hyperref}
\usepackage{graphicx}

\title{\bf{Laboratory Project Two: Design an Oriface Meter Calibration Rig}}
\author{Nicholas Malaya \\ Department of Mechanical Engineering \\ University of Texas at Austin} \date{}

\begin{document}
\maketitle
\date{}

You are to design a test facility to calibrate an orifice meter. The orifice plate has a diameter of
1.8227 in and is installed in nominally a 4 in diameter PVC pipe. The orifice meter is to be used 
to measure flow rates from 5 to 50 cfm. For calibration of the orifice meter you will use a 
laminar flow meter with 0 to 160 cfm and an accuracy of $\pm 1\%$ of the reading. Assume the 
pressure transducers to be used with the orifice meter and laminar flow element meter have an 
accuracy of $\pm 0.02$ in H2O.

\subsection*{A}
\textbf{Assuming a nominal discharge coefficient for the orifice meter of $C_d = 0.62$, what is the
expected range of pressure drop for the orifice meter for the range of flow rates that are to be 
measured?}

Stavros provides an equation for the flow rate as a function of the
pressure drop on page 211 of his text as,  
\begin{equation}
 Q = C_d \frac{\pi d^2 / 4}{\sqrt{1-(d/D)^4}}\sqrt{\frac{2 \Delta p}{\rho}}.
\end{equation}



\subsection*{B}
\textbf{Describe how you would calibrate the orifice meter giving particular attention to what you
would do maximize the accuracy of the final calibration for the orifice meter. For this 
calibration rig, assume that a blower is available that provides a maximum flow rate of 100 
cfm and is reasonably quiet.}

\subsection*{C}
\textbf{Estimate the expected uncertainty for the orifice meter flow measurement after the orifice has
been calibrated. Give separate values for the precision and bias uncertainties.}


\end{document}
