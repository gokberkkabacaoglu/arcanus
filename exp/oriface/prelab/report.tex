\documentclass{article}
\usepackage{amsmath,amssymb}
\usepackage{hyperref}
\usepackage{graphicx}

\title{\bf{Laboratory Project Two: Design an Oriface Meter Calibration Rig}}
\author{Nicholas Malaya \\ Department of Mechanical Engineering \\ University of Texas at Austin} \date{}

\begin{document}
\maketitle
\date{}

You are to design a test facility to calibrate an orifice meter. The orifice plate has a diameter of
1.8227 in and is installed in nominally a 4 in diameter PVC pipe. The orifice meter is to be used 
to measure flow rates from 5 to 50 cfm. For calibration of the orifice meter you will use a 
laminar flow meter with 0 to 160 cfm and an accuracy of $\pm 1\%$ of the reading. Assume the 
pressure transducers to be used with the orifice meter and laminar flow element meter have an 
accuracy of $\pm 0.02$ in H2O.

\subsection*{A}
\textbf{Assuming a nominal discharge coefficient for the orifice meter of $C_d = 0.62$, what is the
expected range of pressure drop for the orifice meter for the range of flow rates that are to be 
measured?}

Stavros provides an equation for the flow rate as a function of the
pressure drop on page 211 of his text as,  
\begin{equation*}
 Q = C_d \frac{\pi d^2 / 4}{\sqrt{1-(d/D)^4}}\sqrt{\frac{2 \Delta p}{\rho}}.
\end{equation*}
From reading, $d = 1.8227$ inches, $D = 4 $ inches, $C_d = 0.62$, $\rho
= 0.074887$, we find that we expect a rather large pressure range,
between:

\begin{eqnarray*}
  0.3827  \text{ psi} \\
 38.2748  \text{ psi}
\end{eqnarray*}

Atmospheric is $\approx 14.6$ psi, so we are dealing with up to 2.6X
larger values. 


\subsection*{B}
\textbf{Describe how you would calibrate the orifice meter giving
particular attention to what you would do maximize the accuracy of the
final calibration for the orifice meter. For this calibration rig,
assume that a blower is available that provides a maximum flow rate of
100 cfm and is reasonably quiet.} 

If I was absolutely certain that I only needed to measure values between the
ranges of 0-50 cfm, then I would only calibrate the oriface meter on
that range of values. If you calibrate for values only inside this
range, you will necessarily find the best fit for values lying in this
range. 

\subsection*{C}
\textbf{Estimate the expected uncertainty for the orifice meter flow
measurement after the orifice has been calibrated. Give separate values
for the precision and bias uncertainties.} 

Bias is tricky here. We have a laminar flow meter with an accuracy of
$\pm 1\%$. Thus, I expect that ``the best we can do'' is to get our two
readings (between the oriface meter and the laminar flow meter) to agree
to within 1\% at 0 cfm. Even then, the reading may be off (biased) by
1\%. 

Precision is arrived at using the more common precision uncertainty
estimates, and requires the forward propagation of error through the
same equation we used previously. As this equation has \emph{numerous}
parameters that almost certainly have uncertainties attached, the actual
uncertainty could be substantial. However, outside of the parametric
uncertainty associated with $d,D,C_d,$ etc. We know that the meter has
1\% accuracy in the reading. This term is squared when solving for
$\Delta p$, and so $(1.01)^2 = 1.0201 $, e.g. I expect 2\% error
to propagate to the oriface entirely due to a 1\% error in measurement
from the laminar flow meter. 

\end{document}
